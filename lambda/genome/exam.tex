\chapter{Practice Exam}
\label{ch:exam-practice}

\begin{preamble}
Good luck!
\end{preamble}

\begin{cluster}[True or False]

\begin{gram}
Select for each question whether it is true or false.
\end{gram}

\begin{parts}
%%%%%%%%%%%%%%%%%%%%%%%%%%%%%%%%%%%%
\begin{problem}[blah][2]
Say you take a deterministic C program and make it
parallel by converting some \texttt{for} loops to \texttt{cilk\_for},
and adding some \texttt{cilk\_spawn} and \texttt{cilk\_sync} commands.
Assuming the resulting program has no race conditions, it always
returns the same result as the original sequential program.

\begin{pickone}
\choice True
\correctchoice False
\end{pickone}

\end{problem}

%%%%%%%%%%%%%%%%%%%%%%%%%%%%%%%%%%%%
\begin{problem}[2]
Concurrency is just another word for parellelism.  
\begin{pickone}
\choice True
\correctchoice False
\end{pickone}
\end{problem}

%%%%%%%%%%%%%%%%%%%%%%%%%%%%%%%%%%%%
\begin{problem}[2]
A race condition can occur both in purely functional and non-purely
functional parallel programs.
\begin{pickone}
\choice True
\correctchoice False
\end{pickone}
\end{problem}

%%%%%%%%%%%%%%%%%%%%%%%%%%%%%%%%%%%%
\begin{problem}[2]
Locks are the only way to prevent race conditions.
\begin{pickone}
\choice True
\correctchoice False
\end{pickone}
\end{problem}

%%%%%%%%%%%%%%%%%%%%%%%%%%%%%%%%%%%%
\begin{problem}[2]

When executing a parallel algorithm, we organize the computation into
collection of threads, which are then mapped to the processors by a
thread scheduler.

\begin{pickone}
\correctchoice True
\choice False
\end{pickone}
\end{problem}

%%%%%%%%%%%%%%%%%%%%%%%%%%%%%%%%%%%%
\begin{problem}

When executing a parallel algorithm, we organize the computation into
collection of threads and assign the threads to processors by using a
thread scheduler.

\begin{pickone}
\correctchoice True
\choice False
\end{pickone}
\end{problem}

%%%%%%%%%%%%%%%%%%%%%%%%%%%%%%%%%%%%
\begin{problem}[2]

A fork-join parallel computation can be represented as a DAG of
threads.

\begin{pickone}
\correctchoice True
\choice False
\end{pickone}
\end{problem}


%%%%%%%%%%%%%%%%%%%%%%%%%%%%%%%%%%%%
\begin{problem}[2]

Centralized schedulers that maintain a single pool of threads to
execute can perform well in practice, partly because they can balance
the total load relatively evenly between existing processors.

\begin{pickone}
\choice True
\correctchoice False
\end{pickone}
\end{problem}

%%%%%%%%%%%%%%%%%%%%%%%%%%%%%%%%%%%%
\begin{problem}[2]

Distributed schedulers are perfectly greedy schedulers.

\begin{pickone}
\choice True
\correctchoice False
\end{pickone}
\end{problem}

%%%%%%%%%%%%%%%%%%%%%%%%%%%%%%%%%%%%
\begin{problem}

Distributed schedulers are not perfectly greedy schedulers, but they
still perform well because they can reduce contention. 

\begin{pickone}
\correctchoice True
\choice False
\end{pickone}


\end{problem}


%%%%%%%%%%%%%%%%%%%%%%%%%%%%%%%%%%%%
\begin{problem}[2]

We can measure the work efficiency of a parallel algorithm by
measuring the running time (work) of the algorithm on a single core,
divided by the running time (work) of the sequential elision of the
algorithm.

\begin{pickone}
\choice True
\correctchoice False
\end{pickone}

\end{problem}
 
%% Comparison should be with respect to the best baseline.


%%%%%%%%%%%%%%%%%%%%%%%%%%%%%%%%%%%%
\begin{problem}[2]

We can improve the work efficiency of a parallel algorithm by using
granularity control.

\begin{pickone}
\correctchoice True
\choice False
\end{pickone}


\end{problem}

%%%%%%%%%%%%%%%%%%%%%%%%%%%%%%%%%%%%
\begin{problem}[2]
One way to improve the empirical work efficiency and thus the speedup
of a parallel algorithm is to switch to a serial algorithm when the
problem size is small. This is called granularity control.


\begin{pickone}
\correctchoice True
\choice False
\end{pickone}

\end{problem}

%%%%%%%%%%%%%%%%%%%%%%%%%%%%%%%%%%%%
\begin{problem}[2]

In class, you have learned of a technique for controlling granularity
automatically by using asypmtotic cost annotations specifying the span
of the  computation to be controlled.


\begin{pickone}
\correctchoice True
\choice False
\end{pickone}

\end{problem}

%%%%%%%%%%%%%%%%%%%%%%%%%%%%%%%%%%%%
\begin{problem}[2]

In class, you have learned of a technique for controlling granularity
automatically by using asypmtotic cost annotations specifying the work
of the  computation to be controlled.

\begin{pickone}
\choice True
\correctchoice False
\end{pickone}

\end{problem}

%%%%%%%%%%%%%%%%%%%%%%%%%%%%%%%%%%%%
\begin{problem}[2]

Atomic read-modify-write operations such as compare-and-swap aim to
provide for performing multiple instructions such as loading from a
memory location and writing into that memory atomically.

\begin{pickone}
\correctchoice True
\choice False
\end{pickone}


\end{problem} 


%%%%%%%%%%%%%%%%%%%%%%%%%%%%%%%%%%%%
\begin{problem}[2]

Some atomic read-modify-write operations such as compare-and-swap
suffer from the ABA problem.

\begin{pickone}
\correctchoice True
\choice False
\end{pickone}


\end{problem} 

%%%%%%%%%%%%%%%%%%%%%%%%%%%%%%%%%%%%
\begin{problem}[2]

Some atomic read-modify-write operations such as
load-link/store-conditional can prevent the ABA problem.

\begin{pickone}
\correctchoice True
\choice False
\end{pickone}
\end{problem}



%%%%%%%%%%%%%%%%%%%%%%%%%%%%%%%%%%%%
\begin{problem}[2]

The ``ABA problem'' was nomed after the Swedish pop band ``ABBA''.

\begin{pickone}
\choice True
\correctchoice False
\end{pickone}

\end{problem}

\end{parts}
\end{cluster}

\begin{cluster}{(Cost Analysis) Recurrences}

\begin{gram}
Recall that $f(n)$ is $\Theta(g(n))$
if $f(n) \in O(g(n))$ and $g(n) \in O(f(n))$.
%Find tight bounds in
%terms of $\Theta$ for the following recurrences.
Give a closed-form
solution in terms of $\Theta$ for the following recurrences.
Also, state whether the recurrence is dominated at the root, the
leaves, or equally at all levels of the recurrence tree.
You do not have to show your work, but it might help you get partial
credit.
\end{gram}

\begin{parts}

\begin{problem}[2]
\[
f(n) = 3 f(n/2) + n^2
\]

\begin{pickone}
\correctchoice Root dominated
\choice Leaves Dominated
\choice Approximately Balanced
\end{pickone}
\solution
$n^2$
\end{problem}

\begin{solution}
$n^2$
\end{solution}

\end{parts}
\end{cluster}
