%% This is teaching plan.

\begin{notesonly}

\paragraph{Lecture 1: connectivity, graph theory, edge contraction.} 
\begin{enumerate}

\item
Draw graph that has edges between two
students if they agree (A lovers have an edge and B lovers have on
edge).  The graph will have two components, each of which is a
complete.  The complete graph will be too large for all students so
pick some vertices, you can use for example 3 vertices. You can pick
something that looks like the example graphs used for graphs in the
lecture notes.

\item Another graph to use is a person with their dog.  The vertices
  can be labeled ``human'' and ``dogpq'' where p and q are the paws.
  You can draw them as two  components and then draw a leash to
  connect them.  I did this and it was nice and funny.

\item Now, go over some definitions.
\begin{enumerate}

\item Subgraphs, vertex and edge induced.

\item Connectivity. 

\item Connected component.

\item Graph partitioning. 

\item Naming and partition map.
\end{enumerate}

\item Go back to the graphs for connected components.

\item Ask them for an algorithm to find the A and B lovers given
  this graph.  Discuss BFS and DFS based approaches.  Discuss the case
  where instead of just A and B, you have many components, where BFS
  will perform poorly.



\item 
Start the lecture with a game. Select a group of students that prefer
A over B and another group that prefers B over A.
%
A and B could be tea and coffee, dark or milk chocolate, creamy of
chunky peanut butter. 
%
Call about 20 students to the front of the lecture and ask them mix up
randomly as in a cocktail party.

%
Now ask each student to partner up with a student that has the same
preference.  Students can talk to anyone.
%
In graph theory, this is called a vertex matching.
%
Now, tell them to come together into two groups where one group
consists of students that prefer A over B and the other group consists
of students that prefer B over A. 
%
In graph theory this is called a connected component.  They have found
connected components in the graph.


\item Draw another graph where again you have edges as above but not
  all the edges, you can omit some edges.  For example, you can have
  each component such as a chain. Discuss BFS approach and talk about
  the problem.

\item Summary: you have introduced connectivity, discussed
  graph-search based solutions and why they don't work.  Now,
  presumably the students also did edge contraction and solved the
  connectivity problem.  So this is what we are going to build on.


\item Talk about techniques to solve the connectivity problem.  Divide
  and conquer. Why does it not work. Graph partitioning NP hard.

\item Contraction can work.  Allude to the fact that this is how the
  students solved the problem.  

\item Graph contraction design technique.

\item How to contract a graph? Compute vertex partitioning.  Pick a
  vertex to represent each partition.  Map to super-vertices.  Drop
  internal edges. Reroute cut edges.


\item We repeat, each graph contraction is called a round.

\item Based on how we select each partition, we have different
  algorithms. We will talk about edge partitioning now.


\item Edge partitioning.
\begin{enumerate}
\item Edge partitioning: each part consists of a single vertex or
  two vertices connected by an edge.

\item Edge contraction: graph contraction with edge partitioning. Use
  the circle example~\exref{gc::ep::circle}

\item How to select an edge partitioning?  \exref{gc::ec::matching}
  and show that this is not always easy.

  We can select a set of edges that do not share a vertex and then let
  the vertices not included be singletons.  The problem of finding
  such a set of edges is called the vertex matching problem: select a
  subset of edges such that no two edges.

  This can be turned into a game.  Perhaps introduce in the beginning.
  Ask the volunteer to partner someone (one and only one) in the group
  that matches their interest.


\item Maximal vertex matching. Can be solved requires too much
  work. We don't need a maximal one.

\item Puzzle: how to use a vertex matching this?

\item Greedy algorithm.  The problem with it.

\item Puzzle: how to find one in parallel?  We need to break the
  symmetry.  Can we use randomization?

\item There will likely be many suggestion.  We will analyze the
  algorithm where we flip a coin for each edge and select the edge if
  the edges incident on its end points all flip tails and the edge
  itself flips heads.

\item Analyze the algorithm on a circle

\item Can we improve the bounds?  Each vertex can select a neighbor,
  improves to n/4?
  Or each edge can flip a number between 1 and n and selects if it is
  the maximal. this would improve to n/3.

\item Does not work for many graph, talk about a star graph.


\end{enumerate}

\end{enumerate}


\paragraph{Lecture 2: star contraction.} 

\begin{enumerate}
\item Review:

\begin{enumerate}
\item Subgraphs, vertex induced, partitions, graph partitioning.
\item Connected components.
\item Graph contraction.  Idea: if small solve, otherwise partition
  the graph, and replace each part with a super-vertex, deleting
  internal but keeping cut edges.  Recur.

\item Graph contraction example: Draw example 16.4
\item Big question: how to partitong the graph
\item Edge partitioning and edge contraction. Idea: select a disjoins
  set of edges and contract them. Show
\item Draw example \exref{gc::ec::edge-contraction-example}
\item Describe the analysis on circles.  Describe why it fails on
  general graphs.
\end{enumerate}

\item Star graph.
\item Star partition.
\item Connectivity.
\item Analysis.

\end{enumerate}


\end{notesonly}

