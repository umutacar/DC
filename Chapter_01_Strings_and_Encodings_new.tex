%./texmlt -meta ./meta -preamble latex_preamble/preamble.tex ./01_\iref{definition:String-word-empty-string}{string}s_and_Encodings/Chapter_01_\iref{definition:String-word-empty-string}{string}s_and_Encodings.tex

\chapter{Languages, Encodings and Problems}
\label{chapter:Languages-Encodings-and-Problems}

\begin{preamble}
In the beginning, our goal is to build up, completely formally/mathematically, the important notions related to computation and algorithms. Our starting point is this chapter, which deals with how to formally represent data and how to formally define the concept of a \iref{definition:Computational-problem}{computational problem}.

In theoretical computer science, every kind of data is represented/encoded using finite-length \iref{definition:String-word-empty-string}{string}s. In this chapter, we introduce you to the formal definitions related to \iref{definition:String-word-empty-string}{string}s and \iref{definition:Encoding-of-a-set}{encoding}s of objects with \iref{definition:String-word-empty-string}{string}s. We also present the definitions of ``\iref{definition:Computational-problem}{computational problem}'' and ``\iref{definition:Decision-problem}{decision problem}''. 

All the definitions in this chapter are at the foundation of the formal study of computation.
\end{preamble}

%%%%%%%%%%%%%%%%%%%%%%%%%%%%%%%%%%%
%%%%%%%%%%%%%%%%%%%%%%%%%%%%%%%%%%%
%%%%%%%%%%%%%%%%%%%%%%%%%%%%%%%%%%%


\section{Alphabets and Strings}
\label{section:Alphabets-and-Strings}


\begin{flex}
\begin{definition}[Alphabet, symbol/character] \label{definition:Alphabet-symbol-character}
An \defn{alphabet} is a non-empty, finite set, and is usually denoted by $\Sigma$. 
The elements of $\Sigma$ are called \defn{symbols} or \defn{characters}.
\end{definition}

\begin{example}[Unary alphabet] \label{example:Unary-alphabet}
A unary \iref{definition:Alphabet-symbol-character}{alphabet} consists of one symbol. A common choice for that symbol is $\s{1}$. 
So an example of a unary \iref{definition:Alphabet-symbol-character}{alphabet} is $\Sigma = \{\s{1}\}$.
\end{example}

\begin{example}[Binary alphabet] \label{example:Binary-alphabet}
A binary \iref{definition:Alphabet-symbol-character}{alphabet} consists of two \iref{definition:Alphabet-symbol-character}{symbols}. 
Often we represent those \iref{definition:Alphabet-symbol-character}{symbols} using $\s{0}$ and $\s{1}$. 
So an example of a binary \iref{definition:Alphabet-symbol-character}{alphabet} is $\Sigma = \{\s{0},\s{1}\}$.
Another example of a binary \iref{definition:Alphabet-symbol-character}{alphabet} is $\Sigma=\{\s{a},\s{b}\}$ where $\s{a}$ and $\s{b}$ are the \iref{definition:Alphabet-symbol-character}{symbols}.
\end{example}

\begin{example}[Ternary alphabet] \label{example:Ternary-alphabet}
A ternary \iref{definition:Alphabet-symbol-character}{alphabet} consists of three \iref{definition:Alphabet-symbol-character}{symbols}. 
So $\Sigma=\{\s{0},\s{1},\s{2}\}$ and $\Sigma=\{\s{a},\s{b},\s{c}\}$ are examples of ternary \iref{definition:Alphabet-symbol-character}{alphabet}s. 
\end{example}
\end{flex}


\begin{flex}
\begin{definition}[String/word, empty string] \label{definition:String-word-empty-string}
Given an \iref{definition:Alphabet-symbol-character}{alphabet} $\Sigma$, a \defn{string} (or \defn{word}) over $\Sigma$ is a (possibly infinite) sequence of \iref{definition:Alphabet-symbol-character}{symbols}, written as $a_1a_2a_3\ldots$, where each $a_i \in \Sigma$. 
The \iref{definition:String-word-empty-string}{string} with no \iref{definition:Alphabet-symbol-character}{symbols} is called the \defn{empty string} and is denoted by $\epsilon$.
\end{definition}

\begin{example}[Strings over the unary alphabet] \label{example:Strings-over-the-unary-alphabet}
For $\Sigma = \{\s{1}\}$, the following is a list of 6 \iref{definition:String-word-empty-string}{string}s over $\Sigma$: 
\[
    \epsilon,\; \s{1},\; \s{11},\; \s{111},\; \s{1111},\; \s{11111}.
\]
Furthermore, the infinite sequence $\s{111111}\ldots$ is also a \iref{definition:String-word-empty-string}{string} over $\Sigma$.
\end{example}

\begin{example}[Strings over the binary alphabet] \label{example:Strings-over-the-binary-alphabet}
For $\Sigma = \{\s{0},\s{1}\}$, the following is a list of 8 \iref{definition:String-word-empty-string}{string}s over $\Sigma$: 
\[
\epsilon,\; \s{0},\; \s{1},\; \s{00},\; \s{01},\; \s{10},\; \s{11},\; \s{000}.
\] 
The infinite \iref{definition:String-word-empty-string}{string}s $\s{000000}\ldots$, $\s{111111}\ldots$ and $\s{010101}\ldots$ are also examples of \iref{definition:String-word-empty-string}{string}s over $\Sigma$.
\end{example}
\end{flex}


\begin{note}[Strings and quotation marks] \label{note:Strings-and-quotation-marks}
In our notation of a \iref{definition:String-word-empty-string}{string}, we do not use quotation marks. For instance, we use the notation $\s{1010}$ rather than ``$\s{1010}$'', even though the latter notation using the quotation marks is the standard one in many programming \iref{definition:Language}{language}s. Occasionally, however, we may use quotation marks to distinguish a \iref{definition:String-word-empty-string}{string} like ``$\s{1010}$'' from another type of object with the representation $1010$ (e.g. the binary \emph{number} $1010$).
\end{note}


\begin{flex}
\begin{definition}[Length of a string] \label{definition:Length-of-a-string}
The \defn{length of a string} $w$, denoted $|w|$, is the the number of \iref{definition:Alphabet-symbol-character}{symbols} in $w$. 
If $w$ has an infinite number of \iref{definition:Alphabet-symbol-character}{symbols}, then the length is undefined.
\end{definition}

\begin{example}[Lengths of $01001$ and $\epsilon$] \label{example:Lengths-of-01001-and-epsilon}
Let $\Sigma=\{\s{0},\s{1}\}$. 
The length of the \iref{definition:String-word-empty-string}{word} $\s{01001}$, denoted by $|\s{01001}|$, is equal to $5$. 
The length of $\epsilon$ is 0.
\end{example}
\end{flex}


\begin{flex}
\begin{definition}[Star operation on alphabets] \label{definition:Star-operation-on-alphabets}
Let $\Sigma$ be an \iref{definition:Alphabet-symbol-character}{alphabet}. 
We denote by $\Sigma^*$ the set of \emph{all} strings over $\Sigma$ consisting of finitely many \iref{definition:Alphabet-symbol-character}{symbols}. 
Equivalently, using set notation,
\[
    \Sigma^* = \{a_1a_2\ldots a_n : \text{ $n \in \mathbb{N}$, and $a_i \in \Sigma$ for all $i$}\}.
\]
\end{definition}

\begin{example}[$\{a\}^*$] \label{example:a}
For $\Sigma = \{\s{a}\}$, $\Sigma^*$ denotes the set of all finite-length \iref{definition:String-word-empty-string}{word}s consisting of $\s{a}$'s. 
So
\[
    \{\s{a}\}^* = \{\epsilon, \s{a}, \s{aa}, \s{aaa}, \s{aaaa}, \s{aaaaa}, \ldots \}.
\]
\end{example}

\begin{example}[$\{0,1\}^*$] \label{example:01}
For $\Sigma = \{\s{0},\s{1}\}$, $\Sigma^*$ denotes the set of all finite-length \iref{definition:String-word-empty-string}{word}s consisting of $\s{0}$'s and $\s{1}$'s. 
So
\[
    \{\s{0},\s{1}\}^* = \{\epsilon, \s{0}, \s{1}, \s{00}, \s{01}, \s{10}, \s{11}, \s{000}, \s{001}, \s{010}, \s{011}, \s{100}, \s{101}, \s{110}, \s{111}, \ldots\}.
\]
\end{example}
\end{flex}


\begin{note}[Finite vs infinite strings] \label{note:Finite-vs-infinite-strings}
We often use the \iref{definition:String-word-empty-string}{word}s ``\iref{definition:String-word-empty-string}{string}'' and ``\iref{definition:String-word-empty-string}{word}'' to refer to a finite-length \iref{definition:String-word-empty-string}{string}/\iref{definition:String-word-empty-string}{word}. 
When we want to talk about infinite-length \iref{definition:String-word-empty-string}{string}s, we explicitly use the \iref{definition:String-word-empty-string}{word} ``infinite''.
\end{note}


\begin{note}[Size of $\Sigma^*$] \label{note:Size-of-Sigma}
By Definition~\ref{definition:Alphabet-symbol-character}, an \iref{definition:Alphabet-symbol-character}{alphabet} $\Sigma$ cannot be the empty set. 
This implies that $\Sigma^*$ is an infinite set since there are infinitely many \iref{definition:String-word-empty-string}{string}s of finite length over a non-empty $\Sigma$. We will later see that $\Sigma^*$ is always \emph{countably} infinite.
\end{note}


\begin{flex}
\begin{definition}[Reversal of a string] \label{definition:Reversal-of-a-string}
The \defn{reversal of a string} $w = a_1a_2\ldots a_n$, denoted $w^R$, is the \iref{definition:String-word-empty-string}{string} $w^R = a_na_{n-1}\ldots a_1$.
\end{definition}

\begin{example}[Reversal of $01001$] \label{example:Reversal-of-01001}
The reversal of $\s{01001}$ is $\s{10010}$.
\end{example}

\begin{example}[Reversal of $1$] \label{example:Reversal-of-1}
The reversal of $\s{1}$ is $\s{1}$.
\end{example}

\begin{example}[Reversal of $\epsilon$] \label{example:Reversal-of-epsilon}
The reversal of $\epsilon$ is $\epsilon$.
\end{example}
\end{flex}


\begin{flex}
\begin{definition}[Concatenation of strings] \label{definition:Concatenation-of-strings}
The \defn{concatenation of strings} $u$ and $v$ in $\Sigma^*$, denoted by $uv$ or $u \cdot v$, is the \iref{definition:String-word-empty-string}{string} obtained by joining together $u$ and $v$. 
\end{definition}

\begin{example}[Concatenation of $101$ and $001$] \label{example:Concatenation-of-101-and-001}
If $u = \s{101}$ and $v = \s{001}$, then $uv = \s{101001}$.
\end{example}

\begin{example}[Concatenation of $101$ and $\epsilon$] \label{example:Concatenation-of-101-and-epsilon}
If $u = \s{101}$ and $v = \epsilon$, then $uv = \s{101}$.
\end{example}

\begin{example}[Concatenation of $\epsilon$ and $\epsilon$] \label{example:Concatenation-of-epsilon-and-epsilon}
If $u = \epsilon$ and $v = \epsilon$, then $uv = \epsilon$.
\end{example}
\end{flex}


\begin{flex}
\begin{definition}[Powers of a string] \label{definition:Powers-of-a-string}
For $n \in \mathbb{N}$, the $n$'th \defn{power of a string} $u$, denoted by $u^n$, is the \iref{definition:String-word-empty-string}{word} obtained by concatenating $u$ with itself $n$ times.
\end{definition}

\begin{example}[Third power of $101$] \label{example:Third-power-of-101}
If $u = \s{101}$ then $u^3 = \s{101101101}$.
\end{example}

\begin{example}[Zeroth power of a string] \label{example:Zeroth-power-of-a-string}
For any \iref{definition:String-word-empty-string}{string} $u$, $u^0 = \epsilon$.
\end{example}
\end{flex}


\begin{flex}
\begin{definition}[Substring] \label{definition:Substring}
We say that a \iref{definition:String-word-empty-string}{string} $u$ is a \defn{substring} of \iref{definition:String-word-empty-string}{string} $w$ if $w = xuy$ for some \iref{definition:String-word-empty-string}{string}s $x$ and $y$.
\end{definition}

\begin{example}[$101$ as a substring] \label{example:101-as-a-substring}
The \iref{definition:String-word-empty-string}{string} $\s{101}$ is a sub\iref{definition:String-word-empty-string}{string} of $\s{11011}$ and also a sub\iref{definition:String-word-empty-string}{string} of $\s{0101}$. 
On the other hand, it is not a sub\iref{definition:String-word-empty-string}{string} of $\s{1001}$.
\end{example}
\end{flex}


%%%%%%%%%%%%%%%%%%%%%%%%%%%%%%%%%%%
%%%%%%%%%%%%%%%%%%%%%%%%%%%%%%%%%%%
%%%%%%%%%%%%%%%%%%%%%%%%%%%%%%%%%%%


\section{Languages}
\label{section:Languages}


\begin{flex}
\begin{definition}[Language] \label{definition:Language}
Any (possibly infinite) subset $L \subseteq \Sigma^*$ is called a \defn{language} over the \iref{definition:Alphabet-symbol-character}{alphabet} $\Sigma$.
\end{definition}

\begin{example}[Language of even length strings] \label{example:Language-of-even-length-strings}
Let $\Sigma$ be an \iref{definition:Alphabet-symbol-character}{alphabet}.
Then $L = \{w \in \Sigma^* : \text{ $|w|$ is even}\}$ is a \iref{definition:Language}{language}.
\end{example}

\begin{example}[A language with one word] \label{example:A-language-with-one-word}
Let $\Sigma = \{\s{0},\s{1}\}$.
Then $L = \{\s{101}\}$ is a \iref{definition:Language}{language}.
\end{example}

\begin{example}[$\Sigma^*$ as a language] \label{example:Sigma-as-a-language}
Let $\Sigma$ be an \iref{definition:Alphabet-symbol-character}{alphabet}.
Then $L = \Sigma^*$ is a \iref{definition:Language}{language}.
\end{example}

\begin{example}[Empty set as a language] \label{example:Empty-set-as-a-language}
Let $\Sigma$ be an \iref{definition:Alphabet-symbol-character}{alphabet}.
Then $L = \varnothing$ is a \iref{definition:Language}{language}.
\end{example}
\end{flex}


\begin{note}[Size of a language] \label{note:Size-of-a-language}
Since a \iref{definition:Language}{language} is a set, the \emph{size of a language} refers to the size of that set. 
A \iref{definition:Language}{language} can have finite or infinite size. 
This is not in conflict with the fact that every \iref{definition:Language}{language} consists of finite-length \iref{definition:String-word-empty-string}{string}s. 
\end{note}


\begin{note}[$\{\epsilon\}$ vs $\varnothing$] \label{note:varnothing-vs-epsilon}
The \iref{definition:Language}{language} $\{\epsilon\}$ is not the same \iref{definition:Language}{language} as $\varnothing$. 
The former has size $1$ whereas the latter has size $0$. 
\end{note}


\begin{flex}
\begin{exercise}[Structural induction on words] \label{exercise:Structural-induction-on-words}
Let \iref{definition:Language}{language} $L \subseteq \{\s{0},\s{1}\}^*$ be recursively defined as follows:
\begin{itemize}
    \item (base case) $\epsilon \in L$;
    \item (recursive rule) if $x, y \in L$, then $\s{0}x\s{1}y\s{0} \in L$.
\end{itemize}
This means that every \iref{definition:String-word-empty-string}{word} in the \iref{definition:Language}{language} is derived starting from the base case, and applying the recursive rule a finite number of times. 
Show, using (structural) induction, that for any \iref{definition:String-word-empty-string}{word} $w \in L$, the number of $\s{0}$'s in $w$ is exactly twice the number of $\s{1}$'s in $w$.
\end{exercise}

\begin{solution}
Let $\mathbf{0}(w)$ denote the number of $\s{0}$'s in $w$ and let $\mathbf{1}(w)$ denote the number of $\s{1}$'s in $w$. Given $L$ as defined above, the question asks us to show that for any $w \in L$, $\mathbf{0}(w) = 2 \cdot \mathbf{1}(w)$. We will do so by structural induction.\footn{This means that implicitly, the parameter being inducted on is the minimum number of applications of the recursive rule needed to create an object. And in this case, explicitly stating the parameter being inducted on or the induction hypothesis is not needed.}

The base case corresponds to $w = \epsilon$, and in this case, $\mathbf{0}(w) = \mathbf{1}(w) = 0$, and therefore $\mathbf{0}(w) = 2 \cdot \mathbf{1}(w)$ holds.

To carry out the induction step, consider an arbitrary \iref{definition:String-word-empty-string}{word} $w \neq \epsilon$ in $L$. Then by the definition of $L$, we know that there exists $x$ and $y$ in $L$ such that $w = \s{0}x\s{1}y\s{0}$. Furthermore, by induction hypothesis, 
\begin{equation*} %\label{eq:structural-induction-1}
    \mathbf{0}(x) = 2 \cdot \mathbf{1}(x) \quad \quad (*)
\end{equation*}
and 
\begin{equation*} %\label{eq:structural-induction-2}
    \mathbf{0}(y) = 2 \cdot \mathbf{1}(y). \quad \quad (**)
\end{equation*}
We are done once we show $\mathbf{0}(w) = 2 \cdot \mathbf{1}(w)$. We establish this via the following chain of equalities: 
\begin{align*}
    \mathbf{0}(w) & = 2 + \mathbf{0}(x) + \mathbf{0}(y) & \text{since $w = \s{0}x\s{1}y\s{0}$} \\
    & = 2 + 2 \cdot \mathbf{1}(x) + 2 \cdot \mathbf{1}(y) & \text{by $(*)$ and $(**)$}\\
    & = 2 \cdot (1 + \mathbf{1}(x) + \mathbf{1}(y)) \\
    & = 2 \cdot \mathbf{1}(w).
\end{align*}

\end{solution}
\end{flex}


\begin{flex}
\begin{definition}[Reversal of a language] \label{definition:Reversal-of-a-language}
The \defn{reversal of a language} $L \subseteq \Sigma^*$, denoted $L^R$, is the \iref{definition:Language}{language}
\[
    L^R = \{w^R \in \Sigma^* : w \in L\}. 
\]
\end{definition}

\begin{example}[Reversal of $\{\epsilon, 1, 1010\}$] \label{example:Reversal-of-epsilon-1-1010}
The reversal of the \iref{definition:Language}{language} $\{\epsilon, \s{1}, \s{1010}\}$ is $\{\epsilon, \s{1}, \s{0101}\}$.
\end{example}
\end{flex}


\begin{flex}
\begin{definition}[Concatenation of languages] \label{definition:Concatenation-of-languages}
The \defn{concatenation of languages} $L_1, L_2 \subseteq \Sigma^*$, denoted $L_1L_2$ or $L_1 \cdot L_2$, is the \iref{definition:Language}{language}
\[
    L_1L_2 = \{uv \in \Sigma^* : u \in L_1, v \in L_2\}.
\]
\end{definition}

\begin{example}[Concatenation of $\{\epsilon, 1\}$ and $\{0, 01\}$] \label{example:Concatenation-of-epsilon-1-and-0-01}
The concatenation of \iref{definition:Language}{language}s $\{\epsilon, \s{1}\}$ and $\{\s{0}, \s{01}\}$ is the \iref{definition:Language}{language}
\[
    \{\s{0}, \s{01}, \s{10}, \s{101}\}.
\] 
\end{example}
\end{flex}


\begin{flex}
\begin{definition}[Powers of a language] \label{definition:Powers-of-a-language}
For $n \in \N$, the $n$'th \defn{power of a language} $L \subseteq \Sigma^*$, denoted $L^n$, is the \iref{definition:Language}{language} obtained by concatenating $L$ with itself $n$ times, that is,\footn{We can omit parentheses as the order in which the concatenation $\cdot$ is applied does not matter.}
\[
    L^n = \underbrace{L \cdot L \cdot L \cdots L}_{n \text{ times}}.
\]
Equivalently, 
\[
    L^n = \{u_1u_2\cdots u_n \in \Sigma^* : u_i \in L \text{ for all } i \in \{1,2,\ldots,n\}\}.
\] 
\end{definition}

\begin{example}[$\{1\}^3$] \label{example:13}
The 3rd power of $\{\s{1}\}$ is the \iref{definition:Language}{language} $\{\s{111}\}$.
\end{example}

\begin{example}[$\{\epsilon, 1\}^3$] \label{example:epsilon-13}
The 3rd power of $\{\epsilon, \s{1}\}$ is the \iref{definition:Language}{language} $\{\epsilon, \s{1}, \s{11}, \s{111}\}$.
\end{example}

\begin{example}[$L^0$] \label{example:L0}
The 0th power of any \iref{definition:Language}{language} $L$ is the \iref{definition:Language}{language} $\{\epsilon\}$.
\end{example}
\end{flex}


\begin{flex}
\begin{definition}[Star operation on a language] \label{definition:Star-operation-on-a-language}
The \defn{star of a language} $L \subseteq \Sigma^*$, denoted $L^*$, is the \iref{definition:Language}{language} 
\[
L^* = \bigcup_{n \in \N} L^n.
\]
Equivalently, 
\[
L^* = \{u_1u_2\cdots u_n \in \Sigma^* : n \in \N, u_i \in L \text{ for all } i \in \{1,2,\ldots,n\}\}.
\]
\end{definition}

\begin{example}[$\Sigma^*$] \label{example:Sigma-star}
Given an \iref{definition:Alphabet-symbol-character}{alphabet} $\Sigma$, consider the \iref{definition:Language}{language} $L = \Sigma \subseteq \Sigma^*$\footn{Technically $L$ is a set of strings and $\Sigma$ is a set of symbols, so the equality notation is not entirely accurate. Hopefully the intention is clear however: symbols can be viewed as length-1 strings.}. Then $L^*$ is equal to $\Sigma^*$.
\end{example}

\begin{example}[$\{00\}^*$] \label{example:00-star}
If $L = \{\s{00}\}$, then $L^*$ is the \iref{definition:Language}{language} consisting of all \iref{definition:String-word-empty-string}{word}s containing an even number of $\s{0}$'s and no other symbol. 
\end{example}

\begin{example}[$(\{00\}^*)^*$] \label{example:00-starstar}
Let $L$ be the \iref{definition:Language}{language} consisting of all \iref{definition:String-word-empty-string}{word}s containing an even number of $\s{0}$'s and no other symbol. Then $L^* = L$. 
\end{example}
\end{flex}


\begin{flex}
\begin{exercise}[Can you distribute star over intersection?] \label{exercise:Can-you-distribute-star-over-intersection}
Prove or disprove: If $L_1, L_2 \subseteq \{\s{a},\s{b}\}^*$ are \iref{definition:Language}{language}s, then $(L_1 \cap L_2)^* = L_1^* \cap L_2^*$.
\end{exercise}

\begin{solution}
We disprove the statement by providing a counterexample. Let $L_1 = \{\s{a}\}$ and $L_2 = \{\s{aa}\}$. Then $L_1 \cap L_2 = \emptyset$, and so $(L_1 \cap L_2)^* = \{\epsilon\}$. On the other hand, $L_1^* \cap L_2^* = L_2^* = \{\s{aa}\}^*$.
\end{solution}
\end{flex}


\begin{flex}
\begin{exercise}[Can you interchange star and reversal?] \label{exercise:Can-you-interchange-star-and-reversal}
Is it true that for any \iref{definition:Language}{language} $L$, $(L^*)^R = (L^R)^*$? Prove your answer.
\end{exercise}


\begin{solution}
We will prove that for any \iref{definition:Language}{language} $L$, $(L^*)^R = (L^R)^*$. To do this, we will first argue $(L^*)^R \subseteq (L^R)^*$ and then argue $(L^R)^* \subseteq (L^*)^R$.

To show the first inclusion, it suffices to show that any $w \in (L^*)^R$ is also contained in $(L^R)^*$. We do so now. Take an arbitrary $w \in (L^*)^R$. Then for some $n \in \N$, $w = (u_1u_2\ldots u_n)^R$, where $u_i \in L$ for each $i$. Note that $w = (u_1u_2\ldots u_n)^R = u_n^R u_{n-1}^R \ldots u_1^R$, and $u_i^R \in L^R$ for each $i$. Therefore $w \in (L^R)^*$.

To show the second inclusion, it suffices to show that any $w \in (L^R)^*$ is also contained in $(L^*)^R$. We do so now. Take an arbitrary $w \in (L^R)^*$. This means that for some $n \in \N$, $w = v_1 v_2 \ldots v_n$, where $v_i \in L^R$ for each $i$. For each $i$, define $u_i = v_i^R$ (and so $u_i^R = v_i$). Note that each $u_i \in L$ because $v_i \in L^R$. We can now rewrite $w$ as $w = u_1^R u_2^R \ldots u_n^R$, which is equal to $(u_n u_{n-1} \ldots u_1)^R$. Since each $u_i \in L$, this shows that $w \in (L^*)^R$.

Since we have shown both $(L^*)^R \subseteq (L^R)^*$ and $(L^R)^* \subseteq (L^*)^R$, we conclude that $(L^*)^R = (L^R)^*$.
\end{solution}
\end{flex}


% %%%%%%%%%%%%%%%%%%%%%%%%%%%%%%%%%%%
% %%%%%%%%%%%%%%%%%%%%%%%%%%%%%%%%%%%
% %%%%%%%%%%%%%%%%%%%%%%%%%%%%%%%%%%%


\section{Encodings}
\label{section:Encodings}


\begin{flex}
\begin{definition}[Encoding of a set] \label{definition:Encoding-of-a-set}
Let $A$ be a set (which is possibly countably infinite\footn{We assume you know what a countable set is, however, this concept is reviewed in the next chapter.}), and let $\Sigma$ be an \iref{definition:Alphabet-symbol-character}{alphabet}. 
An \defn{encoding} of the elements of $A$, using $\Sigma$, is an injective function $\text{Enc}: A \to \Sigma^*$. 
We denote the \iref{definition:Encoding-of-a-set}{encoding} of $a \in A$ by $\langle a \rangle$.\footn{Note that this angle-bracket notation does not specify the underlying encoding function as the particular choice of encoding function is often unimportant.} 

If $w \in \Sigma^*$ is such that there is some $a \in A$ with $w = \langle a \rangle$, then we say $w$ is a \defn{valid encoding} of an element in $A$. 

A set that can be encoded is called \defn{encodable}.\footn{Not every set is encodable. Can you figure out exactly which sets are encodable?}
\end{definition}

\begin{example}[Decimal encoding of naturals] \label{example:Decimal-encoding-of-naturals}
When we (humans) communicate numbers among ourselves, we usually use the base-10 representation, which corresponds to an \iref{definition:Encoding-of-a-set}{encoding} of $\N$ using the \iref{definition:Alphabet-symbol-character}{alphabet} $\Sigma = \{\s{0},\s{1},\s{2},\ldots, \s{9}\}$. For example, we encode the number four as $\s{4}$ and the number twelve as $\s{12}$.
\end{example}

\begin{example}[Binary encoding of naturals] \label{example:Binary-encoding-of-naturals}
As you know, every number has a base-$2$ representation (which is also known as the binary representation). This representation corresponds to an \iref{definition:Encoding-of-a-set}{encoding} of $\N$ using the \iref{definition:Alphabet-symbol-character}{alphabet} $\Sigma = \{\s{0},\s{1}\}$. For example, four is encoded as $\s{100}$ and twelve is encoded as $\s{1100}$.
\end{example}

\begin{example}[Binary encoding of integers] \label{example:Binary-encoding-of-integers}
An integer is a natural number together with a sign, which is either negative or positive. Let $\text{Enc} : \N \to \{\s{0},\s{1}\}^*$ be any binary \iref{definition:Encoding-of-a-set}{encoding} of $\N$. Then we can extend this \iref{definition:Encoding-of-a-set}{encoding} to an \iref{definition:Encoding-of-a-set}{encoding} of $\Z$, by defining $\text{Enc}':\Z \to \{\s{0},\s{1}\}^*$ as follows:
\[
\text{Enc}'(x) = 
\begin{cases}
\s{0}\text{Enc}(x) & \text{if $x \geq 0$}, \\
\s{1}\text{Enc}(|x|) & \text{if $x < 0$}.
\end{cases}
\]
Effectively, this \iref{definition:Encoding-of-a-set}{encoding} of integers takes the \iref{definition:Encoding-of-a-set}{encoding} of natural numbers and precedes it with a bit indicating the integer's sign.
\end{example}

\begin{example}[Unary encoding of naturals] \label{example:Unary-encoding-of-naturals}
It is possible (and straightforward) to encode the natural numbers using the \iref{definition:Alphabet-symbol-character}{alphabet} $\Sigma = \{\s{1}\}$ as follows. Let $\text{Enc}(n) = \s{1}^n$ for all $n \in \N$.
\end{example}

\begin{example}[Ternary encoding of pairs of naturals] \label{example:Ternary-encoding-of-pairs-of-naturals}
Suppose we want to encode the set $A = \N \times \N$ using the \iref{definition:Alphabet-symbol-character}{alphabet} $\Sigma = \{\s{0},\s{1},\s{2}\}$. One way to accomplish this is to make use of a binary \iref{definition:Encoding-of-a-set}{encoding} $\text{Enc}': \N \to \{\s{0},\s{1}\}^*$ of the natural numbers. With $\text{Enc}'$ in hand, we can define $\text{Enc} : \N \times \N \to \{\s{0},\s{1},\s{2}\}^*$ as follows. For $(x,y) \in \N \times \N$, $\text{Enc}(x,y) = \text{Enc}'(x)\s{2}\text{Enc}'(y)$. Here the symbol $\s{2}$ acts as a separator between the two numbers. To make the separator symbol advertise itself as such, we usually pick a symbol like $\s{\$}$ rather than $\s{2}$. So the ternary \iref{definition:Alphabet-symbol-character}{alphabet} is often chosen to be $\Sigma = \{\s{0},\s{1},\s{\$}\}$.
\end{example}

\begin{example}[Binary encoding of pairs of naturals] \label{example:Binary-encoding-of-pairs-of-naturals}
Having a ternary \iref{definition:Alphabet-symbol-character}{alphabet} to encode pairs of naturals was convenient since we could use the third symbol as a separator. It is also relatively straightforward to take that ternary \iref{definition:Encoding-of-a-set}{encoding} and turn it into a binary \iref{definition:Encoding-of-a-set}{encoding}, as follows. Encode every element of the ternary \iref{definition:Alphabet-symbol-character}{alphabet} in binary using two bits. For instance, if the ternary \iref{definition:Alphabet-symbol-character}{alphabet} is $\Sigma = \{\s{0},\s{1},\s{\$}\}$, then we could encode $\s{0}$ as $\s{00}$, $\s{1}$ as $\s{01}$ and $\s{\$}$ as $\s{11}$. This mapping allows us to convert any encoded \iref{definition:String-word-empty-string}{string} over the ternary \iref{definition:Alphabet-symbol-character}{alphabet} into a binary \iref{definition:Encoding-of-a-set}{encoding}. For example, a \iref{definition:String-word-empty-string}{string} like $\s{\$0\$1}$ would have the binary representation $\s{11001101}$.
\end{example}

\begin{example}[Ternary encoding of graphs] \label{example:Ternary-encoding-of-graphs}
Let $A$ be the set of all undirected graphs.\footn{We will define graphs formally in a future chapter, however, we assume you are already familiar with the concept.} Every graph $G=(V,E)$ can be represented by its $|V|$ by $|V|$ adjacency matrix. In this matrix, every row corresponds to a vertex of the graph, and similarly, every column corresponds to a vertex of the graph. The $(i,j)$'th entry contains a $1$ if $\{i, j\}$ is an edge, and contains a $0$ otherwise. Below is an example.
\begin{center}
    \includegraphics[width=10cm]{01_Strings_and_Encodings/media/graph-adj-matrix.png}
\end{center}
Such a graph can be encoded using a ternary \iref{definition:Alphabet-symbol-character}{alphabet} as follows. Take the adjacency matrix of the graph, view each row as a binary \iref{definition:String-word-empty-string}{string}, and concatenate all the rows by putting a separator symbol between them. The \iref{definition:Encoding-of-a-set}{encoding} of the above example would be
\[
\langle G \rangle = \s{0101\$1010\$0101\$1010}.
\]
\end{example}

\begin{example}[Encoding of Python functions] \label{example:Encoding-of-Python-functions}
Let $A$ be the set of all functions in the programming \iref{definition:Language}{language} Python. Whenever we type up a Python function in a code editor, we are creating a \iref{definition:String-word-empty-string}{string} representation/\iref{definition:Encoding-of-a-set}{encoding} of the function, where the \iref{definition:Alphabet-symbol-character}{alphabet} is all the Unicode \iref{definition:Alphabet-symbol-character}{symbols}.\footn{\url{https://en.wikipedia.org/wiki/Unicode}} For example, consider a Python function named absValue, which we can write as
\begin{verbatim}
def absValue(N):
    if (N < 0): return -N
    else: return N
\end{verbatim}
By writing out the function, we have already encoded it. More specifically, $\langle \text{absValue} \rangle$ is the \iref{definition:String-word-empty-string}{string}
% \[
% \s{def absValue(N):\n    if (N < 0): return -N\n    else: return N}
% \]
\begin{verbatim}
def absValue(N):\n    if (N < 0): return -N\n    else: return N
\end{verbatim}
%(Here we are using quotation marks to denote the \iref{definition:String-word-empty-string}{string} so it is clear that the period at the end does not belong to the \iref{definition:String-word-empty-string}{string}.)
\end{example}
\end{flex}


\begin{flex}
\begin{exercise}[Unary encoding of integers] \label{exercise:Unary-encoding-of-integers}
Describe an \iref{definition:Encoding-of-a-set}{encoding} of $\Z$ using the \iref{definition:Alphabet-symbol-character}{alphabet} $\Sigma = \{\s{1}\}$.
\end{exercise}

\begin{solution}
Let $\text{Enc}:\Z \to \{\s{1}\}^*$ be defined as follows:
\[
\text{Enc}(x) = 
\begin{cases}
\s{1}^{2x-1} & \text{if $x > 0$}, \\
\s{1}^{-2x}  & \text{if $x \leq 0$.}
\end{cases}
\]
This solution is inspired by thinking of a bijection between integers and naturals. Indeed, the function $f: \Z \to \N$ defined by  
\[
f(x) = 
\begin{cases}
2x-1 & \text{if $x > 0$}, \\
-2x  & \text{if $x \leq 0$,}
\end{cases}
\]
is such a bijection.
\end{solution}
\end{flex}


%%%%%%%%%%%%%%%%%%%%%%%%%%%%%%%%%%%
%%%%%%%%%%%%%%%%%%%%%%%%%%%%%%%%%%%
%%%%%%%%%%%%%%%%%%%%%%%%%%%%%%%%%%%


\section{Computational Problems and Decision Problems}
\label{section:Computational-Problems-and-Decision-Problems}


\begin{flex}
\begin{definition}[Computational problem] \label{definition:Computational-problem}
Let $\Sigma$ be an \iref{definition:Alphabet-symbol-character}{alphabet}. Any function $f: \Sigma^* \to \Sigma^*$ is called a \defn{computational problem} over the \iref{definition:Alphabet-symbol-character}{alphabet} $\Sigma$. 
\end{definition}

\begin{example}[Addition as a computational problem] \label{example:Addition-as-a-computational-problem}
Consider the function $g:\N \times \N \to \N$ defined as $g(x, y) = x + y$. 
This is a function that expresses the addition problem in naturals. 
We can view $g$ as a \iref{definition:Computational-problem}{computational problem} over an \iref{definition:Alphabet-symbol-character}{alphabet} $\Sigma$ once we fix an \iref{definition:Encoding-of-a-set}{encoding} of the domain $\N \times \N$ using $\Sigma$ and an \iref{definition:Encoding-of-a-set}{encoding} of the codomain $\N$ using $\Sigma$. 
For convenience, we take $\Sigma = \{\s{0},\s{1},\s{\$}\}$. Let $\text{Enc}$ be the \iref{definition:Encoding-of-a-set}{encoding} of $\N \times \N$ as described in Example~\ref{example:Ternary-encoding-of-pairs-of-naturals}. 
Let $\text{Enc}'$ be the \iref{definition:Encoding-of-a-set}{encoding} of $\N$ as described in Example~\ref{example:Binary-encoding-of-naturals}. 
Note that $\text{Enc}'$ leaves the symbol $\s{\$}$ unused in the \iref{definition:Encoding-of-a-set}{encoding}. 
We now define the \iref{definition:Computational-problem}{computational problem} $f$ corresponding to $g$. 
If $w \in \Sigma^*$ is a \iref{definition:String-word-empty-string}{word} that corresponds to a valid \iref{definition:Encoding-of-a-set}{encoding} of a pair of numbers $(x, y)$ (i.e., $\text{Enc}(x,y) = w$), then define $f(w)$ to be $\text{Enc}'(x+y)$. 
If $w \in \Sigma^*$ is \emph{not} a \iref{definition:String-word-empty-string}{word} that corresponds to a valid \iref{definition:Encoding-of-a-set}{encoding} of a pair of numbers (i.e., $w$ is not in the image of $\text{Enc}$), then define $f(w)$ to be $\s{\$}$. 
In the codomain, the $\s{\$}$ symbol serves as an ``error'' indicator.
\end{example}
\end{flex}


\begin{important}[Computational problem as mapping instances to solutions] \label{important:Computational-problem-as-mapping-instances-to-solutions}
A \iref{definition:Computational-problem}{computational problem} is often derived from a function $g: I \to S$, where $I$ is a set of objects called \emph{instances} and $S$ is a set of objects called \emph{solutions}. 
The derivation is done through \iref{definition:Encoding-of-a-set}{encoding}s $\text{Enc}: I \to \Sigma^*$ and $\text{Enc}': S \to \Sigma^*$. 
With these \iref{definition:Encoding-of-a-set}{encoding}s, we can create the \iref{definition:Computational-problem}{computational problem} $f : \Sigma^* \to \Sigma^*$. 
In particular, if $w = \langle x \rangle$ for some $x \in I$, then we define $f(w)$ to be $\text{Enc}'(g(x))$.

\begin{center}
\includegraphics[width=6cm]{01_Strings_and_Encodings/media/encoding-comm-diagram.png}
%\begin{tikzcd}
%  I \arrow[r, "g"] \arrow[d, swap, "\text{Enc}"] & S \arrow[d, "\text{Enc}'"] \\
%  \Sigma^* \arrow[r, "f"] & \Sigma^* 
%\end{tikzcd}
\end{center}

One thing we have to be careful about is defining $f(w)$ for a \iref{definition:String-word-empty-string}{word} $w \in \Sigma^*$ that does not correspond to an \iref{definition:Encoding-of-a-set}{encoding} of an object in $I$ (such a \iref{definition:String-word-empty-string}{word} does not correspond to an instance of the \iref{definition:Computational-problem}{computational problem}). 
To handle this, we can identify one of the \iref{definition:String-word-empty-string}{string}s in $\Sigma^*$ as an \emph{error} string and define $f(w)$ to be that \iref{definition:String-word-empty-string}{string}.
\end{important}


\begin{flex}
\begin{definition}[Decision problem] \label{definition:Decision-problem}
Let $\Sigma$ be an \iref{definition:Alphabet-symbol-character}{alphabet}. Any function $f: \Sigma^* \to \{0,1\}$ is called a \defn{decision problem} over the \iref{definition:Alphabet-symbol-character}{alphabet} $\Sigma$. 
The codomain of the function is not important as long as it has two elements. 
Other common choices for the codomain are $\{\text{No}, \text{Yes}\}$, $\{\text{False}, \text{True}\}$ and $\{ \text{Reject}, \text{Accept}\}$.
\end{definition}

\begin{example}[Primality testing as a decision problem] \label{example:Primality-testing-as-a-decision-problem}
Consider the function $g:\N \to \{\text{False}, \text{True}\}$ such that $g(x) = \text{True}$ if and only if $x$ is a prime number.
We can view $g$ as a \iref{definition:Decision-problem}{decision problem} over an \iref{definition:Alphabet-symbol-character}{alphabet} $\Sigma$ once we fix an \iref{definition:Encoding-of-a-set}{encoding} of the domain $\N$ using $\Sigma$. 
Take $\Sigma = \{\s{0},\s{1}\}$. 
Let $\text{Enc}$ be the \iref{definition:Encoding-of-a-set}{encoding} of $\N$ as described in Example~\ref{example:Binary-encoding-of-naturals}. 
We now define the \iref{definition:Decision-problem}{decision problem} $f$ corresponding to $g$. 
If $w \in \Sigma^*$ is a \iref{definition:String-word-empty-string}{word} that corresponds to an \iref{definition:Encoding-of-a-set}{encoding} of a prime number, then define $f(w)$ to be $\text{True}$. Otherwise, define $f(w)$ to be $\text{False}$. (Note that in the case of $f(w) = \text{False}$, either $w$ is the \iref{definition:Encoding-of-a-set}{encoding} of a composite number, or $w$ is not a valid \iref{definition:Encoding-of-a-set}{encoding} of a natural number.
\end{example}
\end{flex}


\begin{note}[Decision problem as mapping instances to $0$ or $1$s] \label{note:Decision-problem-as-mapping-instances-to-0-or-1s}
As with a \iref{definition:Computational-problem}{computational problem}, a \iref{definition:Decision-problem}{decision problem} is often derived from a function $g: I \to \{0,1\}$, where $I$ is a set of instances. 
The derivation is done through an \iref{definition:Encoding-of-a-set}{encoding} $\text{Enc}: I \to \Sigma^*$, which allows us to define the \iref{definition:Decision-problem}{decision problem} $f: \Sigma^* \to \{0,1\}$. Any \iref{definition:String-word-empty-string}{word} $w \in \Sigma^*$ that does not correspond to an \iref{definition:Encoding-of-a-set}{encoding} of an instance is mapped to $0$ by $f$.
\end{note}

\begin{important}[Correspondence between decision problems and languages] \label{important:Correspondence-between-decision-problems-and-languages}
There is a one-to-one correspondence between \iref{definition:Decision-problem}{decision problem}s and \iref{definition:Language}{language}s. Let $f:\Sigma^* \to \{0,1\}$ be some \iref{definition:Decision-problem}{decision problem}. Now define $L \subseteq \Sigma^*$ to be the set of all \iref{definition:String-word-empty-string}{word}s in $\Sigma^*$ that $f$ maps to 1. This $L$ is the \iref{definition:Language}{language} corresponding to the \iref{definition:Decision-problem}{decision problem} $f$. Similarly, if you take any \iref{definition:Language}{language} $L \subseteq \Sigma^*$, we can define the corresponding \iref{definition:Decision-problem}{decision problem} $f:\Sigma^* \to \{0,1\}$ as $f(w) = 1$ if and only if $w \in L$. We consider the set of \iref{definition:Language}{language}s and the set of \iref{definition:Decision-problem}{decision problem}s to be the same set of objects.
\begin{center}
    \includegraphics[width=12cm]{01_Strings_and_Encodings/media/decision-problem-language.png}
\end{center}
\end{important}



%%%%%%%%%%%%%%%%%%%%%%%%%%%%%%%%%%%%%%%%%%%%%
%%%%%%%%%%%%%%%%%%%%%%%%%%%%%%%%%%%%%%%%%%%%%
%%%%%%%%%%%%%%%%%%%%%%%%%%%%%%%%%%%%%%%%%%%%%



