\chapter[100]{Example Quiz I}
\label{ch:quiz-ii}

Some preamble \somecommand

\begin{preamble}
This is an example quiz.
\end{preamble}
Some tailtex with some \command

\paragraph{First paragraph of chapter.}

Atom preamble
\begin{gram}
First paragraph of chapter.
\end{gram}
Some tailtex with some \command
\newpage

\section{Material}

\paragraph{First Paragraph of Material}

Some preamble \somecommand
\begin{cluster}
\begin{datatype}[Sets]
\label{XXadt:sets} 
For a universe of elements $\uuu$ that support equality (e.g. the integers or strings), the 
\adt{SET} abstract data type is a type $\sss$ representing the power 
set of $\uuu$ (i.e., all subsets of $\uuu$) limited to sets of finite 
size, along with the functions below. 
{\normalsize
input{./sets-and-tables/sig-sets}
}
Where $\tynat$ is 
the natural numbers (non-negative integers) and $\bbb = \{\texttt{true},
\texttt{false}\}$.
%; for a set $A$ of type $\sss$.
%, $\means{A}$ denotes the  (mathematical) set of keys in the set.
\end{datatype}
Some tailtex with some \command
\end{cluster}

Some tailtex with some \command

\section{Sets ADT}

Some preamble \somecommand
\begin{cluster}
\begin{datatype}[Sets]
\label{XXadt:sets} 
For a universe of elements $\uuu$ that support equality (e.g. the integers or strings), the 
\adt{SET} abstract data type is a type $\sss$ representing the power 
set of $\uuu$ (i.e., all subsets of $\uuu$) limited to sets of finite 
size, along with the functions below. 
{\normalsize
input{./sets-and-tables/sig-sets}
}
Where $\tynat$ is 
the natural numbers (non-negative integers) and $\bbb = \{\texttt{true},
\texttt{false}\}$.
%; for a set $A$ of type $\sss$.
%, $\means{A}$ denotes the  (mathematical) set of keys in the set.
\end{datatype}

Some preamble \somecommand
\begin{syntax}[Sets] 
\label{XXsyn:sets}
In \pml{}  we use the standard set notation $\cset{e_o,e_1,\cdots,e_n}$ to
  indicate a set.  The notation $\emptyset$ or $\cset{}$ refers to an
  empty set. We also use the conventional mathematical syntax for set
  functions such as $|S|$ (size), $\cup$ (union), $\cap$
  (intersection), and $\setminus$ (difference).  In addition, we use
  set comprehensions for $\cdvar{filter}$ and for constructing sets from
  other sets.
\end{syntax}

Some preamble \somecommand
\begin{gram}
  The objects that are contained in a set are called~\defn{members}~or the~\defn{elements}~of the set.  Recall that a~\defn{set}~is a collection of distinct objects.  This requires that
  the universe $\uuu$ they come from support equality.  It might seem that
  all universes support equality, but consider functions.  When are
  two functions equal?   It is not even decidable whether two
  functions are equal.   From a practical matter, there is no way to
  implement sets without an equality function over potential
  elements.   In fact efficient implementations additionally require
  either a hash function over the elements of $\uuu$ and/or a total ordering.
\end{gram}

Some preamble \somecommand
\begin{problem}[10][Problem within Cluster]
Body of Problem within Cluster
\end{problem}

Some tailtex with some \command

\end{cluster}

Some preamble \somecommand
\begin{gram}
The Set ADT consists of basic functions on sets.  
%
The function $\cdvar{size}$ takes a set and returns the number of elements
in the set.
% 
The function $\cdvar{toSeq}$ converts a set to a sequence by ordering the
elements of the set in an unspecified way. 
%
Since elements of a set do not necessarily have a total ordering, the
resulting order is arbitrary.
%
This means that $\cdvar{toSeq}$ is possibly non-deterministic---it could
return different orderings in different implementations.
%, or even on different runs of the same implementation.
%
We specify $\cdvar{toSeq}$ as follows
\[
\cdvar{toSeq}~(\{x_0,x_1,\ldots,x_n\} : \sss): seq = \cseq{x_0,x_1,\ldots,x_n}
\]
where the $x_i$ are an arbitrary ordering. 
\end{gram}


Some preamble \somecommand
\begin{gram}
Several functions enable constructing sets.
%
The function $\cdvar{empty}$ returns an empty set:
%
\[
\cdvar{empty} : \sss = \emptyset
\]
%
The function $\cdvar{singleton}$ constructs a singleton set from a given
element.
%
\[
\cdvar{singleton} (x : \uuu) : \sss = \{x \}
\]
%
The function $\cdvar{fromSeq}$ takes a sequence and returns a set consisting of the
distinct elements of the sequence, eliminating duplicate elements.
%
We can specify $\cdvar{fromSeq}$ as returning the range of the sequence
$A$ (recall that a sequence is a partial function mapping from natural numbers
to elements of the sequence).
%
\[
\cdvar{fromSeq}~(a : seq) : \sss = \cdvar{range}~a
\]
%
\end{gram}

Some preamble \somecommand
\begin{gram}
%
Several functions operate on sets to produce new sets.
%
The function $\cdvar{filter}$ selects the elements of a sequence that
satisfy a given Boolean function, i.e., 
%
\[
\cdvar{filter}~(f : \uuu \ra \tybool)~(a : \sss) : \sss = \{ x \in a \sucht f(x)\} .
\]
%
The functions $\cdvar{intersection}$, $\cdvar{difference}$, and $\cdvar{union}$
perform the corresponding set operation on their arguments:
%
\[
\begin{array}{l}
\cdvar{intersection}~(a : \sss)~(b : \sss) : \sss = a \cap b\\
\cdvar{difference}~(a  : \sss)~(b : \sss) : \sss = a \setminus b\\
\cdvar{union}~(a : \sss)~(b : \sss) : \sss = a \cup b
\end{array}
\]
%
We refer to the functions  $\cdvar{intersection}$, $\cdvar{difference}$, and $\cdvar{union}$
as \defn{bulk updates}, because they allow updating with a large set
of elements ``in bulk.''
\end{gram}

\begin{gram}
The functions $\cdvar{find}$, $\cdvar{insert}$, and $\cdvar{delete}$ are singular
versions of the bulk functions $\cdvar{intersection}$, $\cdvar{union}$, and
$\cdvar{difference}$ respectively.
%
The $\cdvar{find}$ function checks whether an element is in a set---it is
the basic membership test for sets.
%
\[
find~(a  : \sss)~(x : \uuu) : \tybool = \left\{
                \begin{array}{ll}
                \cd{true} & \cd{if}~x \in A \\
                \cd{false} & \cd{otherwise}
                \end{array} \right.
\]
%
We can also specify the $\cdvar{find}$ function is in terms of set
intersection:
\[
find~(a : \sss)~(x : \uuu) : \tybool = \csetsize{a \cap \cset{x}} = 1.
\]
%
The functions $\cdvar{delete}$ and $\cdvar{insert}$ 
%
delete an existing element from a set, and
%
insert a new element into a set,
%
respectively:
%
\[
\begin{array}{l}
\cdvar{delete}~(a  : \sss)~(x  : \uuu) : \sss = a \setminus \cset{x}.\\
\cdvar{insert}~(a : \sss)~(x : \uuu) : \sss = a \cup \cset{x}
\end{array}
\]
%
\end{gram}

\begin{gram}
Iteration and reduction over sets can be easily defined by converting
them to sequences, as in

\[
\begin{array}{l}
\cdvar{iterate}~f~x~a = \cdvar{Sequence.iterate}~f~(\cdvar{toSeq}~a)\\
\cdvar{reduce}~f~x~a = \cdvar{Sequence.reduce}~f~(\cdvar{toSeq}~a)
\end{array}
\]
\end{gram}

Some tailtex with some \command


\section[50]{Basics}

\begin{problem}[10][First Problem]
Standalone Problem
\end{problem}

\begin{mproblem}[Multipart Problem I]
\label{cl:quiz-ii::1}

\begin{gram}
Some flavor text for problem 1.
\end{gram}

\begin{gram}
\label{quiz-ii::1::1}
Para I of Multipart problem I.
\end{gram}

\begin{problem}[20][Problem I.1]
\depend{quiz::ii::1::1}
Select one of the following
\begin{xchoice}
\choice This is the first choice $a = z^2$.
\choice This is the second choice $b = y^2$.
\choice This is the third choice $c$.
\choice* This is the fourth choice $d$.
\choice This is the fifth choice $d$.
\end{xchoice}

\solution
This is a solution 1 .

\explain
This is an explanation 1.

\rubric
Rubric for problem I.1
\end{problem}


\begin{problem}[20][Problem I.2]
%\depend{quiz-ii::1::1}
Select one of the following
\begin{xchoice}
\choice This is the first choice $a = z^2$.
\choice This is the second choice $b = y^2$.
\choice* This is the third choice $c$.
\choice* This is the fourth choice $d$.
\choice This is the fifth choice $d$.
\end{xchoice}


\help 
Hint  for Problem I.2.

\solution
This is a solution I.2.

\explain
This is an explanation I.

\end{problem}







\begin{problem}[40][Problem I.3]
\label{quiz-ii:2}
\depend { quiz-ii::1::1,  quiz-ii::1::2}
Select one of the following
\begin{xchoice}
\choice This is the first choice $a = z^2$.
\choice This is the second choice $b = y^2$.
\choice* This is the third choice $c$.
\choice* This is the fourth choice $d$.
\choice This is the fifth choice $d$.
\end{xchoice}

\help  for Problem I.3

\rubric
Rubric for problem I.3
\end{problem}

\end{mproblem}
Some tailtex with some \command


\section[50]{Sequences}
\label{sec:quiz-ii::middle}

\begin{gram}
Here are some assumptions
\begin{description}
\item[A] Assumption A
\begin{enumerate}
\item Assumption A.1
\item Assumption A.2
\end{enumerate}

\item[B] Assumption B
\end{description}
\end{gram}


\begin{problem}[70][Xchoice I]
Select one of the following
\begin{xchoice}
\choice This is the first choice $a = z^2$.
\choice This is the second choice $b = y^2$.
\choice* This is the third choice $c$.
\choice* This is the fourth choice $d$.
\choice This is the fifth choice $d$.
\end{xchoice}
\end{problem}

\begin{problem}[80][Xchoice I.2]
Select one of the following
\begin{xchoice}
\choice This is the first choice $a = z^2$.
\choice This is the second choice $b = y^2$.
\choice* This is the third choice $c$.
\choice* This is the fourth choice $d$.
\choice This is the fifth choice $d$.
\end{xchoice}
\end{problem}


\begin{problem}[90][Xchoice II]
Select one of the following
\begin{xchoice}
\choice This is the first choice $a = z^2$.
\choice This is the second choice $b = y^2$.
\choice* This is the third choice $c$.
\choice* This is the fourth choice $d$.
\choice This is the fifth choice $d$.
\end{xchoice}
\end{problem}

\begin{problem}[100][Xchoice III]
Select one of the following
\begin{xchoice}
\choice This is the first choice $a = z^2$.
\choice This is the second choice $b = y^2$.
\choice* This is the third choice $c$.
\choice* This is the fourth choice $d$.
\choice This is the fifth choice $d$.
\end{xchoice}
\end{problem}
Some tailtex with some \command


\section[100]{Graphs}

\begin{problem}[110][Harder Xchoice I]
Select one of the following
\begin{xchoice}
\choice This is the first harder choice $a = z^2$.
\choice This is the second harder choice $b = y^2$.
\choice* This is the third harder choice $c$.
\choice* This is the fourth harder choice $d$.
\choice This is the fifth harder choice $d$.
\end{xchoice}
\end{problem}


\begin{problem}[120][Harder Xchoice II]
Select one of the following
\begin{xchoice}
\choice This is the first harder choice $a = z^2$.
\choice This is the second harder choice $b = y^2$.
\choice* This is the third harder choice $c$.
\choice* This is the fourth harder choice $d$.
\choice This is the fifth harder choice $d$.
\end{xchoice}
\end{problem}

\begin{problem}[130][Harder Xchoice III]
Select one of the following
\begin{xchoice}
\choice This is the first harder choice $a = z^2$.
\choice This is the second harder choice $b = y^2$.
\choice* This is the third harder choice $c$.
\choice* This is the fourth harder choice $d$.
\choice This is the fifth harder choice $d$.
\end{xchoice}
\end{problem}



\begin{mproblem}[2000][Multipart Problem II]
\begin{gram}[200]
Para I of Problem II
\end{gram}


\begin{problem}[210][Problem II.1]
Select one of the following
\begin{xchoice}
\choice This is the first choice $a = z^2$.
\choice This is the second choice $b = y^2$.
\choice* This is the third choice $c$.
\choice* This is the fourth choice $d$.
\choice This is the fifth choice $d$.
\end{xchoice}
\help Hint  for Problem II.1
\end{problem}


\begin{problem}[220][Problem II.2]
Select one of the following
\begin{xchoice}
\choice This is the first choice $a = z^2$.
\choice This is the second choice $b = y^2$.
\choice* This is the third choice $c$.
\choice* This is the fourth choice $d$.
\choice This is the fifth choice $d$.
\end{xchoice}
\help 
Hint  for Problem II.2
\end{problem}

\begin{problem}[230][Problem II.3]
Select one of the following
\begin{xchoice}
\choice This is the first choice $a = z^2$.
\choice This is the second choice $b = y^2$.
\choice* This is the third choice $c$.
\choice* This is the fourth choice $d$.
\choice This is the fifth choice $d$.
\end{xchoice}

\help
Hint  for Problem II.3
\end{problem}


\begin{problem}[240][Problem II.4]
This is a free response question.

\help
(Hint Field:) Here is a hint II.

\solution
Here is the solution II.

\explain
This is an explanation II.

\help
Hint  for Problem II.4
\end{problem}


\end{mproblem}
