\newcommand{\ouralgorithm}{the source-paths algorithm}
\newcommand{\Ouralgorithm}{The source-paths algorithm}

\newcommand{\readarrow}{\ensuremath{\Longrightarrow}}
\newcommand{\currenttime}{\ttt{currentTime}\xspace}
\newcommand{\tags}[1]{\ensuremath{\mathsf{tags}(#1)}}
\newcommand{\weight}[1]{\ensuremath{\mathsf{weight}(#1)}}
\newcommand{\dist}[2]{\ensuremath{\delta(#1,#2)}}


\newcommand{\qqquad}{\quad\quad\quad}

\newcommand{\cutspace}{\vspace{-4mm}}

\newcommand{\bomb}[1]{\fbox{\mbox{\emph{\bf {#1}}}}}

%\newcommand{\myparagraph}[1]{{\bf {#1}}}
\newcommand{\vparagraph}[1]{\vspace{-.1in}\paragraph{#1}}

% formatting stuff
\newcommand{\codecolsep}{1ex}



\newcommand{\rlabel}[1]{\hspace*{-1mm}\mbox{\small{\bf ({#1})}}}

\newcommand{\tablerow}{\\[5ex]}
\newcommand{\tableroww}{\\[7ex]}
\newcommand{\tableline}{
\vspace*{2ex}\\
\hline\\ 
\vspace*{2ex}}

% Don't care
\newcommand{\dontcare}{\_
}

%% filter and quicksort stuff
\newcommand{\ncf}[2]{C^{fil}_{\ensuremath{{#1},{#2}}}}
\newcommand{\ncq}[1]{C^{qsort}_{\ensuremath{#1}}}
\newcommand{\nuf}[3]{P^{fil}_{#1,(\ensuremath{{#2},{#3}})}}
\newcommand{\nuq}[2]{P^{qsort}_{(\ensuremath{{#1},{#2}})}}

%% shorthands
\newcommand{\ddg}{{\sc ddg}}
\newcommand{\ncpa}{change-propagation algorithm}
\newcommand{\adg}{{\sc adg}}
\newcommand{\nwrite}{\texttt{write}}
\newcommand{\nread}{\texttt{read}}
\newcommand{\nmodr}{\texttt{mod}}
\newcommand{\ttt}[1]{\texttt{#1}}
\newcommand{\nmodl}{\texttt{modl}}
\newcommand{\nnil}{\ttt{NIL}}
\newcommand{\ncons}[2]{\ttt{CONS({\ensuremath{#1},\ensuremath{#2}})}}
\newcommand{\nfilter}{\texttt{filter}}
\newcommand{\nfilterp}{\texttt{filter'}}
\newcommand{\naqsort}{\texttt{qsort'}}
\newcommand{\nqsort}{\texttt{qsort}}
\newcommand{\nqsortp}{\texttt{qsort'}}
\newcommand{\nnewMod}{\texttt{newMod}}
\newcommand{\nchange}{\texttt{change}}
\newcommand{\npropagate}{\texttt{propagate}}
\newcommand{\ndest}{\texttt{d}}
\newcommand{\ninit}{\texttt{init}}



%% Comment sth out. 
\newcommand{\out}[1] {}
\newcommand{\sthat}{\ensuremath{~|~}}

%% definitions
\newcommand{\defi}[1]{{\bfseries\itshape #1}}


% Code listings.
\newcounter{codeLineCntr}
\newcommand{\codeLine}
 {\refstepcounter{codeLineCntr}{\thecodeLineCntr}}
\newcommand{\codeLineL}[1]
 {\refstepcounter{codeLineCntr}\label{#1}{\thecodeLineCntr}}
\newcommand{\codeLineNN}{} %% NN = No-Number (and no change to counter)
\newcommand{\codeNoLine}{}

\newenvironment{codeListing}
 {\setcounter{codeLineCntr}{0}
%  \fontsize{10}{12}
 % the first one is width the second is height
 \fontsize{9}{11}
  \fontsize{8}{8}
  \vspace{-.1in}
  \ttfamily\begin{tabbing}}
  {\end{tabbing}
   \vspace{-.1in}}

\newenvironment{codeListing8}
 {\setcounter{codeLineCntr}{0}
%  \fontsize{8}{10}
  \fontsize{8}{8}
  \vspace{-.1in}
  \ttfamily
  \begin{tabbing}}
 {\end{tabbing}
 \vspace{-.1in}
}

\newenvironment{codeListing8h}
 {\setcounter{codeLineCntr}{0}
  \fontsize{8.5}{10.5}
  \vspace{-.1in}
  \ttfamily
  \begin{tabbing}}
 {\end{tabbing}
 \vspace{-.1in}
}


\newenvironment{codeListing9}
 {\setcounter{codeLineCntr}{0}
  \fontsize{9}{11}
  \vspace{-.1in}
  \ttfamily
  \begin{tabbing}}
 {\end{tabbing}
 \vspace{-.1in}
}

\newenvironment{codeListing10}
 {\setcounter{codeLineCntr}{0}
  \fontsize{10}{12}
  \vspace{-.1in}
  \ttfamily
  \begin{tabbing}}
 {\end{tabbing}
 \vspace{-.1in}
}


\newenvironment{codeListingNormal}
 {\setcounter{codeLineCntr}{0}
  \vspace{-.1in}
  \ttfamily
  \begin{tabbing}}
 {\end{tabbing}
 \vspace{-.1in}
}

\newcommand{\codeFrame}[1]
{\begin{center}\fbox{\parbox[t]{\columnwidth}{#1}}\end{center}
% \vspace*{-.15in}
}

\newcommand{\halfBox}[1]
{\begin{center}\fbox{\parbox[t]{\columnwidth}{#1}}\end{center}
% \vspace*{-.15in}
}

\newcommand{\fullBox}[1]
{\begin{center}\fbox{\parbox[t]{\textwidth}{#1}}\end{center}
% \vspace*{-.15in}
}

%%Note this is redefined in local-mac.tex for each paper.
\newcommand{\fixedCodeFrame}[1]
{
\begin{center}
\fbox{
\parbox[t]{0.9\columnwidth}{
#1
}
}\end{center}
\vspace{-.2in}
}

% Footnote commands.
\newcommand{\footnotenonumber}[1]{{\def\thempfn{}\footnotetext{#1}}}

% Margin notes - use \notesfalse to turn off notes.
\setlength{\marginparwidth}{0.6in}
\reversemarginpar
\newif\ifnotes
\notestrue
\newcommand{\longnote}[1]{
  \ifnotes
    {\medskip\noindent Note:\marginpar[\hfill$\Longrightarrow$]
      {$\Longleftarrow$}{#1}\medskip}
  \fi}
\newcommand{\note}[1]{
  \ifnotes
    {\marginpar{\raggedright{\tiny #1}}}
  \fi}
\newcommand{\notered}[1]{\ifnotes
    {\marginpar{\raggedright{\tiny
          {\sf\color{red} #1}}}}
    \fi}

\newcommand{\mathcolor}[2]{\text{\textcolor{#1}{\ensuremath{#2}}}}

% Stuff not wanted.
\newcommand{\punt}[1]{}

% Sectioning commands.
\newcommand{\subsec}[1]{\subsection{\boldmath #1 \unboldmath}}
\newcommand{\subheading}[1]{\subsubsection*{#1}}
\newcommand{\subsubheading}[1]{\paragraph*{#1}}

% Reference shorthands.
\newcommand{\spref}[1]{Modified-Store Property~\ref{sp:#1}}
\newcommand{\prefs}[2]{Properties~\ref{p:#1} and~\ref{p:#2}}
\newcommand{\pref}[1]{Property~\ref{p:#1}}


\newcommand{\partref}[1]{Part~\ref{part:#1}}
\newcommand{\chref}[1]{Chapter~\ref{ch:#1}}
\newcommand{\chreftwo}[2]{Chapters \ref{ch:#1} and~\ref{ch:#2}}
\newcommand{\chrefthree}[3]{Chapters \ref{ch:#1}, and~\ref{ch:#2}, and~\ref{ch:#3}}
\newcommand{\secref}[1]{Section~\ref{sec:#1}}
\newcommand{\subsecref}[1]{Subsection~\ref{subsec:#1}}
\newcommand{\secreftwo}[2]{Sections \ref{sec:#1} and~\ref{sec:#2}}
\newcommand{\secrefthree}[3]{Sections \ref{sec:#1},~\ref{sec:#2},~and~\ref{sec:#3}}
\newcommand{\appref}[1]{Appendix~\ref{app:#1}}
\newcommand{\figref}[1]{Figure~\ref{fig:#1}}
\newcommand{\figreftwo}[2]{Figures \ref{fig:#1} and~\ref{fig:#2}}
\newcommand{\figrefthree}[3]{Figures \ref{fig:#1}, \ref{fig:#2} and~\ref{fig:#3}}
\newcommand{\figreffour}[4]{Figures \ref{fig:#1},~\ref{fig:#2},~\ref{fig:#3}~and~\ref{fig:#4}}
\newcommand{\figpageref}[1]{page~\pageref{fig:#1}}
\newcommand{\tabref}[1]{Table~\ref{tab:#1}}
\newcommand{\tabreftwo}[2]{Tables~\ref{tab:#1} and~\ref{tab:#1}}

\newcommand{\stref}[1]{step~\ref{step:#1}}
\newcommand{\caseref}[1]{case~\ref{case:#1}}
\newcommand{\lineref}[1]{line~\ref{line:#1}}
\newcommand{\linereftwo}[2]{lines \ref{line:#1} and~\ref{line:#2}}
\newcommand{\linerefthree}[3]{lines \ref{line:#1},~\ref{line:#2},~and~\ref{line:#3}}
\newcommand{\linerefrange}[2]{lines \ref{line:#1} through~\ref{line:#2}}
\newcommand{\thmref}[1]{Theorem~\ref{thm:#1}}
\newcommand{\thmreftwo}[2]{Theorems \ref{thm:#1} and~\ref{thm:#2}}
\newcommand{\thmpageref}[1]{page~\pageref{thm:#1}}
\newcommand{\lemref}[1]{Lemma~\ref{lem:#1}}
\newcommand{\lemreftwo}[2]{Lemmas \ref{lem:#1} and~\ref{lem:#2}}
\newcommand{\lemrefthree}[3]{Lemmas \ref{lem:#1},~\ref{lem:#2},~and~\ref{lem:#3}}
\newcommand{\lempageref}[1]{page~\pageref{lem:#1}}
\newcommand{\corref}[1]{Corollary~\ref{cor:#1}}
\newcommand{\defref}[1]{Definition~\ref{def:#1}}
\newcommand{\defreftwo}[2]{Definitions \ref{def:#1} and~\ref{def:#2}}
\newcommand{\defpageref}[1]{page~\pageref{def:#1}}
\renewcommand{\eqref}[1]{Equation~(\ref{eq:#1})}
\newcommand{\eqreftwo}[2]{Equations (\ref{eq:#1}) and~(\ref{eq:#2})}
\newcommand{\eqpageref}[1]{page~\pageref{eq:#1}}
\newcommand{\ineqref}[1]{Inequality~(\ref{ineq:#1})}
\newcommand{\ineqreftwo}[2]{Inequalities (\ref{ineq:#1}) and~(\ref{ineq:#2})}
\newcommand{\ineqpageref}[1]{page~\pageref{ineq:#1}}
\newcommand{\itemref}[1]{Item~\ref{item:#1}}
\newcommand{\itemreftwo}[2]{Item~\ref{item:#1} and~\ref{item:#2}}

% Useful shorthands.
\newcommand{\abs}[1]{\left| #1\right|}
\newcommand{\card}[1]{\left| #1\right|}
\newcommand{\norm}[1]{\left\| #1\right\|}
\newcommand{\floor}[1]{\left\lfloor #1 \right\rfloor}
\newcommand{\ceil}[1]{\left\lceil #1 \right\rceil}
  \renewcommand{\choose}[2]{{{#1}\atopwithdelims(){#2}}}
\newcommand{\ang}[1]{\langle#1\rangle}
\newcommand{\paren}[1]{\left(#1\right)}
\newcommand{\prob}[1]{\Pr\left\{ #1 \right\}}
\newcommand{\expect}[1]{\mathrm{E}\left[ #1 \right]}
\newcommand{\expectsq}[1]{\mathrm{E}^2\left[ #1 \right]}
\newcommand{\variance}[1]{\mathrm{Var}\left[ #1 \right]}
%\newcommand{\twodots}{\mathinner{\ldotp\ldotp}}

% Standard number sets.
\newcommand{\reals}{{\mathrm{I}\!\mathrm{R}}}
\newcommand{\integers}{\mathbf{Z}}
\newcommand{\naturals}{{\mathrm{I}\!\mathrm{N}}}
\newcommand{\rationals}{\mathbf{Q}}
\newcommand{\complex}{\mathbf{C}}

% Set notation
\newcommand{\sect}{\cap}

% Special styles.
%\newcommand{\proc}[1]{\ifmmode\mbox{\textsc{#1}}\else\textsc{#1}\fi}
\newcommand{\procdecl}[1]{
  \proc{#1}\vrule width0pt height0pt depth 7pt \relax}
%  \newcommand{\func}[1]{\ifmmode\mathrm{#1}\else\textrm{#1}fi} %
%  Multiple cases.  
\renewcommand{\cases}[1]{\left\{
  \begin{array}{ll}#1\end{array}\right.}
  \newcommand{\cif}[1]{\mbox{if $#1$}} 

%% spacing hacks
\newcommand{\longpage}{\enlargethispage{\baselineskip}}
\newcommand{\shortpage}{\enlargethispage{-\baselineskip}}



%% Notes, todos, and remarks
\newcounter{remark}[section]

\newcommand{\myremark}[3]{
\refstepcounter{remark}
\[
\left\{
\sf 
\parbox{0.8\columnwidth}
{
{\bf {#1}'s remark~\theremark:} 
{#3}
}
\right.
\]
\marginpar{\bf {#2}.~\theremark}
}



\newcommand{\rremark}[1]{\myremark{Rohan}{R}{#1}}
\newcommand{\uremark}[1]{\myremark{Umut}{U}{#1}}
%\newcommand{\uremark}[1]{}
\newcommand{\gremark}[1]{\myremark{Guy}{G}{#1}}
\newcommand{\mremark}[1]{\myremark{Mike}{M}{#1}}
%\newcommand{\ur}[1]{\uremark{#1}}
\newcommand{\rr}[1]{\rremark{#1}}
\newcommand{\ur}[1]{\uremark{#1}}
%% \newcommand{\rr}[1]{}
%% \newcommand{\ur}[1]{}

\newcommand{\gr}[1]{\gremark{#1}}
\newcommand{\mr}[1]{\mremark{#1}}


\newcommand{\todo}[1]{{\bf{[TODO:{#1}]}}}

\newcommand{\defeq}{\stackrel{\text{def}}{=}}
\newcommand{\setbuild}[2]{\{#1~|~#2\}}

%
\newcommand{\twork}[1]{\mathcal{W}\left({#1}\right)}
\newcommand{\tspan}[1]{\mathcal{S}\left({#1}\right)}
\newcommand{\tmax}[2]{\operatorname{Max}\left({#1},{#2}\right)}

\newcommand{\bigml}{\textsf{BigML}\xspace}

\newcommand{\taskcost}{I}
\newcommand{\taskgc}{III}
\newcommand{\taskhmm}{III}
\newcommand{\subtaskgcs}{II.A}
\newcommand{\subtaskgcp}{II.B}
\newcommand{\tasksch}{II}
\newcommand{\taskac}{IV}
\newcommand{\taskpml}{V}
\newcommand{\taskeval}{VI}

%% language
\newcommand{\lfs}{\ensuremath{\mathcal{L}_{FS}}\xspace}
\newcommand{\lfp}{\ensuremath{\mathcal{L}}\xspace}

\newcommand{\kwparcompose}{~~~~||~~~~}
\newcommand{\kwseqcompose}{~~~~;~~~~}
\newcommand{\cdparens}[1]{\mathcd{(}{#1}\mathcd{)}}
\newcommand{\kw}[1]{\mbox{\ttt{#1}}}
\newcommand{\Int}{\kw{int}}
\newcommand{\Letpar}{\kw{letpar}}
\newcommand{\Slet}{\kw{let}}
\newcommand{\Mod}[1]{{#1}~\kw{mod}}
\newcommand{\kwn}{\kw{n}}
\newcommand{\kwlet}[3]{\kw{let}~{#1}={#2}~\kw{in}~{#3}~\kw{end}}
\newcommand{\kwletpar}[5]{\kw{letpar}~\cdparens{{#1},{#3}}=\cdparens{{#2},{#4}}~\kw{in}~{#5}~\kw{end}}
\newcommand{\kwifthenelse}[3]{\kw{if}\,{#1}\,\kw{then}\,{#2}\,\kw{else}\,{#3}}
\newcommand{\kwfun}[3]{\ensuremath{\kw{fun}~{#1}~{#2}~\kw{is}~{#3}~\kw
{end}}}
\newcommand{\kwsfun}[3]{\ensuremath{\kw{fun}_{\kw{s}}~{#1}~{#2}~\kw{is}~{#3}~\kw{end}}}
\newcommand{\kwcfun}[3]{\ensuremath{\kw{fun}_{\kw{c}}~{#1}~{#2}~\kw{is}~{#3}~\kw{end}}}
\newcommand{\kwpair}[2]{\ensuremath{\lparen{#1},{#2}}\rparen}
 \newcommand{\kwpar}[2]{\ensuremath{\llparenthesis\,{#1},{#2}\,\rrparenthesis}}
\newcommand{\kwapply}[2]{\ensuremath{{#1}~{#2}}}
\newcommand{\kwnadd}[2]{\ensuremath{{#1} \oplus {#2}}}

\newcommand{\kwmod}[1]{\ensuremath{\kw{mod}\cdparens{#1}}}
\newcommand{\kwread}[3]{\ensuremath{\kw{read}~{#1}~\kw{as}~{#2}~\kw{in}~{#3}~\kw{end}}}
\newcommand{\kwfst}[1]{\ensuremath{\kw{fst}\cdparens{#1}}}
\newcommand{\kwsnd}[1]{\ensuremath{\kw{snd}\cdparens{#1}}}

\newcommand{\trread}[3]{\kw{READ}_{#1}{#2}~\kw{IN}~{#3}}
\newcommand{\trletpar}[3]{\kw{INPAR}~{#1}~\kw{AND}~{#2}~\kw{THEN}~{#3}}

%% shorthands
\renewcommand{\a}{\ensuremath{\alpha}}
\renewcommand{\b}{\ensuremath{\beta}}
\newcommand{\h}{\ensuremath{\eta}}
\renewcommand{\r}{\ensuremath{\rho}}
\newcommand{\s}{\ensuremath{\sigma}}



% Relations
\newcommand{\red}{\Downarrow}
\newcommand{\redgc}{\stackrel{gc?}{\Longrightarrow}}
\newcommand{\alloc}{\stackrel{alloc}\Longrightarrow}
\newcommand{\la}{\leftarrow}
\newcommand{\ra}{\rightarrow}

\newcommand{\sunion}[2]{{#1} \stackrel{?}{\bigcup} {#2}}
\newcommand{\spush}[2]{{#1} \stackrel{?}{\downarrow} {#2}}

% store, heap
\newcommand{\mks}[2]{{#1} \stackrel{?}{::} {#2}}
\newcommand{\mkh}[2]{{#1}_{#2}}
\newcommand{\mkhe}{\mkh{\emptyset}{\emptyset}}
\newcommand{\hemp}{\emptyset}

\newcommand{\locs}[1]{\mathcal{L}({#1})}
\newcommand{\htop}[1]{\mathcal{T}({#1})}
\newcommand{\roots}[1]{\mathcal{R}({#1})}


\newcommand{\hmm}{\textsf{HMM}}


\newcommand{\cd}[1]{\lstinline{#1}}

\newcommand{\defn}[1]{\emph{\textbf{#1}}}
\newcommand{\dom}[1]{\textsf{dom}(#1)}
\newcommand{\rng}[1]{\textsf{range}(#1)}
