\chapter{Practice Exam II}

\begin{preamble}
\newcommand{\numquestions}{5}

\begin{itemize}
\item There are \numquestions{} questions worth a total of \numpoints{} points.
  The last few pages are an appendix with costs of sequence,
  set and table operations.
\item You have 80 minutes to complete this examination.
\item Please answer all questions in the space provided with the
  question.  Clearly indicate your answers.
%% \item There are scratch pages at the end of the exam for your use. In no
%%   circumstance will anything you write on these pages count for or against
%%   your score on this exam.
\item You may refer to your one double-sided $8\frac{1}{2} \times 11$in
  sheet of paper with notes, but to no other person or source, during the
  examination.

\item Your answers for this exam must be written in blue or black ink.

\end{itemize}
\end{preamble}

%\input{../../../setstables/classes}


\titledquestion{Short Answers}

\begin{problem}[4p][Classes]

\used{Fall 14, Final}


Lets say you are given a table that maps every student to the set of
classes they take. 

\answer
Fill in the algorithm below that returns all classes,
assuming there is at least one student in each class.  Your algorithm
must run in $O(m \log n)$ work and $O((\log m)(\log n))$ span, where
$n$ is the number of students and $m$ is the sum of the number of
classes taken across all students.    Note, our solution is one line.


\vspace{.3in}
\begin{lstlisting}[numbers=none]
fun allClasses($T$ : classSet studentTable) : classSet = 
@\vspace{.5in}@
\end{lstlisting}

\sol
\begin{lstlisting}[numbers=none]
fun allClasses($T$) = Table.reduce Set.union $\emptyset$ $T$
\end{lstlisting}

\end{problem}


%% \newpage
%% \input{../../../bfs/shortest-weighted}

\begin{problem}[10pts][Shortest Weighted]
\used{Fall 14, Practice Exam II}

\answer

Given a graph with integer edge weights between $1$ and $5$
(inclusive), you want to find the shortest \emph{weighted} path
between a pair of vertices.  How would you reduce this problem to the
shortest \emph{unweighted} path problem, which can be solved using
BFS?

\sol
  Replace each edge with weight $i$ with a simple path of $i$ edges
  each with weight $1$. Then solve with BFS.
\end{problem}



%% \newpage
%% \input{../../../setstables/bingled}
\titledquestion{(Sets and Tables) Bingled}

After forming your company Bingle to index the web allowing word
searches based on logical combination of terms (e.g. ``big'' and
``small''), you discover that there are already a couple companies out
there that do it....and lo-and-behold, they even have similar names.
You therefore decide to extend yours with additional features.  In
particular you want to support phrase queries: e.g. find all
documents where ``fun algorithms'' appears.

You decide the right way to represent the index is as a table of sets
where the keys of the table are strings (i.e. the words) and the
elements of the sets are pairs of values consisting of a document
identifier and an integer location in the document where the string
appears.  So, for example the following collection of three documents
with integer document identifiers:

\begin{quote}
\begin{tabbing}
 $\langle$
  \=(1, \ctext{``the big dog''}), \\
  \>(2, \ctext{``a big dog ate a hat''}),\\
  \>(3, \ctext{``i read a big book''}) $\rangle$
\end{tabbing}
\end{quote}

the document index would be represented as
\begin{lstlisting}[numbers=none]
idx = $\{~\ctext{``a''} \mapsto \cset{(2,0),(2,4),(3,2)},$
        $\ctext{``big''} \mapsto \cset{(1,1),(2,1),(3,3)},$
        $\ctext{``dog''} \mapsto \cset{(1,2),(2,2)},$
        $\ldots~\}$
\end{lstlisting}

In particular you want to support the following interface
\begin{lstlisting} [numbers=none]
signature INDEX = sig
  type word = string
  type docId = int
  type index = docIdIntSet wordTable
  
  (* represents all documents and all locations where a phrase appears *)
  type docList

  val makeIndex : (docId * string) seq -> index    
  val find : index -> word -> docList
  val adj : docList * docList -> docList
  val toSeq : docList -> docId seq 
end
\end{lstlisting}
%
where, given an index \sml{I},
\texttt{toSeq (adj (find I "210", find I "rocks"))} would return a sequence of
identifiers of documents
where ``210'' appears immediately before ``rocks'', and \\
%
\sml{toSeq (adj (find I "Umut", adj (find I "loves", find I "climbing")))}\\
%
would return a sequence of identifiers of documents
where the phrase ``Umut loves climbing'' appears.

 
\begin{mproblem}

\begin{problem}[8pts]
\freeresponse
Show SML code to generate the index from the sequence of documents.
It should not be more than 8 lines of code and assuming all words have
length less than some constant, must run in $O(n \log n)$ work and
$O(\log^2 n)$ span, where $n$ is the total number of words across all
documents.    You can use a function
\begin{quote}
\sml{val toWords~:~string -> string seq}
\end{quote}
that breaks a text string into a sequence of words.

\begin{lstlisting}[numbers=none]
type index = docIdIntSet wordTable

fun makeIndex (docs : (docId  * string) seq) : index =
  let
\end{lstlisting}

\sol
\begin{lstlisting} [numbers=none]
  fun tagWords (id,doc) = 
      let val words = toWords doc
      in  Seq.tabulate (fn i = (nth i words, (id, i)) (length words) 
      end

  val allPairs = Seq.flatten (Seq.map tagWords docs)

  val wordTable = Table.collect allPairs
in
    Table.map Set.fromSeq wordTable
end
\end{lstlisting}

\end{problem}


\begin{problem}
\freeresponse
% Show code for a function \texttt{adj(docList1, docList2)} that given
% two docLists returns a docList in which those two words
% are adjacent.  For example for the index generated from the documents
% above, 
% \begin{quote}
% \texttt{toSeq(adj(find idx "a", find idx "big"))} 
% \end{quote}
% would return
% $\cseq{2, 3}$.  For full credit \texttt{adj} must be an associative operator.
Define the \sml{docList} type and implement the function \sml{adj} as defined above.
You might find the function \sml{setmap} useful. The solution should
only be a few lines of code.
\begin{lstlisting}[numbers=none]
fun setmap f s = Set.fromSeq (Seq.map f (Set.toSeq s)) 
\end{lstlisting}

\begin{lstlisting}[numbers=none]

type docList = 

fun adj (                ,               ) : docList =
\end{lstlisting}

\sol
\begin{lstlisting}[numbers=none]
type docList = (docIdIntSet)*int

fun adj ((d1,l1), (d2,l2)) = 
  let 
      val d2' = setmap (fn (d,i) = (d, i-l1)) d2
  in  
      (Set.intersection (d1,d2'), l1+l2)
  end

(* FYI: not part of exam *)
fun find idx word = 
  case Table.find idx word =>
    NONE => (Set.empty(), 1)
  | SOME d => (d, 1)
\end{lstlisting}
\end{problem}
   
\end{mproblem}

%% \newpage
%% \input{../../../shortest-paths/dijkstra-astar}

%% \newpage
%% \input{../../../shortest-paths/wormholes}

%% \newpage
%% \input{../../../graphs/scc2}

%% \end{questions}

%% \newpage\section*{Appendix: Library Functions}
%% \input{../../../latex/libfuncs}

