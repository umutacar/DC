\chapter{Asymptotics}
\label{ch:analysis::asymptotics}

\begin{cluster}
\label{grp:prmbl:analysis::asymptotics::describes}

\begin{preamble}
\label{prmbl:analysis::asymptotics::describes}
This chapter describes the asymptotic notation that is used nearly
universally in computer science to analyze the resource consumption of
algorithms.

\end{preamble}
\end{cluster}


\section{Basics}
\label{sec:analysis::asymptotics::asymptotics}

\begin{flex}
\label{grp:grm:analysis::asymptotics::analyzing}

\begin{gram}
\label{grm:analysis::asymptotics::analyzing}
When analyzing algorithms, we are usually interested in
costs such as the total work, the running time, or space usage. 
In such analysis, we typically characterize the behavior of an
algorithm with a~\defn{numeric function} from the domain of natural
numbers (typically input sizes) to the codomain of real numbers (cost).

\end{gram}

\begin{example}[Numeric Functions]
\label{xmpl:analysis::asymptotics::numeric-functions}
By analyzing the work of the algorithm~$A$ for problem~$P$ in terms of
its input size $n$, we may obtain the numeric function
$$W_A(n) = 2n\lg{n} + 3n + 4\lg{n} + 5.$$  
By applying the analysis method to another algorithm, algorithm~$B$,
we may derive the numeric function
$$ W_B(n) = 6n + 7\lg^2{n} + 8\lg{n} + 9.$$
Both of these functions are numeric because their domain is the natural
numbers.

\end{example}
\end{flex}

\begin{flex}
\label{grp:grm:analysis::asymptotics::given}

\begin{gram}
\label{grm:analysis::asymptotics::given}
When given numeric functions, how should we interpret them?
Perhaps more importantly given two algorithms and their work cost as
represented by two numeric functions, how should we compare them?
One option would be to calculate the two functions for varying values
of~$n$ and pick the algorithm that does the least amount of work for
the values of $n$ that we are interested in.

In computer science, we typically care about the cost of an algorithm
for large inputs.
We are therefore usually interested in the~\defn{growth} or
the~\defn{growth rate} of the functions.
Asymptotic analysis offers a technique for comparing algorithms by
comparing the growth rate of their cost functions as the sizes get
large (approach infinity).

\end{gram}

\begin{example}[Asymptotics]
\label{xmpl:analysis::asymptotics::asymptotics}
Consider two algorithms~$A$ and $B$ for a problem~$P$ and suppose that
their work costs, in terms of the input size $n$, are 
$$W_A(n) = 2n\lg{n} + 3n + 4\lg{n} + 5, \mbox{and}$$
$$W_B(n) = 6n + 7\lg^2{n} + 8\lg{n}+ 9.$$
Via asymptotic analysis, we derive 
$$W_A(n) \in \Theta(n\lg{n}),~\mbox{and}$$
$$W_B(n) \in \Theta(n).$$
Since ~$n\lg{n}$ grows faster that~$n$, we would usually prefer the
second algorithm, because it performs better for sufficiently large
inputs.

The difference between the exact work expressions and the ``asymptotic
bounds'' written in terms of the ``Theta'' functions is that the
latter ignores so called~\defn{constant factors}, which are the
constants in front of the variables, and~\defn{lower-order terms},
which are the terms such as $3n$ and $4\lg{n}$ that diminish in
growth with respect to $n\lg{n}$ as $n$ increases.

\end{example}
\end{flex}

\begin{cluster}
\label{grp:rmrk:analysis::asymptotics::addition}

\begin{remark}
\label{rmrk:analysis::asymptotics::addition}
In addition to enabling us to compare algorithms, asymptotic analysis
also allows us to ignore certain details such as the exact time an
operation may require to complete on a particular architecture.
This is important because it makes it possible to apply our analysis
to different architectures, where such constant may differ.
Furthermore, it also enables us to create more abstract cost models:
in designing cost models, we assign most operations unit costs
regardless of the exact time they might take on hardware.
This greatly simplifies the definition of the models.

\end{remark}
\end{cluster}

\begin{cluster}
\label{grp:tch:analysis::asymptotics::know}

\begin{teachask}
\label{tch:analysis::asymptotics::know}
Do you know of an algorithm that compared to other algorithms for the
same problem, performs asymptotically better at large inputs but poorly
at smaller inputs.

\end{teachask}
\end{cluster}

\begin{flex}
\label{grp:xrcs:analysis::asymptotics::comparing}

\begin{exercise}
\label{xrcs:analysis::asymptotics::comparing}
    Comparing two algorithms that solve the same problem, one might
    perform better on large inputs and the other on small inputs.  
    Can you give an example? 
  

\end{exercise}

\begin{solution}
\label{sol:analysis::asymptotics::classic}
    There are many such algorithms.  
        A classic example is the merge-sort algorithm that performs
    $\Theta(n\lg{n})$ work, but performs worse on smaller inputs than the
    asymptotically inefficient $\Theta(n^2)$-work insertion-sort
    algorithm.
        Asymptotic notation does not help in comparing the efficiency of
    insertion sort and merge sort at small input sizes.
        For this, we need to compare their actual work functions which include
    the constant factors and lower-order terms that asymptotic notation
    ignores.
  

\end{solution}
\end{flex}


\section{Big-O, big-Omega, and big-Theta}
\label{sec:analysis::asymptotics::big-o-big-omega-and-big-theta}

\begin{cluster}
\label{grp:grm:analysis::asymptotics::idea}

\begin{gram}
\label{grm:analysis::asymptotics::idea}
  The key idea in asymptotic analysis is to understand how the growth
  rate of two functions compare on large input.  In 
  particular as we increase the numeric argument of both functions to
  infinity, does one grow faster, equally fast or slower than the
  other?   In
  answering this question we do not care about small input and 
  do not care about constant factors.  To capture this idea, we use the following
  definition.

\end{gram}
\end{cluster}

\begin{flex}
\label{grp:def:analysis::asymptotics::asymptotic-dominance}

\begin{definition}[Asymptotic dominance]
\label{def:analysis::asymptotics::asymptotic-dominance}
    Let $f(\cdot)$ and $g(\cdot)$ be two numeric functions.  We say that
    $f(\cdot)$~\defn{asymptotically dominates} $g(\cdot)$,  if there
    exists constants~$c > 0$ and~$n_0 > 0$ such that for all 
    $n \ge n_0,$
    \[
      g(n) \le c \cdot f(n).
    \]
    or, equivalently, if
    \[\lim_{n \rightarrow \infty} \frac{g(n)}{f(n)} \leq c~.\]
  

\end{definition}

\begin{example}
\label{xmpl:analysis::asymptotics::function}
    In the following examples, the function $f(\cdot)$ asymptotically 
    dominates and thus grows at least as fast as the function $g(\cdot)$. 
        \[
      \begin{array}{ll}
        f(n) = 2n & g(n) = n \\
        f(n) = 2n & g(n) = 4n \\
        f(n) = n\lg{n} & g(n) = 8n \\
        f(n) = n\lg{n} & g(n) = 8n\lg{n} + 16n \\
        f(n) = n\sqrt{n} & g(n) = n\lg{n} + 2n \\
        f(n) = n\sqrt{n} & g(n) = n\lg^8{n} + 16n \\
        f(n) = n^2 & g(n) = n\lg^2{n} + 4n \\
        f(n) = n^2 & g(n) = n\lg^2{n} + 4n\lg{n} + n \\
      \end{array}
    \]
  

\end{example}
\end{flex}

\begin{cluster}
\label{grp:grm:analysis::asymptotics::ignore}

\begin{gram}
\label{grm:analysis::asymptotics::ignore}
In the definition we ignore all $n$ that are less than 
$n_0$ (i.e. small inputs), and we allow $g(n)$ to be some constant factor, $c$, larger
than $f(n)$ even though $f(n)$ ``dominates''.
When a function $f(\cdot)$ asymptotically dominates (or dominates for short)
$g(\cdot)$,  we sometimes say that $f(\cdot)$ grows as least as fast as
$g(\cdot)$

\end{gram}
\end{cluster}

\begin{flex}
\label{grp:xrcs:analysis::asymptotics::prove}

\begin{exercise}
\label{xrcs:analysis::asymptotics::prove}
Prove that for all $k$, $f(n) = n$ asymptotically dominates $g(n) =
\ln^k n$.

\textbf{Hint}: use L'Hopital's
rule, which states:
\[\mbox{if}~
\lim_{n \rightarrow \infty} f(n) = \infty~\mbox{and}~
\lim_{n \rightarrow \infty} g(n) = \infty,~\mbox{then:}~~
\lim_{n \rightarrow \infty} \frac{g(n)}{f(n)} =
\lim_{n \rightarrow \infty} \frac{g'(n)}{f'(n)}~. 
\]

\end{exercise}

\begin{solution}
\label{sol:analysis::asymptotics::applied}
We have:
\[
\begin{array}{lcl}
\displaystyle\lim_{n \rightarrow \infty} \frac{g(n)}{f(n)}
& = & \displaystyle\lim_{n \rightarrow \infty} \frac{\ln^k n}{n} \\
& = & \left( \displaystyle\lim_{n \rightarrow \infty} \frac{\ln n}{n^{1/k}}
      \right)^k\\ 
& = & \left( \displaystyle\lim_{n \rightarrow \infty} \frac{1/n}{(1/k)
      n^{1/k - 1}}
      \right)^k\\ 
& = & \left( \displaystyle\lim_{n \rightarrow \infty} \frac{k}{n^{1/k}} \right)^k\\ 
& = & 0\\
\end{array}
\]
We applied L'Hospital's rule from the second to the third line.
Since $0$ is certainly upper bounded by a constant $c$, we have that
$f$ dominates $g$.

\end{solution}
\end{flex}

\begin{cluster}
\label{grp:grm:analysis::asymptotics::functions}

\begin{gram}
\label{grm:analysis::asymptotics::functions}
For two functions $f$ and $g$ it is possible neither dominates the
other.
For example, for $f(n) = n \sin(n)$ and 
$g(n) = n \cos(n)$ neither dominates since they keep
crossing.
However, both $f$ and $g$ are dominated by $h(n) = n$. 

The dominance relation defines what is called a \defn{preorder} (distinct
from ``pre-order'' for traversal of a tree) over numeric
functions.  This means that the relation is transitive (i.e., if $f$ dominates
$g$, and $g$ dominates $h$, then $f$ dominates $h$), and reflexive
(i.e., $f$ dominates itself).

\end{gram}
\end{cluster}

\begin{flex}
\label{grp:xrcs:analysis::asymptotics::asymptotic}

\begin{exercise}
\label{xrcs:analysis::asymptotics::asymptotic}
    Prove that asymptotic dominance is transitive.
  

\end{exercise}

\begin{solution}
\label{sol:analysis::asymptotics::dominance}
    By the definition of dominance we have that 
    \begin{enumerate}
    \item 
      for some $c_a, n_a$ and all $n \geq n_a$,  $g(n) \leq c_a \cdot f(n)$,
      and
    \item
      for some $c_b, n_b$ and all $n \geq n_b$,  $h(n) \leq c_b \cdot g(n)$.
    \end{enumerate}
    By plugging in, we have that for all $n \geq \max(n_a,n_b)$
    \[h(n) \leq c_b (c_a f(n))~.\]
    This satisfies the definition  $f$ dominates $h$ with $c = c_a
    \cdot c_b$ and $n_0 = \max(n_a,n_b)$.
  

\end{solution}
\end{flex}

\begin{flex}
\label{grp:def:analysis::asymptotics::o-o-notation}

\begin{definition}[$O, \Omega, \Theta, o, \omega$ Notation]
\label{def:analysis::asymptotics::o-o-notation}
Consider the set of all numeric functions $F$, and $f \in F$.   We define the following sets:
\[
\begin{array}{lclc}
~\mbox{Name} &   & \mbox{Definition}  & \mbox{Intuitively}\\ \hline
~\mbox{big-O} & : & O(f)  = \{g \in F ~\mbox{such that}~
                    f~\mbox{dominates}~g\} & \leq f \\
~\mbox{big-Omega} & : & \Omega(f) = \{g \in F ~\mbox{such that}~ g
                        ~\mbox{dominates}~f\} & \geq f \\
~\mbox{big-Theta} & : & \Theta(f) =  O(f) \cap \Omega(f) & = f\\
~\mbox{little-o} & : & o(f) =  O(f) \setminus \Omega(f) & < f \\
~\mbox{little-omega} & : & \omega(f) =  \Omega(f) \setminus O(f) & > f\\
\end{array}
\]
Here ``$\setminus$'' means set difference.

\end{definition}

\begin{example}
\label{xmpl:analysis::asymptotics::lclcl}
\[
\begin{array}{lclcl}
f(n) & = & 2n & \in & O(n) \\
f(n) & = & 2n & \in & \Omega(n) \\
f(n) & = & 2n & \in &  \Theta(n) \\
f(n) & = & 2n & \in & O(n^2) \\
f(n) & = & 2n & \in & o(n^2) \\ 
f(n) & = & 2n & \in & \Omega(\sqrt{n}) \\
f(n) & = & 2n & \in & \omega(\sqrt{n}) \\
f(n) & = & n\lg^8{n} + 16n & \in & O(n\sqrt{n}) \\
f(n) & = & n\lg^2{n} + 4n\lg{n} + n & \in & \Theta(n \lg^2 n) \\
\end{array}
\]

\end{example}
\end{flex}

\begin{flex}
\label{grp:xrcs:analysis::asymptotics::disprove}

\begin{exercise}
\label{xrcs:analysis::asymptotics::disprove}
Prove or disprove the following statement:
if $g(n) \in O(f(n))$ and $g(n)$ is a finite function ($g(n)$ is finite
for all $n$), then it follows that there exist constants $k_1$ and
$k_2$ such that for all $n \geq 1$,
\[
g(n) \leq k_1\cdot f(n) + k_2.
\]

\end{exercise}

\begin{solution}
\label{sol:analysis::asymptotics::statement}
The statement is correct. 
Because $g(n) \in O(f(n))$, we know by the definition that there exists positive constants $c$ and $n_o$ such that for all $n \geq n_0$, $g(n) \le c \cdot f(n)$.
It follows that for the function $k_1 \cdot f(n) + k_2$ where  $k_1 = c$ and $k_2 = \sum_{i=1}^{n_0}  g(i)$,  we have $g(n) \leq k_1 \cdot f(n) + k_2$.

\end{solution}
\end{flex}

\begin{cluster}
\label{grp:grm:analysis::asymptotics::often}

\begin{gram}
\label{grm:analysis::asymptotics::often}
  We often think of $g(n) \in O(f(n))$ as indicating that $f(n)$ is
  an~\defn{upper bound} for $g(n)$ Similarly $g(n) \in \Omega(f(n))$
  indicates that $f(n)$ is a~\defn{lower bound} for $g(n)$, and
  $g(n) \in \Theta(f(n))$ indicates that $f(n)$ is
  a~\defn{tight bound} for $g(n)$.

\end{gram}
\end{cluster}


\section{Some Conventions}
\label{sec:analysis::asymptotics::conventions}

\begin{cluster}
\label{grp:grm:analysis::asymptotics::using}

\begin{gram}
\label{grm:analysis::asymptotics::using}
When using asymptotic notations, we follow some standard conventions
of convenience.

\end{gram}
\end{cluster}

\begin{cluster}
\label{grp:grm:analysis::asymptotics::writing-instead-of}

\begin{gram}[Writing $=$ Instead of $\in$]
\label{grm:analysis::asymptotics::writing-instead-of}
  In is reasonably common to write $g(n) = O(f(n))$ instead of
  $g(n) \in O(f(n))$ (or equivalently for $\Omega$ and $\Theta$).
  This is often considered abuse of notation since in this context the ``$=$'' does not
  represent any form of equality---it is not even reflexive.   In this
  book we try to avoid using ``$=$'', although we expect it still appears in
  various places.

\end{gram}
\end{cluster}

\begin{cluster}
\label{grp:grm:analysis::asymptotics::common-cases}

\begin{gram}[Common Cases]
\label{grm:analysis::asymptotics::common-cases}
By convention, and in common use, we use the following names:

\begin{tabular}{lcl}
\defn{linear} & : & $O(n)$ \\
\defn{sublinear} & : & $o(n)$ \\
\defn{quadratic} & : & $O(n^2)$ \\
\defn{polynomial} & : & $O(n^k)$, for any constant $k$.\\
\defn{superpolynomial} & : & $\omega(n^k)$, for any constant $k$.\\
\defn{logarithmic} & : & $O(\lg n)$ \\
\defn{polylogarithmic} & : & $O(\lg^k n)$, for any constant $k$.\\
\defn{exponential} & : & $O(a^n)$, for any constant $a > 1$.\\
\end{tabular}

\end{gram}
\end{cluster}

\begin{cluster}
\label{grp:grm:analysis::asymptotics::expressions-as-sets}

\begin{gram}[Expressions as Sets]
\label{grm:analysis::asymptotics::expressions-as-sets}
We typically treat expressions that involve asymptotic notation 
as sets. 
For example, in $g(n) + O(f(n))$, 
represents the set of functions $\{g(n) + h(n) : h(n) \in  f(n)\}$.
The exception is when using big-O in recurrences, which we will
discuss in \chref{analysis::recurrences}.

\end{gram}
\end{cluster}

\begin{cluster}
\label{grp:grm:analysis::asymptotics::subsets}

\begin{gram}[Subsets]
\label{grm:analysis::asymptotics::subsets}
We can use big-O ($\Omega$, $\Theta$) on both the left and right-hand sides of an equation.
In this case we are indicating that one set of functions is
a subset of the other.
For example, consider $\Theta(n) \subset O(n^2)$.
This equation indicates that the set of functions on the left-hand side is contained in
the set on the right hand side.  
Again, sometimes ``$=$'' is used instead of ``$\subset$''.

\end{gram}
\end{cluster}

\begin{cluster}
\label{grp:grm:analysis::asymptotics::the-argument}

\begin{gram}[The Argument]
\label{grm:analysis::asymptotics::the-argument}
  When writing $O(n + a)$ we have to guess what the argument of the
  function is---is it $n$ or is it $a$?   By convention we assume the
  letters $l, m$, and $n$ are
  the arguments when they appear.  A more precise notation would be
  to use $O(\lambda n . n + a)$---after all the
  argument to the big-O is supposed to be a function, not an
  expression.

\end{gram}
\end{cluster}

\begin{cluster}
\label{grp:grm:analysis::asymptotics::multiple-arguments}

\begin{gram}[Multiple Arguments]
\label{grm:analysis::asymptotics::multiple-arguments}
  Sometimes the function used in big-O notation has multiple
  arguments, as in $f(n,m) = n^2 + m \lg n$ and used in $O(f(n,m))$.
  In this case $f(n,m)$ asymptotically dominates $g(n,m)$ if there
  exists constants $c$ and $x_0 > 0$ such that for all inputs where
  $n > x_0$ or $m > x_0$, $g(n, m) \leq c \cdot f(n,m)$.  
 

\end{gram}
\end{cluster}

