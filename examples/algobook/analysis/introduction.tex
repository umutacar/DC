\chapter{Introduction}
\label{ch:analysis::introduction}

\begin{gram}[Algorithm Analysis]
The term~\defn{algorithm analysis} refers to mathematical analysis of
algorithms with the purpose of determining their consumption of
resources such as the amount of total work they perform, the energy
they consume, the time to execute, and the memory or storage space
that they require.
%

When analyzing algorithms, it is important to be \emph{precise} so
that we can compare different algorithms and assess their suitability for
our purposes.
%
It is also equally important to be \emph{abstract} because we don't
want to worry about details of compilers and computer architectures,
and because we want our analysis to remain valid even as these details
change over time.

To find the right balance between precision and abstraction, we rely
on two levels of abstraction: asymptotic analysis and cost models.
%
\begin{itemize}
\item Asymptotic analysis enables abstracting from small factors such as the
exact time a particular operation may require. 
%
\chref{analysis::asymptotics} describes the basics of asymptotic
analysis.

\item Cost models specify the cost of operations available in a
  computational model, usually only up to the precision of the
  asymptotic analysis.  \chref{analysis::models} describes
  machine-based and language-based cost models.
\end{itemize}

Many algorithms in computer science are naturally recursive.  
%
Analyses of such algorithms typically lead us to recurrences, which
are recursive mathematical relations.
%
Solving such recurrences is a basic skill for every computer
scientist.
%
\chref{analysis::recurrences} covers recurrences and the basic
techniques for solving them.

\end{gram}
