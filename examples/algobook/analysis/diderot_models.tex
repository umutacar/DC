\chapter{Cost Models}
\label{ch:analysis::models}

\begin{cluster}
\label{grp:prmbl:analysis::models::algorithmic}

\begin{preamble}
\label{prmbl:analysis::models::algorithmic}
Any algorithmic analysis must assume a~\defn{cost model}~that defines
the resource costs required by a computation.
There are two broadly accepted ways of defining cost models:
machine-based and language-based cost models.

\end{preamble}
\end{cluster}


\section{Machine-Based Cost Models}
\label{sec:analysis::models::machine-based-cost-models}

\begin{cluster}
\label{grp:def:analysis::models::machine-based-cost-model}

\begin{definition}[Machine-Based Cost Model]
\label{def:analysis::models::machine-based-cost-model}
A~\defn{machine-based (cost) model}~takes a machine model and defines the
cost of each instruction that can be executed by the machine---often
unit cost per instruction.
When using a machine-based model for analyzing an algorithm, we
translate the algorithm so that it can be executed on the machine and
then analyze the cost of the machine instructions used by the algorithm.

\end{definition}
\end{cluster}

\begin{cluster}
\label{grp:rmrk:analysis::models::machine-based}

\begin{remark}
\label{rmrk:analysis::models::machine-based}
Machine-based models are suitable for deriving asymptotic bounds
(i.e., using big-O, big-Theta and big-Omega) but not for predicting
exact runtimes.   The reason for this is that on a real machine not all
instructions take the same time, and furthermore not all machines have
the same instructions.

\end{remark}
\end{cluster}


\subsection{RAM Model}
\label{sec:analysis::models::ram-model}

\begin{cluster}
\label{grp:grm:analysis::models::classic}

\begin{gram}
\label{grm:analysis::models::classic}
The classic machine-based model for analyzing sequential algorithms is
the~\defn{Random Access Machine} or~\defn{RAM}.
In this model, a machine consists of a single processor that can
access unbounded memory; the memory is indexed by the natural numbers.
The processor interprets sequences of machine instructions (code) that
are stored in the memory.  Instructions include basic arithmetic and
logical operations (e.g. $\cd{+}$, $\cd{-}$, $\cd{*}$, and, or,
$\cd{not}$), reads from and writes to arbitrary memory locations, and
conditional and unconditional jumps to other locations in the code.
Each instruction takes unit time to execute, including those that
access memory.
The execution-time, or simply~\defn{time} of a computation is measured
in terms of the number of instructions executed by the machine.
Because the model is sequential (there is only one processor) time
and work are the same.

\end{gram}
\end{cluster}

\begin{cluster}
\label{grp:grm:analysis::models::critique-of-the-ram-model}

\begin{gram}[Critique of the RAM Model]
\label{grm:analysis::models::critique-of-the-ram-model}
Most research and development of sequential algorithms has used the RAM model for
analyzing time and space costs.
One reason for the RAM model's success is that it is relatively easy
to reason about the costs because algorithmic pseudo code can usually
be translated to the model.
Similarly, code in low-level sequential languages such as C can also
be translated (compiled) to the RAM Model relatively easily.
When using higher level languages, the translation from the algorithm to
machine instructions becomes more complex and we find ourselves making
strong, possibly unrealistic assumptions about costs, even sometimes
without being aware of the assumptions.  

More broadly, the RAM model becomes difficult to justify in modern
languages.
For example, in object- oriented languages certain operations may
require substantially more time than others.
Likewise, features of modern programming languages such as automatic
memory management can be difficult to account for in
analysis.
Functional features such as higher-order functions are even more
difficult to reason about in the RAM model because their behavior
depends on other functions that are used as arguments.
Such functional features, which are the mainstay of ``advanced''
languages such as the ML family and Haskell, are now being adopted by
more mainstream languages such as Python, Scala, and even for more primitive (closer to the machine)
languages such as C++.
All in all, it requires significant expertise to understand how an
algorithm implemented in modern languages today may be translated to the
RAM model.

\end{gram}
\end{cluster}

\begin{cluster}
\label{grp:rmrk:analysis::models::aspect}

\begin{remark}
\label{rmrk:analysis::models::aspect}
One aspect of the RAM model is the assumption that
accessing all memory locations has the same uniform cost.  On real
machines this is not the case.  In fact, there can be a factor of 100
or more
difference between the time for accessing different locations in
memory.  For example, all machines today have caches and accessing the
first-level cache is usually two orders of magnitude faster than
accessting main memory.

Various extensions to the RAM model have been developed to account for
this non-uniform cost of memory access.  One variant assumes that the
cost for accessing the $i^{th}$ memory location is $f(i)$ for some
function $f$, e.g. $f(i) = \log(i)$.  
Fortunately, however, most algorithms that are good in these more
detailed models are also good in the RAM model.
Therefore analyzing algorithms in the simpler RAM model is often a
reasonable approximation to analyzing in the more refined models.
Hence the RAM has served quite well despite not fully accounting for
non-uniform memory costs.

The model we use in this book also does not account for non-uniform
memory costs, but as with the RAM the model can be refined to account
for it.

\end{remark}
\end{cluster}


\subsection{PRAM: Parallel Random Access Machine}
\label{sec:analysis::models::pram-parallel-random-access-machine}

\begin{cluster}
\label{grp:grm:analysis::models::model}

\begin{gram}
\label{grm:analysis::models::model}
The RAM model is sequential but can be extended to use multiple processors
which share the same memory.  The extended model is called the
Parallel Random Access Machine.

\end{gram}
\end{cluster}

\begin{cluster}
\label{grp:grm:analysis::models::pram-model}

\begin{gram}[PRAM Model]
\label{grm:analysis::models::pram-model}
A \defn{Parallel Random Access Machine}, or \defn{PRAM}, 
consist of $p$ sequential random access machines
(RAMs) sharing  the same memory.
The number of processors, $p$, is a parameter of the machine,
and each processor has a unique index in $\{0, \dots, p-1\}$,
called the~\defn{processor id}.
Processors in the PRAM operate under the control of a common
clock and execute one instruction at each time step.
The PRAM model is most usually used as a {\em synchronous} model,
where all processors execute the same algorithm and operate on the same
data structures.
Because they have distinct ids, however, different processors can do
different computations.

\end{gram}
\end{cluster}

\begin{cluster}
\label{grp:ex:analysis::models::pram-array-add}

\begin{example}
\label{ex:analysis::models::pram-array-add}
We can specify a PRAM algorithm for adding one to each element of an
integer array with $p$ elements as shown below.  In the algorithm,
each processor updates certain elements of the array as determined by
its processor id, $id$.

\[
\begin{array}{l}
\cd{(* Input: integer array A. *)}
\\
\cdvar{arrayAdd} =  A[id] \la A[id]+1
\end{array}
\]
If the array is larger than $p$, then the algorithm would have to
divide the array up and into parts, each of which is updated by one
processor.

\end{example}
\end{cluster}

\begin{cluster}
\label{grp:grm:analysis::models::simd-model}

\begin{gram}[SIMD Model]
\label{grm:analysis::models::simd-model}
Because in the PRAM all processors execute the same algorithm, this
typically leads to computations where each processor executes the same
instruction but possibly on different data.   PRAM algorithms
therefore typically fit into~\defn{single instruction multiple
  data}, or \defn{SIMD}, programming model.
\exref{analysis::models::pram-array-add} shows an example SIMD program.

\end{gram}
\end{cluster}

\begin{cluster}
\label{grp:grm:analysis::models::critique-of-pram}

\begin{gram}[Critique of PRAM]
\label{grm:analysis::models::critique-of-pram}
Since the model is synchronous, and it requires the algorithm designer to map
or schedule computation to a fixed number of processors, the PRAM model can be awkward
to work with.
Adding a value to each element of an array is easy if the array
length is equal to the number of processors, but messier if not, which is
typically the case.
For computations with nested parallelism, such as divide-and-conquer
algorithms, the mapping becomes much more complicated.

We therefore don't use the PRAM model in this book.
Most of the algorithms presented in this book, however, also work with the
PRAM (with more complicated analysis), and many of them were
originally developed using the PRAM model.

\end{gram}
\end{cluster}


\section{Language Based Models}
\label{sec:analysis::models::language}

\begin{cluster}
\label{grp:grm:analysis::models::language-based-cost-models}

\begin{gram}[Language-Based Cost-Models]
\label{grm:analysis::models::language-based-cost-models}
A~\defn{language-based model}~takes a language as the starting point
and defines cost as a function mapping the expressions of the language
to their cost.
Such a cost function is usually defined as a recursive function over
the different forms of expressions in the language.  
To analyze an algorithm by using a language-based model, we apply the
cost function to the algorithm written in the language.
In this book, we use a language-based cost model, called the work-span
model.

\end{gram}
\end{cluster}


\subsection{The Work-Span Model}
\label{sec:analysis::models::language:ws}

\begin{cluster}
\label{grp:grm:analysis::models::language-based}

\begin{gram}
\label{grm:analysis::models::language-based}
Our language-based cost model is based on two cost metrics: work and
span.  Roughly speaking, the~\defn{work} of a computation corresponds to
the total number of operations it performs, and the \defn{span} corresponds
to the longest chain of dependencies in the computation.
The work and span functions can be defined for essentially any
language ranging from low-level languages such as C to higher level
languages such as the ML family.  
In this book, we use the \PML{} language. 

\end{gram}
\end{cluster}

\begin{cluster}
\label{grp:grm:analysis::models::notation-for-work-and-span}

\begin{gram}[Notation for Work and Span]
\label{grm:analysis::models::notation-for-work-and-span}
For an expressions $e$, or an algorithm written in a language, we write 
$W(e)$ for the work of $e$ and $S(e)$ for the span of $e$.

\end{gram}
\end{cluster}

\begin{cluster}
\label{grp:xmpl:analysis::models::notation}

\begin{example}
\label{xmpl:analysis::models::notation}
For example, 
the notation
\[
W(7 + 3)
\]
denotes the work of adding $7$ and $3$.

The notation 
\[
S(\cdvar{fib}(11))
\]
denotes the span for calculating the 11$^{th}$ Fibonacci number using
some particular code for $\cdvar{fib}$.

The notation 
\[
W(\cdvar{mySort}(a))
\]
denotes the work for $\cdvar{mySort}$ applied to the sequence~$a$.

\end{example}
\end{cluster}

\begin{cluster}
\label{grp:grm:analysis::models::using-input-size}

\begin{gram}[Using Input Size]
\label{grm:analysis::models::using-input-size}
Note that in the third example the sequence $a$ is not defined within
the expression.  Therefore we cannot say in general what the work is
as a fixed value.  However, we might be able to use asymptotic
analysis to write a cost in terms of the length of $a$, and in
particular if $\cd{mySort}$ is a good sorting algorithm we would have:
\[
W(\cd{mySort}(a)) = O(|a| \log |a|).
\]
Often instead of writing  $|a|$ to indicate the size of the input, we
use $n$ or $m$ as shorthand.  Also if the cost is for a particular
algorithm, we use a subscript to indicate the algorithm. This leads to
the following notation
\[
W_{\cd{mySort}}(n) = O(n \log n).
\]
where $n$ is the size of the input of $\cd{mysort}$.  When obvious
from the context (e.g. when in a section on analyzing $\cd{mySort}$)
we typically drop the subscript, writing $W(n) = O(n \log n)$.

\end{gram}
\end{cluster}

\begin{cluster}
\label{grp:def:analysis::models::pml}

\begin{definition}[\PML{} Cost Model]
\label{def:analysis::models::pml}
The work and span of \PML{} expressions are defined below.
In the definition and throughout this book, we write~$W(e)$ for the
work of the expression and~$S(e)$ for its span. 
Both work and span are cost functions that map an expression to a
integer cost.
As common in language-based models, the definition follows the
definition of expressions for \PML{} (\chref{sparc}). 
We make one simplifying assumption in the presentation: instead of
considering general bindings, we only consider the case where a single
variable is bound to the value of the expression.

In the definition, the notation $\eval{e}$ evaluates the expression
$e$ and returns the result, and the notation $[v/x]~e$ indicates that
all free (unbound) occurrences of the variable $x$ in the expression
$e$ are replaced with the value $v$.

\[
\begin{array}{lcl}
W(v) & = & 1 
\\
W(\cfn{p}{e})  & = & 1 
\\
  W(e_1~e_2) & =  & W(e_1) + W(e_2) + W([\eval{e_2}/x]~e_3) + 1 
\\
          &    & \mbox{where}~\eval{e_1} = \cfn{x}{e_3}
\\
W(e_1~\cd{op}~e_2) & = &  W(e_1) + W(e_2) + 1
\\
W(e_1~,~e_2) & = & W(e_1) + W(e_2) + 1
\\
W(e_1~||~e_2) &=& W(e_1) + W(e_2) + 1
\\[1ex]
W\left(
\begin{array}{l}
\cif~e_1
\\
\cthen~e_2
\\
\celse~e_3
\end{array}
\right)  
& = & 
\left\{
\begin{array}{ll}
W(e_1) + W(e_2) + 1 & \mbox{if}~\eval{e_1} = \cd{true}
\\
W(e_1) + W(e_3) + 1 & \mbox{otherwise}
\end{array}\right.
\\[1ex]
W\left(
\begin{array}{l}
\clet~x=e_1~
\\
\cin~e_2~\cend
\end{array}
\right) 
& = & W(e_1) + W([\eval{e_1}/x]~e_2) + 1
\\[1ex]
W( ( e ) ) & = & W(e)
\end{array}
\]
\medskip
\[
\begin{array}{lcl}
S(v) & = & 1 
\\
S(\cfn{p}{e})  & = & 1 
\\
S(e_1~e_2) & =  & S(e_1) + S(e_2) + S([\eval{e_2}/x]~e_3) + 1 
\\
          &    & \mbox{where}~\eval{e_1} = \cfn{x}{e_3}
\\
S(e_1~\mbox{op}~e_2) & = &  S(e_1) + S(e_2) + 1
\\
S(e_1~,~e_2) & = & S(e_1) + S(e_2) + 1
\\
S(e_1~||~e_2) &=& \max{}{(S(e_1),S(e_2))} + 1
\\[1ex]
S\left(
\begin{array}{l}
\cif~e_1
\\
\cthen~e_2
\\
\celse~e_3
\end{array}
\right)
& = & 
\left\{
\begin{array}{ll}
S(e_1) + S(e_2) + 1 & \eval{e_1} = \cd{true}\\
S(e_1) + S(e_3) + 1 & \mbox{otherwise}
\end{array}\right.
\\[1ex]
S\left(
\begin{array}{l}
\clet~x=e_1~
\\
\cin~e_2~\cend
\end{array}
\right) 
& = & S(e_1) + S([\eval{e_1}/x]~e_2) + 1
\\[1ex]
S( ( e ) ) & = & S(e)
\\
\end{array}
\]

\end{definition}
\end{cluster}

\begin{cluster}
\label{grp:xmpl:analysis::models::consider}

\begin{example}
\label{xmpl:analysis::models::consider}
Consider the expression $e_1 + e_2$ where $e_1$ and $e_2$ are
themselves other expressions (e.g., function application).  Note that
this is an instance of the rule $e_1~\mbox{op}~e_2$, where \mbox{op} is a plus operation.
In \PML{}, we evaluate this expressions by first evaluating $e_1$ and
then $e_2$ and then computing the sum.  The work of the expressions is
therefore
\[
W(e_1 + e_2) = W(e_1) + W(e_2) + 1
.
\]
The additional $1$ accounts for computation of the sum.   

\end{example}
\end{cluster}

\begin{flex}
\label{grp:grm:analysis::models::span}

\begin{gram}
\label{grm:analysis::models::span}
For the $\cd{let}$ expression, we first evaluate $e_1$ and assign it
to $x$ before we can evaluate $e_2$.  Hence the fact that the span is
composed sequentially, i.e., by adding the spans.

\end{gram}

\begin{example}
\label{xmpl:analysis::models::expressions}
In \pml, $\cd{let}$ expressions compose sequentially.
\[
\begin{array}{lcl}
W(\cd{let}~y = f(x)~\cd{in}~g(y)~\cd{end}) & = & 1 + W(f(x)) + W(g(y))
\\
S(\cd{let}~y = f(x)~\cd{in}~g(y)~\cd{end}) & = & 1 + S(f(x)) + S(g(y))
\end{array}
\]

\end{example}
\end{flex}

\begin{cluster}
\label{grp:tch:analysis::models::expressions}

\begin{teachask}
\label{tch:analysis::models::expressions}
In \PML{}, when are expressions evaluated in parallel?

\end{teachask}
\end{cluster}

\begin{cluster}
\label{grp:grm:analysis::models::rules}

\begin{gram}
\label{grm:analysis::models::rules}
Note that all the rules for work and span 
have the same form except for parallel application, i.e., $(e_1\ ||\ e_2)$.    
Recall that parallel application indicates that the two 
expressions can be evaluated in parallel, and the result is a pair of values 
containing the two results. 
In this case we use maximum for combining the span since we have
to wait for the longer of the two.   In all other
cases we sum both the work and span.
Later we will also add a parallel construct for working with
sequences.

\end{gram}
\end{cluster}

\begin{flex}
\label{grp:xmpl:analysis::models::expression}

\begin{example}
\label{xmpl:analysis::models::expression}
The expression $(\cd{fib}(6)\ ||\ \cd{fib}(7))$ runs the two
calls to {\tt fib} in parallel and returns the pair $(8,13)$.   It
does work 
\[
1 + W(\cd{fib}(6)) + W(\cd{fib}(7))
\] 
and span
\[
1 + \max{}(S(\cd{fib}(6)), S(\cd{fib}(7))).
\]   
If we know that the
span of $\cd{fib}$ grows with the input size, then the span can
be simplified to $1 + S(\cd{fib}(7))$.

\end{example}
\end{flex}

\begin{cluster}
\label{grp:rmrk:analysis::models::assuming}

\begin{remark}
\label{rmrk:analysis::models::assuming}
When assuming purely functional programs, it is
always safe to run things in parallel if there is no explicit
sequencing.     For the expression $e_1 + e_2$, for example, it is
safe to evaluate the
two expressions in parallel, which would give the rule
\[
S(e_1 + e_2) = \max{}(S(e_1), S(e_2)) + 1~,
\]
i.e., we wait for the later of the two expression fo finish, and then spend one
additional unit to do the addition.
However, in this book we use the convention that parallelism has to be stated
explicitly using $||$.

\end{remark}
\end{cluster}

\begin{cluster}
\label{grp:tch:analysis::models::construct}

\begin{teachnote}
\label{tch:analysis::models::construct}
As there is no $||$
construct in ML, in your assignments you will need to specify in comments when two calls run
in parallel.  We will also supply an ML function $\cd{par (f1,f2)}$ with
type 
\[
(\cd{unit} \ra \alpha) \times (\cd{unit} \ra \beta) \ra \alpha
\times \beta.
\]
This function executes the two functions that are passed in as
arguments in parallel and returns their results as a pair.  For
example:

$\cd{par (fn => fib(6), fn => fib(7))}$

returns the pair $(8,13)$.  We need to wrap the expressions in
functions in ML so that we can make the actual implementation run them
in parallel.  If they were not wrapped both arguments would be
evaluated sequentially before they are passed to the function
$\cd{par}$.  

Also in the ML code you do not have the set
notation $\{f(x) : x \in A\}$, but as mentioned before, it is
basically equivalent to a
$\cd{map}$.   Therefore, for ML code you can use the rules:

  \[ W(\mbox{\tt map } f\ \cseq{s_0, \ldots, s_{n-1}}) = 1 + \sum_{i=0}^{n-1} W(f(s_i)) \]
   \[S(\mbox{\tt map } f\ \cseq{s_0, \ldots, s_{n-1}}) = 1 + \max{}_{i=0}^{n-1} S(f(s_i)) \]

\end{teachnote}
\end{cluster}

\begin{cluster}
\label{grp:def:analysis::models::average-parallelism}

\begin{definition}[Average Parallelism]
\label{def:analysis::models::average-parallelism}
Parallelism, sometimes called \defn{average parallelism}, is
defined as the work over the span:
\[ 
\paral = \frac{W}{S}.
\]
Parallelism informs us approximately how many processors we can use
efficiently.

\end{definition}
\end{cluster}

\begin{cluster}
\label{grp:xmpl:analysis::models::mergesort}

\begin{example}
\label{xmpl:analysis::models::mergesort}
For a mergesort with work $\Theta(n \log n)$ and span
$\Theta(\log^2 n)$ the parallelism would be $\Theta(n/\log
n)$.

\end{example}
\end{cluster}

\begin{cluster}
\label{grp:xmpl:analysis::models::work}

\begin{example}
\label{xmpl:analysis::models::work}
Consider an algorithm with work $W(n) = \Theta(n^3)$ and span 
$S(n) = \Theta(n \log n)$.
For $n = 10,000$, $\paral(n) \approx 10^7$, which is a lot of
parallelism.  
But, if $W(n) = \Theta(n^2) \approx 10^8$ then
$\paral(n) \approx 10^3$, which is much less parallelism. 
Note that the decrease in parallelism is not because of an increase in
span but because of a reduction in work.

\end{example}
\end{cluster}

\begin{cluster}
\label{grp:tch:analysis::models::ways}

\begin{teachask}
\label{tch:analysis::models::ways}
What are ways in which we can increase parallelism?

\end{teachask}
\end{cluster}

\begin{cluster}
\label{grp:grm:analysis::models::designing-parallel-algorithms}

\begin{gram}[Designing Parallel Algorithms]
\label{grm:analysis::models::designing-parallel-algorithms}
In parallel-algorithm design, we aim to keep parallelism as high
as possible.
Since parallelism is defined as the amount of work per unit of span, 
we can do this by decreasing span.
We can increase parallelism by increasing work also, but this is
usually not desirable.
In designing parallel algorithms our goals are: 

\begin{enumerate}
\item   to keep work as low as possible, and
\item   to keep span as low as possible.
\end{enumerate}

Except in cases of extreme parallelism, where for example, we may have
thousands or more processors available to use, the first priority is
usually to keep work low, even if it comes at the cost of increasing
span.

\end{gram}
\end{cluster}

\begin{flex}
\label{grp:def:analysis::models::work-efficiency}

\begin{definition}[Work efficiency]
\label{def:analysis::models::work-efficiency}
 We say that a parallel algorithm is {\em
  work efficient} if it perform asymptotically the same work as the best known sequential algorithm for that problem. 

\end{definition}

\begin{example}
\label{xmpl:analysis::models::comparison-based}
A (comparison-based) parallel sorting algorithm with
$\Theta(n\log{n})$ work is work efficient; one with $\Theta(n^2)$ is
not, because we can sort sequentially with $\Theta(n\log{n})$ work.

\end{example}
\end{flex}

\begin{cluster}
\label{grp:nt:analysis::models::book}

\begin{note}
\label{nt:analysis::models::book}
In this book, we will mostly cover work-efficient algorithms where
the work is the same or close to the same as the best sequential time.
Among the algorithm that have the same work as the best sequential
time, our goal will be to achieve the greatest parallelism.

\end{note}
\end{cluster}


\subsection{Scheduling}
\label{sec:analysis::models::language::scheduling}

\begin{cluster}
\label{grp:grm:analysis::models::scheduling}

\begin{gram}
\label{grm:analysis::models::scheduling}
Scheduling involves executing a parallel program by mapping the
computation over the processors in such a way to minimize
the completion time and possibly, the use of other resources such as
space and energy.
There are many forms of scheduling.
This section describes the scheduling problem and briefly reviews one
particular technique called greedy scheduling.

\end{gram}
\end{cluster}


\subsubsection{Scheduling Problem}
\label{sec:analysis::models::scheduling-problem}

\begin{flex}
\label{grp:grm:analysis::models::advantage}

\begin{gram}
\label{grm:analysis::models::advantage}
An important advantage of the work-span model is that it allows us to
design parallel algorithms without having to worry about the details
of how they are executed on an actual parallel machine. 
In other words, we never have to worry about mapping of the parallel
computation to processors, i.e., scheduling. 

Scheduling can be challenging, because a parallel algorithm generates
independently executable \defn{tasks} on the fly as it runs, and it
can generate a large number of them, typically many more than the
number of processors.

\end{gram}

\begin{example}
\label{xmpl:analysis::models::parallel}
A parallel algorithm with $\Theta(n/\lg{n})$ parallelism can easily
generate millions of parallel subcomptutations or task at the same time,
even when running on a multicore computer with $10$ cores.
For example, for $n = 10^8$, the algorithm may generate millions of
independent tasks.

\end{example}
\end{flex}

\begin{cluster}
\label{grp:def:analysis::models::scheduler}

\begin{definition}[Scheduler]
\label{def:analysis::models::scheduler}
A~\defn{scheduling algorithm} or a~\defn{scheduler} is an algorithm
for mapping parallel tasks to available processors.
The scheduler works by taking all parallel tasks, which are generated
dynamically as the algorithm evaluates, and assigning them to
processors.  
If only one processor is available, for example, then all
tasks will run on that one processor.  If two processors are available,
the task will be divided between the two.

Schedulers are typically designed to minimize the execution time of a
parallel computation,
but 
minimizing space usage is also important.

\end{definition}
\end{cluster}

\begin{cluster}
\label{grp:tch:analysis::models::think}

\begin{teachask}
\label{tch:analysis::models::think}
Can you think of a scheduling algorithm?

\end{teachask}
\end{cluster}


\subsubsection{Greedy Scheduling}
\label{sec:analysis::models::greedy-scheduling}

\begin{cluster}
\label{grp:def:analysis::models::greedy-scheduler}

\begin{definition}[Greedy Scheduler]
\label{def:analysis::models::greedy-scheduler}
We say that a scheduler is~\defn{greedy} if whenever there is a
processor available and a task ready to execute, then it assigns the
task to the processor and starts running it immediately.  Greedy
schedulers have an important property that is summarized by the greedy
scheduling principle.

\end{definition}
\end{cluster}

\begin{cluster}
\label{grp:def:analysis::models::greedy}

\begin{definition}[Greedy Scheduling Principle]
\label{def:analysis::models::greedy}
The \defn{greedy scheduling principle} postulates that if a
computation is run on $P$ processors using a greedy scheduler, then
the total time (clock cycles) for running the computation is bounded
by
\[
\begin{array}{lll}
T_P & < & \frac{W}{P} + S
\end{array}
\]
where $W$ is the work of the computation, and $S$ is the span of the
computation (both measured in units of clock cycles).

\end{definition}
\end{cluster}

\begin{cluster}
\label{grp:grm:analysis::models::optimality-of-greedy-schedulers}

\begin{gram}[Optimality of Greedy Schedulers]
\label{grm:analysis::models::optimality-of-greedy-schedulers}
This simple statement is powerful. 

Firstly, the time to execute the computation cannot be less than
$\frac{W}{P}$ clock cycles since we have a total of $W$ clock cycles
of work to do and the best we can possibly do is divide it evenly
among the processors.

Secondly, the time to execute the computation cannot be any less than
$S$ clock cycles, because $S$ represents the longest chain of
sequential dependencies.  Therefore we have

\[
T_p \geq \max{}\left(\frac{W}{P},S\right).
\]

We therefore see that a greedy scheduler does reasonably close to the
best possible.  In particular $\frac{W}{P} + S$ is never more than
twice $\max{}(\frac{W}{P},S)$.


Furthermore, greedy scheduling is particularly good for algorithms
with abundant parallellism.  To see this, let's rewrite the inequality
of the Greedy Principle
in terms of the parallelism $\paral = W/S$:
\[
\begin{array}{lll}
T_P & < & \frac{W}{P} + S \\
    & = &  \frac{W}{P} + \frac{W}{\paral{}}\\
    & = &  \frac{W}{P}\left(1 + \frac{P}{\paral{}}\right).
\end{array}
\]
Therefore, if $\paral{} \gg P$, i.e., the parallelism is much greater
than the number of processors, then the parallel time $T_P$ is close
to $W/P$, the best possible.  In this sense, we can view parallelism
as a measure of the number of processors that can be used effectively.

\end{gram}
\end{cluster}

\begin{cluster}
\label{grp:def:analysis::models::speedup}

\begin{definition}[Speedup]
\label{def:analysis::models::speedup}
The \defn{speedup} $S_P$ of a $P$-processor parallel execution over a
sequential one is defined as
\[
S_P = T_s / T_P, 
\]
where $T_S$ denotes the sequential time.
We use the term \defn{perfect speedup} to refer to a speedup that is
equal to $P$.

When assessing speedups, it is important to select the best sequential
algorithm that solves the same problem (as the parallel one).  

\end{definition}
\end{cluster}

\begin{cluster}
\label{grp:xrcs:analysis::models::describe}

\begin{exercise}
\label{xrcs:analysis::models::describe}
Describe the conditions under which a parallel algorithm would obtain
near perfect speedups.

\end{exercise}
\end{cluster}

\begin{cluster}
\label{grp:rmrk:analysis::models::greedy}

\begin{remark}
\label{rmrk:analysis::models::greedy}
Greedy Scheduling Principle does not account for the time it requires
to compute the (greedy) schedule, assuming instead that such a
schedule can be created instantaneously and at no cost.
This is of course unrealistic and there has been much work on
algorithms that attempt to match the Greedy Scheduling Principle.
No real schedulers can match it exactly, because scheduling itself
requires work.
For example, there will surely be some delay from when a task becomes
ready for execution and when it actually starts executing.
In practice, therefore, the efficiency of a scheduler is quite
important to achieving good efficiency.  
Because of this, the greedy scheduling principle should only be viewed
as an asymptotic cost estimate in much the same way that the RAM model
or any other computational model should be just viewed as an
asymptotic estimate of real time.

\end{remark}
\end{cluster}

