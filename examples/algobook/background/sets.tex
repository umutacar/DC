\chapter{Sets and Relations}
\label{ch:bg::sets-and-relations}

\begin{preamble}
This chapter presents a review of some basic definitions on sets and
relations.
\end{preamble}

\section{Sets}
\label{sec:bg::sets}


\begin{gram}
A~\defn{set} is a collection of distinct objects.  The objects that
are contained in a set, are called~\defn{members} or
the~\defn{elements} of the set.  The elements of a set must be
distinct: a set may not contain the same element more than once. The
set that contains no elements is called the~\defn{empty set} and is
denoted by $\{\}$ or $\emptyset$.
\end{gram}

\begin{gram}[Specification of Sets]
Sets can be specified intentionally, by mathematically describing
their members.
%
For example, the set of natural numbers, traditionally written as
$\tynat$, can be specified~\defn{intentionally} as the set of all
nonnegative integral numbers.
%
Sets can also be specified~\defn{extensionally} by listing their members.
For example, the set $\tynat = \{0, 1, 2, \ldots\}.$
%
We say that an element $x$ is a {\em member of} $A$, written $x \in
A$, if $x$ is in $A$.  
%
More generally, sets can be specified using~\defn{set comprehensions},
which offer a compact and precise way to define sets by mixing
intentional and extensional notation.
\end{gram}

\begin{definition}[Union and Intersection]
For two sets $A$ and $B$, the~\defn{union} $A \cup B$ is defined as
the set containing all the elements of $A$ and $B$.  Symmetrically,
their~\defn{intersection}, $A \cap B$ is the defined as the set
containing the elements that are member of both $A$ and $B$.
% 
We say that $A$ and $B$ are~\defn{disjoint} if their intersection is
the empty set, i.e., $A \cap B = \emptyset$.
%
\end{definition}

\begin{flex}
\begin{definition}[Cartesian Product]
Consider two sets~$A$ and~$B$.  The~\defn{Cartesian product $A \times
  B$} is the set of all ordered pairs $(a,b)$ where $a \in A$ and $b
\in B$, i.e.,
%
\[
A \times B = \cset{(a,b) : a \in A, b \in B}.
\]
\end{definition}

\begin{example}
\label{ex:bg::sets::cartesian}
The Cartesian product of $A = \cset{0,1,2,3}$ and $B = \cset{a,b}$ is
\[
\begin{array}{lll}
A \times B = & \{ & (0,a),(0,b),(1,a),(1,b),
\\
             &    & (2,a),(2,b),(3,a),(3,b) 
\\
             & \}.
\end{array}
\]
\end{example}
\end{flex}

\begin{flex}
\begin{definition}[Set Partition]
Given a set $A$, a partition of $A$ is a set $P$ of non-empty subsets
of $A$ such that each element of $A$ is in exactly one subset in $P$.
%
We refer to each element of $P$ as a~\defn{block} or a~\defn{part} and
the set $P$ as a~\defn{partition} of $A$.
%
More precisely, $P$ is a partition of $A$ if the following conditions
hold:
\begin{itemize}
\item if $B \in P$, then $B \not= \emptyset$,
\item if $A = \bigcup_{B \in P}{B}$, and
\item if $B, C \in P$, then $B = C$ or $B \cap C = \emptyset$.
\end{itemize}
\end{definition}

\begin{example}
If $A = \{1, 2, 3, 4, 5, 6 \}$ then $P = \{ \{1,3,5\}, \{2, 4, 6\} \}$
is a partition of $A$.  The set $\{1,3,5\}$ is a block.

The set $Q = \{ \{1,3,5,6\}, \{2, 4, 6\} \}$ is a not partition of
$A$, because the element $6$ is contained multiple blocks.
\end{example}
\end{flex}

\begin{flex}
\begin{definition}[Kleene Operators]
For any set $\Sigma$, its~\defn{Kleene star} $\Sigma^*$ is the set of
all possible strings consisting of members of $\Sigma$, including the
empty string.

For any set $\Sigma$, its~\defn{Kleene plus} $\Sigma^+$ is the set of
all possible strings consisting of members $\Sigma$, excluding the
empty string.
\end{definition}

\begin{example}
Given $\Sigma = \{\texttt{a},\texttt{b}\}$,
\[
\begin{array}{rl}
\Sigma^* 
%
= 
%
\{
&
\str{}, 
%
\\
&
\str{\texttt{a}}, \str{\texttt{b}}, 
%
\\
&
\str{\texttt{aa}}, \str{\texttt{ab}}, 
\str{\texttt{ba}}, \str{\texttt{bb}}, 
\\
%
&
\str{\texttt{aaa}}, \str{\texttt{aab}}, \str{\texttt{aba}},
\str{\texttt{abb}},
\\
& \str{\texttt{baa}}, \str{\texttt{bab}}, \str{\texttt{bba}}, \str{\texttt{bbb}},
\\
%
&
\ldots
\\
\} &
\end{array}
\]
%
and
%
\[
\begin{array}{rl}
\Sigma^+ 
%
= 
%
\{
&
\str{\texttt{a}}, \str{\texttt{b}}, 
\\
%
&
\str{\texttt{aa}}, \str{\texttt{ab}}, 
\str{\texttt{ba}}, \str{\texttt{bb}}, 
\\
%
&
\str{\texttt{aaa}}, \str{\texttt{aab}}, \str{\texttt{aba}}, \str{\texttt{abb}}, 
\\
&
\str{\texttt{baa}}, \str{\texttt{bab}}, \str{\texttt{bba}}, 
\str{\texttt{bbb}},
\\
%
& \ldots
\\
\} & 
\\
\end{array}
\]
\end{example}
\end{flex}

%% \begin{checkpoint}

%% \begin{questionma}
%% \points 10
%% \prompt Which one of the following are well-defined sets? 
%% \select[1]
%% $\{1, 2, 3\}$
%% \select[0]
%% $\{1, 1, 2, 3\}$
%% \select[1]
%% Sets of all sets that are not members of themselves.
%% \explain
%% This set plays the key role in Russel's paradox.

%% \select[1]
%% $\{'i', 'am', 'new'\}$

%% \select[1]
%% $\{'i', 'am', 'also', 'new'\}$
%% \end{questionma}

%% %% \begin{questionma}
%% %% \points 10
%% %% \prompt Which one of the following are well-defined sets? 
%% %% \select*
%% %% $\{1, 2, 3\}$
%% %% \select[0]
%% %% $\{1, 1, 2, 3\}$
%% %% \select*
%% %% Sets of all sets that are not members of themselves.
%% %% \explain
%% %% This is called Russel's paradox.

%% %% \end{questionma}

%% % This is an example of free-answer quesition with 
%% % simple or flat answers
%% \begin{questionfr}
%% \points 10
%% \prompt Let $A$ be a set of natural numbers of size $n$.  How many
%% unique numbers are there in $A$. 
%% \hint Definition of sets.

%% \answer[10] A sets consists of unique elements.  Thus there are exactly
%% $n$ numbers.

%% \end{questionfr}


%% \begin{questionfr}
%% \points 10
%% \prompt Prove that $A \cap (B \cup C) = (A \cap B) \cup (A \cap C)$
%% \begin{answer}[10]
%% Let $x$ be an element on the left side, then we know that $x \in A$
%% and that $x \in B$ or $x \in C$.  This means that $x$ is a member of
%% one of the two sets being unioned on the right. 

%% Conversely, let $x$ be an element on the right hand side, then we know
%% that $x \in A$ and $x \in B$ or $x \in C$.  In either case $x \in A
%% \cap (B \cup C)$.
%% \end{answer}
%% \begin{answer}[10]
%% This is one of the two distributive laws of sets: which are
%% \begin{eqnarray}
%% A \cap (B \cup C) & = (A \cap B) \cup (A \cap C)
%% \\
%% A \cup (B \cap C) & = (A \cup B) \cap (A \cup C)
%% \end{eqnarray}
%% \end{answer}
%% \end{questionfr}

%% \begin{questionfr}
%% \points 10
%% \prompt List all the partitions of the set $\{1, 2, 3 \}$.
%% \answer
%% \end{questionfr}

%% \end{checkpoint}

\begin{exercise}
Prove that Kleene star and Kleene plus are closed under
string concatenation. 
\end{exercise}


\section{Relations}
\label{sec:bg::relations}


\begin{definition}[Relation]
%\label{def:bg::relations}

A~\defn{(binary)~\defn{relation} from a set~$A$ to set~$B$} is a
subset of the Cartesian product of~$A$ and~$B$.  
%
For a relation~$R \subseteq A \times B$, 
\begin{itemize}
\item the set~$\cset{a : (a,b) \in
  R}$ is referred to as the~\defn{domain} of $R$, and 
\item 
the set~$\cset{b
  : (a,b) \in R}$ is referred to as the~\defn{range} of $R$.
\end{itemize} 
\end{definition}


\begin{definition}[Function]
A~\defn{mapping} or
%
\defn{function from $A$ to $B$}
%
is a relation $R \subset A \times B$
such that $|R| = |\mbox{domain}(R)|$, i.e., for every $a$ in the
domain of $R$ there is only one $b$ such that $(a,b) \in R$.  
%
The \defn{range} of the function is the range of $R$.
%
We call the set $A$ the \defn{domain} and the $B$ the \defn{co-domain}
of the function.
%
\end{definition}


\begin{example}
\label{ex:bg::relations::sequencesdef}
Consider the sets $A = \cset{0,1,2,3}$ and $B = \cset{a,b}$.

The set:
\[X = \cset{(0,a),(0,b),(1,b),(3,a)}\]
is a relation from $A$ to $B$ since $X \subset A \times B$, but not a mapping (function) since $0$
is repeated.   

The set
\[Y = \cset{(0,a),(1,b),(3,a)}\]
is both a relation and a function from $A$ to $B$ since each element
only appears once on the left.  


%% Commented out because sequence is not defined.
%%
%The domain of $Y$ is $\cset{0,1,3}$ and the range is $\cset{a,b}$.  It
%is, however, not a sequence since there is a gap in the domain.

\end{example}

%% \begin{checkpoint}
%% \begin{questionfr}
%% \points 10
%% \prompt 
%% How many functions are there from from the set $\{1, 2, 3\}$ to $\{a, b\}$.
%% \answer
%% \end{questionfr}

%% \begin{questionfr}
%% \points 10
%% \prompt 
%% How many relations are there from from the set $\{1, 2, 3\}$ to $\{a, b\}$.
%% \answer
%% \end{questionfr}

%% \begin{questionfr}
%% \points 10
%% \prompt 
%% Let $f(\cdot): A \ra B$, i.e., $f(\cdot)$ is a function from the set
%% $A$ to $B$.  We can define the inverse function, typically written
%% $f^{-1} (\cdot)$ as a function from $B$ to $A$ such that $f^{-1}(f(x))
%% = x$ for all $x \in A$.  Does such the inverse of a function always
%% exist?  Prove or disprove.
%% \answer
%% \end{questionfr}


%% \end{checkpoint}

