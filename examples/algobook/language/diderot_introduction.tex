\chapter{Introduction}
\label{ch:language:introduction}

\begin{cluster}
\label{grp:grm:language:introduction::book}

\begin{gram}
\label{grm:language:introduction::book}
  In this book we define algorithms and data structures using nested
  parallelism in conjunction with a functional programming style.  Our
  opinion is that this is the best way to capture the core ideas of
  algorithms and parallelism in a concise, clear, safe, and precise way.
  Most of the ideas we present, however, transcend the particular
  style of parallelism and programming we use and will be useful in a
  broad set of programming languages.

\end{gram}
\end{cluster}

\begin{cluster}
\label{grp:grm:language:introduction::nested-parallelism}

\begin{gram}[Nested parallelism]
\label{grm:language:introduction::nested-parallelism}
  Nested parallelism (or nested fork-join parallelism) is a style of
  parallelism in which any task can fork a set of new child tasks to
  run in parallel.  When forking, the parent task suspends, and when
  all the child tasks finish, they ``join'', and the parent continues.
  Since any task can fork new tasks, the forking can be nested.
  Nested parallelism supports a form of parallelism in which
  computation can be cleanly composed either sequentially (within a
  task), or in parallel (among forked tasks).  This in turn leads a
  simple cost model based on analyzing work and span.

  Importantly, the model is sufficiently powerful to capture the
  parallelism in most of the algorithms needed for the purpose of this
  book.  For dynamic programming, we diverge slightly from this model
  to a somewhat more general model.

\end{gram}
\end{cluster}

\begin{cluster}
\label{grp:grm:language:introduction::functional-algorithms}

\begin{gram}[Functional Algorithms]
\label{grm:language:introduction::functional-algorithms}
  Functional programming is a style of programming in which functions
  act like mathematical functions (a mapping from domain to a
  codomain, and no side effects), and can be uses as values (can be
  passed around, stored as data, and created on the fly).  In this
  book we use this style for two important reasons.

\begin{enumerate}
\item Since functions have no side effects, parallelism is inherently
  safe and deterministic. Functions can be applied in parallel, or in
  different orders, without effecting each others outcomes, or the
  result of the final computation.
\item The ability to use functions as values allows powerful
  abstractions.  A large fraction of the functions in many of the
  abstract data types we define, for example, take functions as
  arguments to other functions. Often these functions are created on
  the fly.
\end{enumerate}

The functional programming style is not limited to functional
programming languages.  Today most programming languages support it,
and in many situations the style has become dominant.  

\end{gram}
\end{cluster}

\begin{cluster}
\label{grp:grm:language:introduction::minimal}

\begin{gram}[\pml{}]
\label{grm:language:introduction::minimal}
We use a minimal, perhaps ``toy'',  language called \pml{} to describe algorithms
and data structures.   It only supplies what we need for the purposes of the book,
and is not meant to be fully precise.   We will sometime substitute text for code.  \pml{} has structures for supporting nested
parallelism and supports only functional programming---it does not allow for
side effects.

\pml{}, like many functional languages, is an extension of the
lambda-calculus, which is arguably the first programming language
and the basis of many ideas in modern programming languages.
\chref{lambda-calculus} gives a very brief overview of the lambda
calculus, and \chref{sparc} describes \pml{}.

\end{gram}
\end{cluster}

\begin{cluster}
\label{grp:grm:language:introduction::function-vs-algorithm}

\begin{gram}[Function vs. Algorithm]
\label{grm:language:introduction::function-vs-algorithm}
  Finally, we note that although functions in the functional
  programming style act like mathematical functions---i.e. a mapping
  from inputs to outputs---each is more than just a mathematical
  function.  They not only embody the mapping, but they also specify
  the mechanism (code) by which the output is generated from the
  input.  There can be multiple different definitions of a function
  that describe the same mathematical function, but compute it in
  different ways.  Functions are therefore more accurately algorithms,
  and the input to output map they define is the mathematical
  function.

\end{gram}
\end{cluster}

