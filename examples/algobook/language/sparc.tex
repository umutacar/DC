\chapter{The SPARC Language}
\label{ch:sparc}

\begin{preamble}
This chapter presents \pml{}: a parallel and functional language used
throughout the book for specifying algorithms.

%
\pml is a ``strict'' functional language similar to the ML class of
languages such as Standard ML or SML, Caml, and F\#.
%
In pseudo code, we sometimes use mathematical notation, and even
English descriptions in addition to \pml{} syntax.
%
This chapter describes the basic syntax and semantics of \pml{}; we
introduce additional syntax as needed in the rest of the book.
\end{preamble}

\section{Syntax and Semantics of \pml{}}

\begin{gram}
This section describes the syntax and the semantics of the core subset
of the \pml{} language.  The term~\defn{syntax}~refers to the
structure of the program itself, whereas the
term~\defn{semantics}~refers to what the program computes.
%
Since we wish to analyze the cost of algorithms, we are interested in
not just what algorithms compute, but how they compute.
%
Semantics that capture how algorithms compute are
called~\defn{operational semantics}, and when augmented with specific
costs,~\defn{cost semantics}.
%
Here we describe the syntax of \pml{} and present an informal
description of its operational semantics. 
%
We will cover the cost semantics of \pml in \chref{analysis::models}.
%
While we focus primarily on the core subset of \pml, we also describe
some~\defn{syntactic sugar}~that makes it easier to read or write code
without adding any real power.
%
Even though \pml is a strongly typed language, for our purposes in
this book, we use types primarily as a means of describing and
specifying the behavior of our algorithms.
%
We therefore do not present careful account of \pml's type system.
\end{gram}


\begin{gram}
The definition below shows the syntax of \pml{}.
%
A \pml{} program is an expression, whose syntax,
 describe the computations that can be expressed in \pml{}.
%
When evaluated an expression yield a value.
%
Informally speaking, evaluation of an expression proceeds involves
evaluating its sub-expressions to values and then combining these
values  to compute the value of the expression.
%
\pml{} is a strongly typed language, where every closed expression,
which have no undefined (free) variables, evaluates to a value or runs
forever.
%
\end{gram}

\begin{definition}[\pml{} expressions]
\label{def:sparc::syntax}
\setlength{\tabcolsep}{20mm}
\begin{tabular}{llcll}
Identifier
& $id$ & := & $\ldots$
\\

%
Variables
& $x$ & := & $id$
\\

Type Constructors 
& $tycon$ & := & $id$
\\

Data Constructors 
& $dcon$ & := & $id$
\\

Patterns & $p$ 
& := & x & variable
\\
& & $\mid$  & $( p )$ & parenthesis
\\
& & $\mid$  & $p_1, p_2$ & pair
\\
& & $\mid$  & $dcon~( p )$ & data pattern
\\

Types & $\tau$ &  := & $\tyint$  & integers
\\
& &  $\mid$ &   $\tybool$  & booleans
\\
& &  $\mid$  & $\tau$ $[ * \tau ]^+$  & products
\\
& &  $\mid$  & $\tau \ra \tau$ & functions
\\
& & $\mid$ & $tycon$         &  type constructors
\\
& & $\mid$ & $dty$ & data types
\\

Data Types & $dty$
& := &   $dcon~[\cd{of}~\tau]$ 
\\
&  & $\mid$ &  $dcon~[ \cd{of}~\tau ]~\cd{|}~dty$
\\

Values & $v$
& := & $0  \mid  1  \mid  \ldots$ &  integers
\\
& & $\mid$ &  $-1  \mid -2  \mid \ldots$ &  integers
\\
& & $\mid$ & $\ctrue{} \mid  \cfalse{}$ & booleans
\\
& & $\mid$ & $\cd{not}  \mid \ldots$ & unary operations
\\
& & $\mid$ & $\cand  \mid  \cd{plus}  \mid  \ldots$ & binary operations
\\
& & $\mid$ & $v_1, v_2$  & pairs
\\
& & $\mid$ & $( v )$   & parenthesis
\\
& & $\mid$ & $dcon~( v )$  & constructed data
\\
& & $\mid$ & $\cfn{p}{e}$ & lambda functions
\\

Expression  & $e$ 
& := & $x$ & variables
\\
& & $\mid$ & $v$ & values
\\
& & $\mid$ & $e_1$ op $e_2$ & infix operations
\\
& & $\mid$ & $e_1, e_2$ & sequential pair
\\
& & $\mid$ & $e_1 \cd{||} e_2$ & parallel pair
\\
& & $\mid$ & $( e )$ & parenthesis
\\
& & $\mid$ & $\ccase~e_1~[\cd{|}~p~\cdra~e_2]^+$  & case
\\
& & $\mid$ & $\cif~e_1~\cthen~e_2~\celse~e_3$  & conditionals
\\
& & $\mid$ & $e_1~e_2$ & \mbox{function application}
\\
& & $\mid$ & $\clet~b^+~\cin~e~\cend$ & \mbox{local bindings}
\\

Operations & $op$ & := & $+ \mid - \mid * \mid - \ldots$
\\

Bindings &  $b$ & := & 
$x ( p )$ = $e$ & bind function
\\
&  & $\mid$ & $p = e$ & bind pattern
\\
&  & $\mid$ & $\ctype~tycon = \tau$ & bind type 
\\
&  & $\mid$ & $\ctype~tycon = dty$ & bind datatype
\\

\end{tabular}

\end{definition}

\begin{gram}[Identifiers]
In \pml, variables, type constructors, and data constructors are given
a name, or an~\defn{identifier}.  
%
An identifier consist of only alphabetic and numeric characters (a-z,
A-Z, 0-9), the underscore character (``\_''), and optionally end with
some number of ``primes''.  
%
Example identifiers include, $x'$, $x_1$, $x_l$, $\cdvar{myVar}$,
$\cdvar{myType}$, $\cdvar{myData}$, and $\cdvar{my\_data}$.



Program~\defn{variables},~\defn{type constructors}, and~\defn{data
  constructors}~are all instances of identifiers.
%
%% Free or bound variables
During evaluation of a \pml expression, variables are bound to values,
which may then be used in  a computation later.  
%
In \pml, variable are~\defn{bound}~during function application, as part
of matching the formal arguments to a function to those specified by
the application, and also by $\cd{let}$ expressions.
%
If, however, a variable appears in an expression but it is not bound
by the expression, then it is~\defn{free}~in the expression.
%
We say that an expression is~\defn{closed}~if it
has no free variables.


Types constructors give names to types.  For example, the type of
binary trees may be given the type constructor $\cdvar{btree}$.
%
Since for the purposes of simplicity, we rely on mathematical rather
than formal specifications, we usually name our types behind
mathematical conventions.
%
For example, we denote the type of natural numbers by $\tynat$, the type
of integers by $\tyint$, and the type of booleans by $\tybool$.


Data constructors serve the purpose of making complex data structures.
%
By convention, we will capitalize data constructors, while starting
variables always with lowercase letters.
%
\end{gram}


\begin{gram}[Patterns]
In \pml, variables and data constructors can be used to construct more
complex~\defn{patterns}~over data.
%
For example, a pattern can be a pair $(x,y)$, or a triple of
variables $(x,y,z)$, or it can consist of a data constructor
followed by a pattern, e.g., $\cdvar{Cons}(x)$ or $\cdvar{Cons}(x,y).$
%
Patterns thus enable a convenient and concise way to pattern match
over the data structures in \pml.
\end{gram}

\begin{gram}[Built-in Types]
Types of \pml include base types such as integers $\tyint$, booleans
$\tybool$, product types such as $\tau_1 * \tau_2 \ldots \tau_n$,
function types $\tau_1 \ra \tau_2$ with domain $\tau_1$ and range
$\tau_2$, as well as user defined data types.
%
\end{gram}

\begin{gram}[Data Types]
In addition to built-in types, a program can define new~\defn{data types}~as a
union of tagged types, also called variants, by ``unioning'' them via
distinct~\defn{data constructors}.
%
For example, the following data type defines a point as a
two-dimensional or a three-dimensional coordinate of integers.
%
%
\[
\begin{array}{lcl}
\cd{type}~\cdvar{point} & = & \cdvar{PointTwo}~\cd{of}~\tyint * \tyint
\\
           & | & \cdvar{Point3D}~\cd{of}~\tyint * \tyint * \tyint
\end{array} 
\]
\end{gram}

\begin{gram}[Recursive Data Types]
In \pml recursive data types are relatively easy to define and compute
with. For example, we can define a point list data type as follows
%
\[
\begin{array}{l}
\cd{type}~\cdvar{plist} = \cdvar{Nil}~|~\cdvar{Cons}~\cd{of}~\cdvar{point} * \cdvar{plist}.
\end{array}
\] 
%
Based on this definition the list 
%
\[
\begin{array}{l}
\cdvar{Cons}(\cdvar{PointTwo} (0,0),  
\\
~~~~~~~~~~\cdvar{Cons}(\cdvar{PointTwo} (0,1), 
\\
~~~~~~~~~~~~~~~~~~~~\cdvar{Cons}(\cdvar{PointTwo}(0,2), \cdvar{Nil})))  
\end{array} 
\]
%
defines a list consisting of three points.
\end{gram}

\begin{flex}
\begin{exercise}[Booleans]
Some built-in types such as booleans, $\tybool$, are in fact syntactic
sugar and can be defined by using union types as follows.
%
Describe how you can define booleans using data types of \pml{}.
\end{exercise}

\begin{solution}
Booleans can be defined as follows.

\[
\begin{array}{l}
\cd{type}~\cdvar{myBool} = \cdvar{myTrue}~|~\cdvar{myFalse} 
\end{array}
\] 
%
\end{solution}
\end{flex}


\begin{gram}[Option Type]
Throughout the book, we use~\defn{option}~types quite frequently.
%
Option types for natural numbers can be defined as follows.
%
\[
\begin{array}{l}
\cd{type}~\cdvar{option} = \cdvar{None} ~|~ \cdvar{Some}~\cd{of}~\tynat
\end{array}
\]
%
Similarly, we can define option types for integers.
%
\[
\begin{array}{l}
\cd{type}~\cdvar{intOption} = \cdvar{INone} ~|~ \cdvar{ISome}~\cd{of}~\tyint
\end{array}
\]
%
Note that we used a different data constructor for naturals.  
%
This is necessary for type inference and type checking.
%
Since, however, types are secondary for our purposes in this book, we
are sometimes sloppy in our use of types for the sake of simplicity.
%
For example, we use throughout $\cdvar{None}$ and $\cdvar{Some}$ for option
types regardless of the type of the contents.
\end{gram}

\begin{teachnote}
TODO: SEQUENCES etc.SETS.
\end{teachnote}


\begin{gram}[Values]
Values of \pml, which are the irreducible units of computation
include natural numbers, integers, Boolean values $\ctrue$ and $\cfalse$,
unary primitive operations, such as boolean negation $\cd{not}$,
arithmetic negation $\cminus$, as well as binary operations such as
logical and $\cand$ and arithmetic operations such as $\cplus$.
%
Values also include constant-length tuples, which correspond to
product types, whose components are values.
%
Example tuples used commonly through the book include binary tuples or
pairs, and ternary tuples or triples.
%
Similarly, data constructors applied to values, which correspond to
sum types, are also values.
%

As a functional language, \pml treats all function as values. 
%
The anonymous function
%
$\cd{lambda}~p.~e$
%
is a function whose arguments are specified by the pattern $p$, and
whose body is the expression $e$.
%
\end{gram}

\begin{example}

\begin{itemize}

\item

The function 
%
$\cd{lambda}~x. x + 1$ takes a single variable as an argument and
adds one to it.
%

\item
The function 
%
$\cd{lambda}~(x,y).~x$ takes a pairs as an argument and
returns the first component of the pair.
\end{itemize}
\end{example}


\begin{gram}[Expressions]
Expressions, denoted by $e$ and variants (with subscript, superscript,
prime), are defined inductively, because in many cases, an expression
contains other expressions.
%
Expressions describe the computations that can be expressed in \pml. 
%
Evaluating an expression via the operational semantics of \pml
produce the value for that expression.
%
\end{gram}

\begin{gram}[Infix Expressions]
An~\defn{infix expression}, $e_1~\cdvar{op}~e_2$, involve two expressions
and an infix operator $\cdvar{op}$.  The infix operators include $+$
(plus), $-$ (minus), $*$ (multiply), $/$ (divide),
$<$ (less), $>$ (greater), $\cd{or}$, and $\cd{and}.$
%
For all these operators the infix expression
$e_1~\cdvar{op}~e_2$ is just syntactic sugar for $f(e_1, e_2)$ where
$f$ is the function corresponding to the operator $\cdvar{op}$ (see
parenthesized names that follow each operator above).  
%

We use standard precedence rules on the operators to indicate their
parsing.  For example in the expression
\[
\cd{3 + 4 * 5}
\]
the $*$ has a higher precedence than $+$ and therefore the
expression is equivalent to $3 + (4 * 5)$. 
%

Furthermore all operators are left associative unless stated
otherwise, i.e., that is to say that $a~\cdvar{op}_1~b~\cdvar{op}_2~c =
(a~\cdvar{op}_1~b)~\cdvar{op}_2~c$ if $\cdvar{op}_1$ and $\cdvar{op}_2$ have the
same precedence.  
\end{gram}

%
\begin{example}
The expressions $5 - 4 + 2$ evaluates to $(5-4) + 2 = 3$ not
$5 - (4 + 2) = -1$, because $-$ and $+$ have the same
precedence.
\end{example}

\begin{gram}[Sequential and Parallel Composition]
Expressions include two special infix operators: ``$,$'' and
$||$, for generating ordered pairs, or tuples, either
sequentially or in parallel.
%

The~\defn{comma} operator or~\defn{sequential composition}~as in the
infix expression $(e_1, e_2)$, evaluates $e_1$ and
$e_2$ sequentially, one after the other, and returns the ordered pair
consisting of the two resulting values.
%
Parenthesis delimit  tuples.
%
%

The~\defn{parallel} operator or~\defn{parallel composition}~``$||$'',
as in the infix expression
%
$(e_1~||~e_2)$,
%
evaluates $e_1$ and $e_2$ in parallel, at the same time, and
returns the ordered pair consisting of the two resulting values.
%


The two operators are identical in terms of their return values.
%
However, we will see later, their cost semantics differ: one is
sequential and the other parallel.  The comma and parallel operators
have the weakest, and equal, precedence.
%
\end{gram}

\begin{teachnote}
Cost model pointer.
\end{teachnote}

\begin{example}

\begin{itemize}

\item

The expression 
\[
\begin{array}{l}
\cd{lambda}~(x, y).~(x * x, y * y)
\end{array}
\]
is a function that take two arguments $x$ and $y$ and returns a
pair consisting of the squares $x$ and $y$.
%

\item
The expression
\[
\begin{array}{l}
\cd{lambda}~(x, y).~(x * x~||~y * y)
\end{array}
\]
is a function that take two arguments $\cd{x}$ and $\cd{y}$ and returns a
pair consisting of the squares $\cd{x}$ and $\cd{y}$ by squaring each of
$\cd{x}$ and $\cd{y}$ in parallel.
\end{itemize}

\end{example}

\begin{gram}[Case Expressions]
A~\defn{case expression}~such as 
%
\[
\begin{array}{l}
\cd{case}~e_1 \\
\cd{| Nil}\dra e_2 \\ 
\cd{| Cons}~(x,y)\dra e_3 \\
\end{array}
\]
%
first evaluates the expression $e_1$ to a value $v_1$, which must
return data type.
%
It then matches $v_1$ to one of the patterns, $\cdvar{Nil}$ or
$\cdvar{Cons}~(x,y)$ in our example, binds the variable if any in the
pattern to the respective sub-values of $v_1$, and evaluates the
``right hand side'' of the matched pattern, i.e., the expression $e_2$
or $e_3$.
\end{gram}

\begin{gram}[Conditionals]
A conditional or an~\defn{if-then-else expression},
$\cd{if}~e_1~\cd{then}~e_2~\cd{else}~e_3$, evaluates the expression $e_1$,
which must return a Boolean.
%
If the value of $e_1$ is true then the result of the if-then-else
expression is the result of evaluating $e_2$, otherwise it is the
result of evaluating $e_3$.  
%
This allows for conditional evaluation of expressions.
\end{gram}

\begin{gram}[Function Application]
A~\defn{function application}, $e_1~e_2$, applies the function
generated by evaluating~$e_1$ to the value generated by
evaluating~$e_2$.  
%
For example, lets say that~$e_1$ evaluates to the function~$f$
and~$e_2$ evaluates to the value~$v$, then we apply~$f$ to~$v$ by
first matching~$v$ to the argument of $f$, which is pattern, to
determine the values of each variable in the pattern.
%
We then substitute in the body of~$f$ the value of each variable for
the variable.  To~\defn{substitute}~a value in place of a variable~$x$
in an expression~$e$, we replace each instance of~$x$ with~$v$.
%

For example if function $\cd{lambda}~(x,y).~e$ is applied to the pair
$\cd{(2,3)}$ then $x$ is given value $\cd{2}$ and $y$ is given value
$\cd{3}$. 
%
Any free occurrences of the variables $x$ and $y$ in the
expression $e$ will now be bound to the values $\cd{2}$ and $\cd{3}$
respectively.  
%
We can think of function application as substituting
the argument (or its parts) into the free occurrences of the variables
in its body $e$.
%
The treatment of function application is why we call \pml{} a~\defn{strict}~language. 
%
In strict or call-by-value languages, the argument to the function is
always evaluated to a value before applying the function.
%
In contrast non-strict languages wait to see if the argument will be
used before evaluating it to a value. 
% 
\end{gram}

\begin{example}
\begin{itemize}

\item
The expression
\[
(\cd{lambda}~(x,y).~x / y)~(8,2)
\]
\\
evaluates to $4$ since $8$ and 
$2$ are bound to $x$ and $y$, respectively, and then divided.

\item
The expression 
\[
(\cd{lambda}~(f,x).~f(x,x))~(\cdvar{plus},3)
\] 
%
evaluates to $6$ because $f$ is bound to the function
$\cdvar{plus}$, $x$ is bound to $3$, and then $\cdvar{plus}$ is applied
to the pair $(3,3)$.

\item
The expression
\[
(\cd{lambda}~x.~(\cd{lambda}~y .~x + y ) )~3
\]
\\
%
evaluates to a function that adds $3$ to any integer.
\end{itemize}
\end{example}

\begin{gram}[Bindings]

The~\defn{let expression}, 
\[
\cd{let}~b^+\cd{in}~e~\cd{end},
\] 
%
consists of a sequence of bindings $b^+$, which define local variables
and types, followed by an expression $e$, in which those bindings are
visible.  In the syntax for the bindings, the superscript $+$ means
that $b$ is repeated one or more times.  Each binding $b$ is either a
variable binding, a function binding, or a type binding.
%
The let expression evaluates to the result of evaluating $e$ given the
variable bindings defined in $b$.

A~\defn{function binding}, $x (p) = e$, consists of a function
name, $x$ (technically a variable), the arguments for the function,
$p$, which are themselves a pattern, and the body of the function,
$e$. 

%

Each~\defn{type binding}~equates a type to a base type or a data type.
\end{gram}

\begin{example}
Consider the following let expression.
\[
\begin{array}{l}
\cd{let}\\ 
~~~~x = 2 + 3\\
~~~~f (w) = (w * 4, w - 2)\\
~~~~(y,z) = f(x-1)\\
\cd{in}\\ 
~~~~x + y + z\\
\cd{end} 
\end{array}
\]

The first  binding the variable $x$ to $\cd{2 + 3 = 5}$;
%
The second binding defines a function $f(w)$ which returns a pair;
%
The third binding applies the function $f$ to $x - 1 = 4$
returning the pair $(4 * 4, 4 -2) = (16, 2)$, which
  $y$ and $z$ are bound to, respectively (i.e., $y = 16$ and
  $z = 2$.
%
Finally the let expressions adds $x, y, z$ and yields $5 + 16 + 2$.
The result of the expression is therefore $23$.
\end{example}

\begin{note}
Be careful about defining which variables each binding can see, as
this is important in being able to define recursive functions.  In
\pml{} the expression on the right of each binding in a $\cd{let}$ can
see all the variables defined in previous variable bindings, and can
see the function name variables of all binding (including itself)
within the $\cd{let}$.
%
Therefore the function binding
%
\[
\cd{x}(p) = e
\]
%
is not equivalent to the variable binding
%
\[
\cd{x} = \cd{lambda}~p.e,
\]
%
because in the prior $x$ can be used
in $e$ and in the later it cannot.
%
Function bindings therefore allow for the definition of
recursive functions.  
%
Indeed they allow for mutually recursive functions since the body of
function bindings within the same $\cd{let}$ can reference each other.
\end{note}

\begin{example}
The expression
%
\[
\begin{array}{l}
\cd{let}\\
~~~~f(i) = \cd{if} ~(i < 2) ~\cd{then}~ i ~\cd{else}~ i  * 
f(i - 1) \\
\cd{in} \\ 
~~~~f(5) \\
\cd{end} 
\end{array}
\]
will evaluate to the factorial of $5$, i.e., $5 * 4 * 3 * 2
* 1$, which is $120$.
\end{example}

\begin{example}
The piece of code below illustrates an example use of data types and
higher-order functions.
%
\[
\begin{array}{l}
\cd{let}
\\ 
~~~~\cd{type}~\cdvar{point} = \cdvar{PointTwo}~\cd{of}~\tyint * \tyint
\\ 
~~~~~~~~~~~~~~~~~~~~~~~~|~~\cdvar{PointThree}~\cd{of}~ \tyint * \tyint * \tyint
\\
~~~~\cdvar{injectThree}~(\cdvar{PointTwo}~(x, y)) = \cdvar{PointThree}~(x, y, 0)
\\  
~~~~\cdvar{projectTwo}~(\cdvar{PointThree}~(x, y, z)) = \cdvar{PointTwo}~(x, y)
\\ 
~~~~\cdvar{compose}~f~g = f~g  
\\
~~~~p0 = \cdvar{PointTwo}~(0,0)
\\
~~~~q0 = \cdvar{injectThree}~p0
\\
~~~~p1 = (\cdvar{compose}~\cdvar{projectTwo}~\cdvar{injectThree})~p0
\\
\cd{in} 
\\
~~~~(p0, q0)
\\
\cd{end}
\end{array}
\]

The example code above defines a $\cdvar{point}$ as a two (consisting of
$x$ and $y$ axes) or three dimensional (consisting of $x$, $y$, and
$z$ axes) point in space.
%
The function $\cdvar{injectThree}$ takes a 2D point and transforms it to a 3D
point by mapping it to a point on the $z=0$ plane. 
%
The function $\cdvar{projectTwo}$ takes a 3D point and transforms it to a 2D
point by dropping its $z$ coordinate.
%
The function $\cdvar{compose}$ takes two functions $f$ and $g$ and composes
them.
%
The function $\cdvar{compose}$ is a higher-order function, since id operates
on functions.

The point $p0$ is the origin in 2D.  The point $q0$ is then computed
as the origin in 3D.  The point $p1$ is computed by injecting $p0$ to
3D and then projecting it back to 2D by dropping the $z$ components,
which yields again $p0$.  
%
In the end we thus have $p0 = p1 = (0,0)$. 

\end{example}


\begin{example}
The following \pml code, which defines a binary tree whose leaves and
internal nodes holds keys of integer type.
%
The function $\cdvar{find}$ performs a lookup in a given binary-search tree
$t$, by recursively comparing the key $x$ to the keys along a path in
the tree.

\[
\begin{array}{l}
\cd{type}~\cdvar{tree} = \cdvar{Leaf}~\cd{of}~\tyint~|~\cdvar{Node}~\cd{of}~(\cdvar{tree}, \tyint, \cdvar{tree})
\\
\cdvar{find}~(t, x) = 
\\
~~~~\cd{case}~t
\\ 
~~~~|~\cdvar{Leaf}~y \dra x = y 
\\
~~~~|~\cdvar{Node}~(\cdvar{left}, y, \cdvar{right}) \dra
\\
~~~~~~~~~\cd{if}~x = y~\cd{then} 
\\
~~~~~~~~~~~~~\cd{return}~\cd{true}
\\
~~~~~~~~~\cd{else}~\cd{if}~x < y~\cd{then} 
\\
~~~~~~~~~~~~~\cdvar{find}~(\cdvar{left}, x)
\\
~~~~~~~~~\cd{else}
\\
~~~~~~~~~~~~~\cdvar{find}~(\cdvar{right}, x)
\end{array}
\]
\end{example}


%% \begin{example}
%% The expression: \[(\cfn{(x,y)}{x / y})~(8,2)\] evaluates to $4$ since $8$ and 
%% $2$ are bound to $x$ and $y$, respectively, and then divided.
%% The expression: \[(\cfn{(f,x)}{f(x,x)})~(\cdvar{plus},3)\] evaluates to $6$
%% since $f$ is bound to the function \cdvar{plus}, $x$ is bound to $3$,
%% and then \cdvar{plus} is applied to the pair $(3,3)$.
%% The expression: \[(\cfn{x}{(\cfn{y}{x + y})})~3\] evaluates to a
%% function that adds $3$ to any integer.
%% \end{example}



\begin{remark}~\\
The definition
\[
\cfn{x}{(\cfn{y}{f(x,y)})}
\]
  takes a function $f$ of a pair of arguments and converts it
  into a function that takes one of the arguments and returns a
  function which takes the second argument.  This technique can be
  generalized to functions with multiple arguments and is often
  referred to as~\defn{currying}, named after Haskell Curry
  (1900-1982), who developed the idea.  It has nothing to do with the
  popular dish from Southern Asia, although that might be an easy way
  to remember the term.
\end{remark}

%% \section{Derived Syntax for Loops}

%% \begin{gram}
%% \pml{} does not have explicit syntax for loops but loops can be
%% implemented with recursion.
%% %
%% We use the syntactic sugar, which is defined below, for expressing
%% while loops.
%% %
%% \end{gram}

%% \begin{syntax}[While loops]
%% \label{syn:sparc::while}
%% The~\defn{while loop}~can appear as one of the bindings $b$ in a
%% \cd{let}~expression and has the syntax the following syntax.
%% %
%% \[
%% \begin{array}{ll}
%% xs = 
%% & \cd{start}~p~\cd{and}
%% \\
%% & \cd{while}~e_{continue}~\cd{do}
%% \\
%% &~~~~~b^+
%% \end{array}
%% \]
%% %
%% Here $xs$ are the result variables holding the values computed by the
%% while loop, the pattern $p$ is the initial value for $xs$.
%% %
%% Such a $\cd{while}$ loop evaluates by setting $xs$ to pattern $p$ and
%% then evaluating the loop until $e_{continue}$ evaluates to $\cfalse$.
%% %
%% In a typical use the body of the loop $b^+$ defines the variables
%% $xs$, whose final value will be the value of $xs$ when the loop terminates.

%% We define the while loop syntax above  to be equivalent to the
%% following pair of bindings.

%% \[
%% \begin{array}{l}
%% f~xs = \cd{if not}~e_{continue}~\cd{then}~xs
%% \\
%% ~~~~~~~~~~~~~~~~~\cd{else let}~b^+~\cd{in}~f~xs~\cd{end}
%% \\
%% xs =  f~p
%% \end{array}
%% \]

%% %% \[
%% %% \begin{align}
%% %% f~xs  = & \cd{if not}~e_{continue}~\cd{then}~xs
%% %% \\
%% %%   & \cd{else let}~b^+~\cd{in}~f~xs~\cd{end}
%% %% \\
%% %% \cd{xs} & = & f~p
%% %% \end{align}
%% %% \]

%% Here $xs$, $p$, $e_{continue}$ and $b^+$ are substituted
%% verbatim.  
%% %
%% The loop is expressed as a function that takes $xs$ as an argument and
%% runs the body of the loop until the expression $e_{continue}$ becomes
%% false at which time the variables $xs$ is returned.
%% %
%% The variables $xs$ are passed from one iteration of the while to the next
%% each of which might redefine them in the bindings.  
%% %
%% After the loop terminates, the variables take on the value they had at
%% the end of the last iteration.

%% When evaluated a while loop starts by matching the variables $xs$ to
%% the pattern $p$ and then continues to evaluate the while loop in the
%% usual fashion.
%% %
%% It first checks the value of $e_{continue}$, if it is false, then the
%% evaluation completes.  
%% %
%% If not, then the  bindings in $b^{+}$, which can use the variables
%% $xs$, are evaluated.
%% %
%% Having finished the body, evaluation jumps to the beginning of the
%% $\cd{while}$ and evaluates the termination condition $e_{continue}$, and
%% continues on executing the loop body and so on. 
%% \end{syntax}

%% \begin{example}
%% The following code sums the squares of the integers from 1 to $n$.
%% %
%% \[
%% \begin{array}{l}
%% \cdvar{sumSquares}~(n) = 
%% \\
%% ~~~~\cd{let}~(s,n) = \cd{start}~(0,n)~\cd{and} \\ 
%% ~~~~~~~~~~~~~~~~~~~~~~~~~~\cd{while}~n > 0~\cd{do} \\
%% ~~~~~~~~~~~~~~~~~~~~~~~~~~~~~~s = s + n * n \\
%% ~~~~~~~~~~~~~n = n - 1\\
%% ~~~~\cd{in}~s~\cd{end}
%% \end{array}
%% \]
%% %
%% By definition it is equivalent
%% to the following code.
%% \[
%% \begin{array}{l}
%% \cdvar{sumSquares}~(n) =\\ 
%% ~~~~\cd{let}~f~(s, n) = \\ 
%% ~~~~~~~~~~~~~~~\cd{if}~\cd{not} (n > 0) ~\cd{then}\\
%% ~~~~~~~~~~~~~~~~~~~(s, n) \\
%% ~~~~~~~~~~~~~~~\cd{else}\\
%% ~~~~~~~~~~~~~~~~~~~\cd{let} \\
%% ~~~~~~~~~~~~~~~~~~~~~~~~s = s + n * n \\
%% ~~~~~~~~~~~~~~~~~~~~~~~~n = n - 1 \\
%% ~~~~~~~~~~~~~~~~~~~\cd{in}~f~(s, n)~\cd{end}\\
%% ~~~~~~~~~~~~~(s, n) = f~(0, n)\\
%% ~~~~\cd{in}~s~\cd{end}
%% \end{array}
%% \]
%% \end{example}

