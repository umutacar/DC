\chapter{Parallelism}
\label{ch:introduction::parallelism}

\begin{gram}
The term ``parallelism'' or ``parallel computing'' refers to the ability to run multiple computations (tasks) at the same time.
%
Today parallelism is available in all computer systems, and at many different scales starting with parallelism in the nano-circuits that implement individual instructions, and working the way up to parallel systems that occupy large data centers.  
\end{gram}

\begin{teachask}
Why should we care about parallelism?  
%
Do we even have parallel computers today? 
\end{teachask}
%

\section{Parallel Hardware}
%

\begin{gram}[Multicore Chips]
Since the early 2000s
hardware manufacturers have been placing multiple processing units,
often called ``cores'', onto a single chip.  These cores can be
general purpose processors, or more special purpose processors, such as
those found in~\defn{Graphics Processing Units} (GPUs).  
%
Each core can run in parallel with the others.
%
Today (in year 2018), multicore chips are used in essentially all
computing devices ranging from mobile phones to desktop computers and
servers.
\end{gram}

\begin{gram}[Large-Scale Parallelism]
At the larger scale, many computers can be connected by a network and
used together to solve large problems.  For example, when you perform
a simple search on the Internet, you engage a data center with
thousands of computers in some part of the world, likely near your
geographic location.
%
Many of these computers (perhaps as many as hundreds, if not
thousands) take up your query and sift through data to give you an
accurate response as quickly as possible.
%
Because each computer can itself be parallel (e.g., built with
multicore chips), the scale of parallelism can be quite large, e.g.,
in the thousands.
\end{gram}

%% \begin{teachask}
%% Do you know how many computers are engaged in answering a typical web
%% search? 
%% \end{teachask}
%% %

%% \begin{teachask}
%% What is the advantage of using a parallel algorithm instead of a
%% sequential one?
%% \end{teachask}


%% \begin{teachask}
%% Do you know how much energy it takes to run a computation twice as
%% fast using a sequential computer (one line of computation)? 
%% \end{teachask}

\begin{gram}[Fundamental Reasons for Why Parallelism Matters]
There are several reasons for why parallelism has become prevalent
over the past decade.
%

First, parallel computing is simply faster than sequential computing.
%
This is important, because many tasks must be completed quickly to be
of use. 
%
For example, to be useful, an Internet search should complete in
``interactive speeds'' (usually below $100$ milliseconds).
%
Similarly, a weather-forecast simulation is essentially useless if it
cannot be completed in time.

The second reason is efficiency in terms of energy usage.
%
Due to basic physics, performing a computation twice as fast
sequentially requires eight times as much energy (energy consumption is a cubic function of clock frequency). 
%
With parallelism we don't need to use more energy than sequential
computation, because energy is determined by the total amount of
computation (work).
%

These two factors---time and energy---have become increasingly
important in the last decade.
\end{gram}

\begin{example}
Using two parallel computers, we can perform a computation in half the
time of a sequential computer (operating at the same speed).
%
To this end, we need to divide the computation into two parallel
sub-computations, perform them in parallel and combine their results.
%
This can require as little as half the time as the sequential
computation.
%
Because the total computation that we must do remains the same in both
sequential and parallel cases, the total energy consumed is also the same.
%

The above reasoning holds in theory.
%
In practice, there are overheads to parallelism: the speedup will be
less than two-fold and more energy will be needed.
%
For example, dividing the computation and combining the results could
lead to additional overhead.  
%
Such overhead usually diminishes as the degree of parallelism
increases but not always.

\end{example}

\begin{teachask}
Can you think of consequences of these developments in hardware?  

Answer:
These developments in hardware make the specification, the design, and
the implementation of parallel algorithms an important topic.

\end{teachask}


\begin{example}
As is historically popular in explaining algorithms, we can establish
an analogy between parallel algorithms and cooking.  As in a kitchen
with multiple cooks, in parallel algorithms you can do things in
parallel for faster turnaround time.  For example, if you want to
prepare 3 dishes with a team of cooks you can do so by asking each
cook to prepare one.
%
Doing so will often be faster that using one cook.  But there are some
overheads, for example, the work has to be divided as evenly as
possible.  Obviously, you also need more resources, e.g., each cook
might need their own cooking pan.
\end{example}

\begin{example}[Comparison to Sequential]
\label{ex:intro::intro::example-runs}
One way to quantify the advantages of parallelism is to compare
its performance to sequential computation.
%
The table below illustrates the sort of performance improvements
that can achieved today.  
%
%
These timings are taken on a 32 core commodity server machine.  In the
table, the sequential timings use sequential algorithms while the
parallel timings use parallel algorithms.  Notice that
the~\defn{speedup} for the parallel 32 core version relative to the
sequential algorithm ranges from approximately 12 (minimum spanning
tree) to approximately 32 (sorting).

  \begin{tabular}{l  c c c}
    \toprule
    \textbf{Application} & \textbf{Sequential} & \textbf{Parallel} &
    \textbf{Parallel}
\\
     & & \textbf{P = 1} & \textbf{P = 32}
\\
    \midrule
    Sort $10^7$ strings &        2.9 &  2.9 &  .095\\
    Remove duplicates for $10^7$ strings &      .66 &  1.0 & .038\\
    Minimum spanning tree for $10^7$ edges    &    1.6 & 2.5  & .14\\
    Breadth first search for $10^7$ edges  &   .82  & 1.2 &  .046\\
    \bottomrule
  \end{tabular}
\end{example}


\section{Parallel Software}

\begin{teachask}
But why after all, do we have to do anything differently to take
advantage of parallelism?  
\end{teachask}

\begin{gram}[Challenges of Parallel Software]
\label{gr:intro::intro::parallelism::challenges}
It would be convenient to use sequential algorithms on parallel
computers, but this does not work well because parallel computing
requires a different way of organizing the computation.
%
The fundamental difference is that in parallel algorithms,
computations must actually be~\defn{independent} to be performed in
parallel.
%
By independent we mean that computations do not depend on each other.
%
Thus when designing a parallel algorithm, we have to identify the
underlying dependencies in the computation and avoid creating
unnecessary dependencies.
%
This design challenge is an important focus of this book.
\end{gram}

%% \begin{example}
%% Suppose that you want to run many searches on a database of student
%% records.  To improve your search time, you decide to sort the records
%% by the student name so that you can use a fast binary search algorithm
%% to find each student.  Since the binary search has to wait for the
%% sort to complete, you cannot sort and search in parallel.  You can
%% however perform all the searches in parallel.   You can also sort in
%% parallel by using a parallel sorting algorithm, as we will describe
%% in this book.
%% \end{example}

\begin{example}
Going back to our cooking example, suppose that we want to make a
frittata in our kitchen with 4 cooks.
%
Making a frittata is not easy.
%
It involves cleaning and chopping vegetables, beating eggs,
sauteeing, as well as baking.
%
For the frittata to be good, the cooks must follow a specific receipe
and pay attention to the dependencies between various tasks.
%
For example,
%
vegetables cannot be sauteed before they are washed, and 
%
the eggs cannot be fisked before they are broken!
%
\end{example}

\begin{gram}[Coding Parallel Algorithms]
\label{gr:intro::intro::parallelism::challenges-software}
Another important challenge concerns the implementation and use of
parallel algorithms in the real world.
%
The many forms of parallelism, ranging from small to large scale, and
from general to special purpose, have led to many different programming
languages and systems for coding parallel algorithms.
%
These different programming languages and systems often target a
particular kind of hardware, and even a particular kind of problem
domain.  
%
As it turns out, one can easily spend weeks or even months optimizing a
parallel sorting algorithm on specific parallel hardware, such as a
multicore chip, a GPU, or a large-scale massively parallel distributed
system.
\end{gram}

%% The diversity of parallel hardware and software makes it difficult to
%% learn both the high-level ideas of developing parallel algorithms and
%% the optimization techniques required to achieve efficiency on a variety
%% of different machines.
%% %
%% For example, would a particular parallel sorting algorithm be
%% theoretically efficient on a large-scale system? How about on a small
%% scale system?  What would be needed to implement an optimized parallel
%% sorting algorithm for a GPU?

\begin{gram}
Maximizing speedup by coding and optimizing an algorithm is not the
goal of this book.
%
Instead, our goal is to cover general design principles for parallel
algorithms that can be applied in essentially all parallel systems,
from the data center to the multicore chips on mobile phones.
%
We will learn to think about parallelism at a high-level, learning
general techniques for designing parallel algorithms and data
structures, and learning how to approximately analyze their costs.
%
The focus is on understanding when things can run in parallel, and
when not due to dependencies.  
%
There is much more to learn about parallelism, and we hope you
continue studying this subject.
\end{gram}

\begin{example}
There are separate systems for coding parallel numerical
algorithms on shared memory hardware, for coding graphics algorithms
on Graphical Processing Units (GPUs), and for coding data-analytics
software on a distributed system.
%
Each such system tends to have its own programming interface, its own
cost model, and its own optimizations, making it practically
impossible to take a parallel algorithm and code it once and for all 
possible applications.
%
Indeed, it can require a significant effort to implement even a simple
algorithm and optimize it to run well on a particular parallel system. 
\end{example}

%% For example, it is unlikely that unoptimized code can obtain the
%% speedups discussed in \exref{intro::intro::example-runs}.
%% %


%% \begin{checkpoint}

%% \begin{questionma}
%% \points 10
%% \prompt Which of the following hold for the current state of the art
%% in computing.

%% \select There are not many parallel computers today.

%% \select* There are different sorts of parallel hardware readily
%% available to everyday consumers today.
%% \explain Discussed in the unit above.

%% \select Parallel hardware is available today but expensive. 

%% \select* There are different forms of parallelism ranging from small
%% scale (involing just a few processors) to large scale (involving
%% thousands even millions of processors).

%% \explain Discussed in the unit above.
%% \end{questionma}

%% \begin{questionmc}
%% \points 10

%% \prompt What can be said about parallel software today?

%% \choice Nearly all software is sequential.
%% \explain Much software today is sequential but many are not.

%% \choice Parallel software is everywhere and easy to develop.
%% \explain Parallel software is common but not easy to develop.

%% \choice* Parallel software is quite widespread but it is not easy to
%% develop.  
%% \explain Parallel software is indeed quite widespread and it
%% is not easy to develop.

%% \choice Parallel software is difficult to develop and is typically
%% used very specific domains areas such as scientific computing.

%% \end{questionmc}

%% \begin{questionfr}[Advantages of Parallelism]
%% \points 10
%% \prompt
%% What are the two advantages of parallelism over sequential
%% computation.  

%% \begin{answer}
%% \begin{enumerate}
%% \item Speed: a parallel algorithm using $P$ processors can be as much
%%   as $P$ times faster than sequential.
%% \item Energy efficiency: when computing sequentially, speeding a
%%   computation by a factor of two requires eight times as much energy.
%%   Using parallelism, it is theoretically the same, though in practice
%%   today. there are overheads.
%% \end{enumerate}

%% \end{answer}
%% \end{questionfr}


%% \begin{questionfr}[Essence of Parallelism]
%% \points 10
%% \prompt
%% When designing a parallel algorithm, we have to be careful in
%% identifying the computations that can be performed in parallel.  What
%% is the key property of such computations? 

%% \begin{answer}
%% They have to be independent, that is one cannot use data that is
%% generated by the other.
%% \end{answer}
%% \end{questionfr}

%% \end{checkpoint}


\section{Work, Span, Parallel Time}
\label{sec:introduction::parallelism::work-span}

\begin{gram}
This section describes the two measures---work and span---that we use
to analyze algorithms.  Together these measures capture both the
sequential time and the parallelism available in an algorithm.
%
We typically analyze both of these asymptotically, using
for example the big-O notation.
%
\end{gram}

\subsection{Work and Span}

\begin{gram}[Work]
The~\defn{work} of an algorithm corresponds to the total number of
primitive operations performed by an algorithm.  If running on a
sequential machine, it corresponds to the sequential time.
%
On a parallel machine, however, work can be divided among multiple
processors and thus does not necessarily correspond to time.
%

The interesting question is to what extent can the work be divided and
performed in parallel.  Ideally we would like to divide the work
evenly.  If we had $W$ work and $P$ processors to work on it in
parallel, then even division would give each processor $\frac{W}{P}$
fraction of the work, and hence the total time would be $\frac{W}{P}$.
%
An algorithm that achieves such ideal division is said to
have~\defn{perfect speedup}.  Perfect speedup, however, is not always
possible.
%
\end{gram}

\begin{example}
A fully sequential algorithm, where each operation depends on prior
operations leaves no room for parallelism.
%
We can only take advantage of one processor and the time would not be
improved at all by adding more.  
%

More generally, when executing an algorithm in parallel, we cannot
break dependencies, if a task depends on another task, we have to
complete them in order.
\end{example}


\begin{teachask}
Can you come up with an example where perfect speedup is not possible?
\end{teachask}
%

\begin{teachnote}
For example, when cooking a frittata, you cannot cook the egg that is
not broken, nor can we add the eggs to the pan until the vegetables
are sauteed.
\end{teachnote}

\begin{gram}[Span]
The second measure,~\defn{span}, enables analyzing to what extent the
work of an algorithm can be divided among processors.
The~\defn{span}~of an algorithm basically corresponds to the longest
sequence of dependences in the computation.  It can be thought of as
the time an algorithm would take if we had an unlimited number of
processors on an ideal machine.
\end{gram}
%


\begin{teachask}
How do we calculate the work and span of an algorithm?
\end{teachask}

\begin{flex}
\begin{definition}[Work and Span]
We calculate the work and span of algorithms in a very
simple way that just involves composing costs across subcomputations.
%
Basically we assume that sub-computations are either composed
sequentially (one must be performed after the other) or in parallel
(they can be performed at the same time).
%
We then calculate the work as the sum of the work of the
subcomputations.
%
For span, we differentiate between sequential and parallel composition:
%
we calculate span as the sum of the span of sequential
subcomputations or maximum of the span of the parallel
subcomputations.
%
More concretely, given two subcomputations with work $W_1$ and $W_2$
and span $S_1$ and $S_2$, we can calculate the work and the span of
their sequential and parallel composition as follows.
%
In calculating the overall work and span, the unit cost $1$ accounts
for the cost of (parallel or sequential) composition.


\begin{center}
\renewcommand{\arraystretch}{1.5}
\begin{tabular}{lcc}
\toprule
                          &  \bf $W$ (Work) & \bf $S$ (span)\\
\midrule
\bf Sequential composition & $1 + W_1 + W_2$ & $1 + S_1+ S_2$\\
\midrule
\bf Parallel composition   & $1 + W_1 + W_2$ & $1 + \max(S_1, S_2)$\\
\bottomrule
\end{tabular}
\end{center}
\end{definition}

\begin{note}
The intuition behind the definition of work and span is that work
simply adds, whether we perform computations sequentially or in
parallel.  The span, however, only depends on the span of the maximum
of the two parallel computations.  It might help to think of work as
the total energy consumed by a computation and span as the minimum
possible time that the computation requires.  Regardless of whether
computations are performed serially or in parallel, energy is equally
required; time, however, is determined only by the slowest
computation.
\end{note}


\begin{example}
Suppose that we have $30$ eggs to cook using $3$ cooks.  Whether all
$3$ cooks to do the cooking or just one, the total work remains
unchanged: $30$ eggs need to be cooked.
%
Assuming that cooking an egg takes $5$ minutes, the total work
therefore is $150$ minutes.
%
The span of this job corresponds to the longest sequence of
dependences that we must follow.
%
Since we can, in principle, cook all the eggs at the same time, 
span is 5 minutes.
%

Given that we have $3$ cooks, how much time do we actually need?
%
It should be clear that each cook can cook 10 eggs, for a total time of $50$
minutes.   Later we will discuss the ``the greedy scheduling
principle'' which tells us that given a task with $W$ work and $S$
span, and using a greedy schedule, the time is upper bounded by $W/P+
S$.    In our case this would be $150/3 + 5 = 55$.
\end{example}
\end{flex}


\begin{example}[Parallel Merge Sort]
\label{ex:intro::intro::mergesort-cost}
As an example, consider the parallel $\cd{mergeSort}$ algorithm for
sorting a sequence of length $n$.  The work is the same as the
sequential time, which you might know is
\[
W(n) = O(n \lg{n}).
\] 
%
%In Chapter~\ref{ch:divide-and-conquer} 
We will see that the span for
$\cd{mergeSort}$ is
\[
S(n) = O(\lg^2{n}).
\]

Thus, when  sorting a million keys and ignoring constant factors, 
work is $10^6\lg (10^6) > 10^7$, and 
%
span is 
$\lg^2(10^6) < 500.$
%
\end{example}

\begin{gram}[Parallel Time]
%As we shall see in \secref{design::scheduling}, 
Even though work and span, are abstract measures of real costs, they
can be used to predict the run-time on any number of processors.
%
Specifically, if for an algorithm the work dominates, i.e., is much
larger than, span, then we expect the algorithm to deliver good
speedups.
\end{gram}
%

\begin{flex}
\begin{exercise}
How would you expect the parallel mergesort algorithm, $\cd{mergeSort}$,
mentioned in the example above to perform as we increase the number of
processors dedicated to running it?
\end{exercise}

\begin{solution}
Recall that the work of parallel merge sort is $O(n\lg{n})$, whereas
the span is $O(\lg^2{n})$.  
%
Since span is much smaller than the work, we would expect to get good
(close to perfect) speedups when using a small to moderate number of
processors, e.g., couple of tens or hundreds, because the work term
will dominate.
%
We would expect for example the running time to halve when we double
the number of processors.
%
We should note that in practice, speedups tend to be more conservative
due to natural overheads of parallel execution and due to other
factors such as the memory subsystem that can limit parallelism. 
\end{solution}
\end{flex}

\subsection{Work Efficiency}

\begin{gram}
If algorithm $A$ has less work than algorithm $B$, but has greater
span then which algorithm is better?  In analyzing sequential
algorithms there is only one measure so it is clear when one algorithm
is asymptotically better than another, but now we have two measures.
In general the work is more important than the span.  
%
This is because the work reflects the total cost of the computation
(the processor-time product).  Therefore typically the goal is to
first reduce the work and then reduce the span by designing
asymptotically work-efficient algorithms that perform no work
than the best sequential algorithm for the same problem. 
%
However, sometimes it is worth giving up a little in work to gain a
large improvement in span.
%
\end{gram}

\begin{flex}
\begin{definition}[Work Efficiency]
\label{def:intro::intro::work-efficiency}
We say that a parallel algorithm is~\defn{(asymptotically) work
  efficient}, if the work is asymptotically the same as the time for
an optimal sequential algorithm that solves the same problem.
\end{definition} 

\begin{example}
\href{ex:intro::intro::mergesort-cost}{The parallel $\cd{mergeSort}$ function} described in is work efficient since it does $O(n \log n)$ work, which optimal time for comparison based sorting. 

In this course we will try to develop work-efficient or close to work-efficient algorithms.
\end{example}


\end{flex}



