\chapter{Introduction}
\label{ch:design::intro}

\begin{cluster}
\label{grp:grm:design::intro::designing}

\begin{gram}
\label{grm:design::intro::designing}
  Designing algorithms requires a toolbox of design techniques.   In
  this chapter we will describe some of these techniques.  
  \chref{design::basics}  covers the two most basic techniques,
  which are algorithmic reductions and brute force.   In the first we
  simply convert one problem to another for which we already
  have a solution.   For the second, we find a solution, or the best
  solution by trying ``all possibilities''.    \chref{design::dc}
  covers Divide-and-Conquer.     This is an approach you have most
  likely seen before, but we go into some more depth.
  \chref{design::contraction} covers a technique called
  ``contraction''.   The basic idea is to contract a problem into a
  single smaller instance of itself, recurse on the smaller instance,
  and then use it to help solve the larger instance.
  \chref{mcss} covers an example problem, the Maximum Contiguous
  Subsequence Sum (MCSS) problem, and shows how it can be solved by
  using many different techniques.
  
  In later parts we will cover other techniques such as randomization,
  and dynamic programming.

\end{gram}
\end{cluster}

