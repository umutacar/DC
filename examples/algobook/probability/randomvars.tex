\chapter{Random Variables}
\label{ch:probability::randvar}

\begin{preamble}
This chapter introduces the random variables and their use in probability theory.
\end{preamble}


\begin{flex}
\begin{definition}[Random Variable]
\label{def:probability::rv}
A \defn{random variable} $X$ is a real-valued function on the outcomes
of an experiment, i.e., $X : \SSpace \to \R$, i.e., it assigns a real
number to each outcome.
%
For a given probability space there can be many random variables, each keeping
track of different quantities.
%
We typically denote random variables by capital letters from
the end of the alphabet, e.g. $X$, $Y$, and $Z$.
%
We say that a random variable is \defn{discrete} if its range is
finite or countable infinite.  Throughout this book, we only consider
discrete random variables. 
\end{definition}

\begin{example}
For throwing two dice, we can define random variable as the sum of
the two dice
\[
X(d_1,d_2) = d_1+d_2~,
\]
the product of two dice
\[
Y(d_1,d_2) = d_1 \times d_2~,
\]
or the value of the first dice
the two dice:
\[
Z(d_1,d_2) = d_1~.
\]
\end{example}
\end{flex}

\begin{flex}
\begin{definition}[Indicator Random Variable]
A random variable is called an \defn{indicator random variable} if it
takes on the value~$1$ when some condition is true and~$0$
otherwise.
\end{definition}
\begin{example}
  For throwing two dice, we can define indicator random variable
  as getting doubles
\[
Y(d_1,d_2) =
\left\{
\begin{array}{ll}
1 & \mbox{if}~d_1 = d_2
\\
0 & \mbox{if}~d_1 \not= d_2 ~.
\end{array}
\right.
\]
Using our shorthand, the event $\event{X}{4}$ corresponds
to the event ``the dice sum to 4''.
\end{example}
\end{flex}

\begin{flex}
\begin{gram}[Notation]
For a random variable $X$ and a value $x \in \R$, we use the following
shorthand for the event corresponding to $X$ equaling $x$:
\[\event{X}{x} \equiv \csetf{y \in \SSpace}{X(y)  = x}~,\]
and when applying the probability measure we use the further
shorthand
\[\prob{X = x} \equiv \prob{\event{X}{x}}~.\]
\end{gram}

\begin{example}
  For throwing two dice, and $X$ being a random variable representing
  the sum of the two dice, $\event{X}{4}$ corresponds to the
  event ``the dice sum to 4'', i.e. the set
  \[\{y \in \SSpace~|~X(y) = 4\} = \{(1,3),(2,2),(3,1)\}~.\]

  Assuming unbiased coins, we have that
  \[\prob{X = 4} = 1/12~.\]
\end{example}
\end{flex}

\begin{remark}
The term random variable might seem counter-intuitive
because it is actually a function not a variable.
%
As such, a random variable is not really random, because it is a well
defined deterministic function on the sample space.  
%
However, when thought in conjunction with the random experiment that
selects the events, a random variable takes on its value based on a
random process.
\end{remark}

\section{Probability Mass Function}
\label{sec:probability::randvar::pmf}

\begin{flex}
\begin{definition}[Probability Mass Function]
For a discrete random variable $X$, we define its \defn{probability
  mass function} or \defn{PMF}, written $\pmf{X}(\cdot)$, for short as
a function mapping each element $x$ in the range of the random
variable to the probability of the event $\event{X}{x}$, i.e.,
\[
\pmf{X}(x) = \prob{X = x}.
\]
\end{definition}

\begin{example}
The probability mass function for the indicator random variable $X$
indicating whether the outcome of a roll of dice is comes up even is 
\[
\begin{array}{lll}
\pmf{X}(0) = \prob{\event{X}{0}} = \prob{\{1, 3, 5\}} =
1/2,~\mbox{and}
\\
\pmf{X}(1) = \prob{\event{X}{1}} = \prob{\{2, 4, 6\}} =
1/2.
\end{array}
\]

The probability mass function for the random variable $X$
that maps each outcome in a roll of dice to the smallest Mersenne
prime number no less than the outcome is
\[
\begin{array}{lll}
\pmf{X}(3) = \prob{\event{X}{3}} = \prob{\{1, 2, 3\}} =
1/2,~\mbox{and}
\\
\pmf{X}(7) = \prob{\event{X}{7}} = \prob{\{4, 5, 6\}} =
1/2.
\end{array}
\]
\end{example}
\end{flex}

\begin{gram}
Note that much like a probability measure, a probability mass function is
a non-negative function.
%
It is also additive in a similar sense: for any distinct $x$ and $x'$, the
events $\event{X}{x}$ and $\event{X}{x'}$ are disjoint.
%
Thus for any set $\bar{x}$ of values of $X$, we have 
\[
\prob{X \in \bar{x}} = \sum_{x \in \bar{x}}{\pmf{X}(x)}.
\]
%
Furthermore, since $X$ is a function on the sample space, the events
corresponding to the different values of $X$ partition the sample
space, and we have
\[
\sum_{x}{\pmf{X}(x)} = 1.
\]
%
These are the important properties of probability mass functions: they
are non-negative, normalizing, and are additive in a certain sense. 
\end{gram}

\begin{gram}
We can also compute the probability mass function for multiple random
variables defined for the same probability space.
%
For example, the \defn{joint probability mass function} for two random
variables $X$ and $Y$, written $\pmf{X,Y}(x,y)$ denotes the
probability of the event $\event{X}{x} \cap \event{Y}{y}$, i.e.,
\[
\pmf{X,Y}(x,y) = \prob{\event{X}{x} \cap \event{Y}{y}} = \prob{X = x, Y = y}.
\]
%
Here
 $\prob{X = x, Y = y}$ is shorthand for   $\prob{\event{X}{x} \cap \event{Y}{y}}$.

%% Given the joint probability mass function for a pair of random variables
%% $X$ and $Y$, we can calculate the probability mass function for any of
%% them, which is also called the \defn{marginal probability mass
%%   function} as follows
%% \[
%% \pmf{X}(x) = \sum_{y}{\pmf{X,Y}(x,y)},~\mbox{and}
%% \\
%% \pmf{Y}(y) = \sum_{x}{\pmf{X,Y}(x,y)}.
%% \]

In our analysis or randomized algorithms, we shall repeatedly
encounter a number of well-known random variables and create new ones
from existing ones by composition.
\end{gram}


\section{Bernoulli, Binomial, and Geometric RVs}
\label{sec:probability::randvar::standards}

\begin{gram}[Bernoulli Random Variable]
Suppose that we toss a  coin that comes up a head with probability $p$
and a tail with probability $1-p$.
%
The \defn{Bernoulli random variable} takes the value $1$ if the coin
comes up heads and $0$ if it comes up tails.
%
In other words, it is an indicator random variable indicating heads.
%
Its probability mass function is 
\[
\pmf{X}(x) = 
\left\{
\begin{array}{ll}
p & \mbox{if}~x = 1
\\
1-p & \mbox{if}~x = 0.
\end{array}
\right.
\]
\end{gram}

\begin{gram}[Binomial Random Variable]
Consider $n$ Bernoulli trials with probability $p$.  We call the
random variable $X$ denoting the number of heads in the $n$ trials as
the \defn{Binomial random variable}.
%
Its probability mass function for any $0 \le x \le n$ is 
%
\[
\pmf{X}(x) = {n \choose x}\,p^x\,(1-p)^{n-x}.
\]
\end{gram}

\begin{gram}[Geometric Random Variable]
Consider performing Bernoulli trials with probability $p$ until the
coin comes up heads and  $X$ denote the number of trials needed to
observe the first head. 
%
The random variable $X$ is called the \defn{geometric random variable}.
%
Its probability mass function for any $0 \le x$ is 
%
\[
\pmf{X}(x) = (1-p)^{x-1} p.
\]
\end{gram}

\section{Functions of Random Variables}
\label{sec:probability::randvar::functionsof}

\begin{flex}
  \begin{gram}
    It is often useful to ``apply'' a function to one or more random
    variables to generate a new random variable.   Specifically if we
    have a
    function $f : \mathbb{R} \rightarrow \mathbb{R}$ and
    a random variable $X$ we can compose the two giving a new
    random variable:
    \[Y(x) = f(X(x)) \]
    We often write this shorthand as $Y = f(X)$.
    Similarly for two random variables $X$ and $Y$
    we write $Z = X + Y$ as shorthand for 
        \[Z(x) = X(x) + Y(x) \]
    or equivalently
        \[Z = \lambda x . (X(x) + Y(x)) \]

%
The probability mass function for the new variable can be computed by
``massing'' the probabilities for each value.
%
For example, for a function of a random variable $Y = f(X)$, we can
write the probability mass function as 
\[
\pmf{Y}(y) = \prob{Y = y} = \sum_{x \sucht f(x) = y}{\pmf{X}(x)}~.
\]
%
%% Similarly, for a function of two random variables $Z = g(X,Y)$ defined
%% on the same probability space, we can write the probability mass
%% function as
%% \[
%% \pmf{Z}(z) = \sum_{(x,y) \sucht g(x,y) = z}{\pmf{X,Y}(x,y)}.
%% \]
\end{gram}

\begin{example}
Let $X$ be a Bernoulli random variable with parameter $p$.  We can
define a new random variable $Y$ as a transformation of $X$ by a
function $f(\cdot)$.  For example, $Y = f(X) = 9X + 3$ is random
variable that transforms $X$, e.g., $X = 1$ would be transformed to $Y
= 12$.
%
The probability mass function for $Y$ reflects that of $X$,
%
Its probability mass function is 
\[
\pmf{Y}(y) = 
\left\{
\begin{array}{ll}
p & \mbox{if}~y = 12
\\
1-p & \mbox{if}~y = 3.
\end{array}
\right.
\]

\end{example}

\begin{example}
Consider the random variable $X$ with the probability mass function 
\[
\pmf{X}(x) = 
\left\{
\begin{array}{lll}
0.25  & \mbox{if} & x = -2
\\
0.25  & \mbox{if} & x = -1
\\
0.25  & \mbox{if} & x = 0
\\
0.25  & \mbox{if} & x = 1
\end{array}
\right.
\]

We can calculate the probability mass function for the random variable
$Y = X^2$ as follows $\pmf{Y}(y) = \sum_{x \sucht x^2 =
  y}\pmf{X}(x)$.
%
This yields
\[
\pmf{Y}(y) = 
\left\{
\begin{array}{lll}
0.25  & \mbox{if} & y = 0
\\
0.5  & \mbox{if} & y = 1
\\
0.25  & \mbox{if} & y = 4.
\end{array}
\right.
\]

\end{example}
\end{flex}

\section{Conditioning}
\label{sec:probability::randvar::conditioning}

\begin{flex}
\begin{gram}
In the same way that we can condition an event on another, we can also
condition a random variable on an event or on another random variable.
%
Consider a random variable $X$ and an event $A$ in the same probability
space, we define the \defn{conditional probability mass function} of
$X$ conditioned on  $A$ as 
\[
\pmf{X \given A} = \prob{X = x \given A} = \frac{\prob{\event{X}{x} \cap A}}{\prob{A}}.
\]
%
Since for different values of $x$, $\event{X}{x} \cap A$'s are
disjoint and since $X$ is a function over the sample space,
conditional probability mass functions are normalizing just like
ordinary probability mass functions, i.e., $\sum_{x \in X}{\pmf{X \given A}(x)} = 1$.
%
Thus just as we can treat conditional probabilities as ordinary
probabilities, we can treat conditional probability  mass functions
also as ordinary probability mass functions.
\end{gram}

%% \begin{notesonly}
%% Note that the events $\event{X}{x} \cap A$ are disjoint for different
%% values of $x$ and thus their union is $P(A)$.  We thus have $\pmf{X
%%   \given A}(x) = 1$.
%% \end{notesonly}

\begin{example}
Roll a pair of dice and let $X$ be the sum of the face values.  Let
$A$ be the event that the second roll came up $6$.
%
We can find the conditional probability mass function 
\[
\begin{array}{lll}
\pmf{X \given A}(x) & = & \frac{\prob{\event{X}{x} \cap A}}{\prob{A}}
\\
& = & 
\left\{
\begin{array}{ll}
\frac{1/36}{1/6} = 1/6 & \mbox{if}~x = 7,  \ldots, 12.
\\
0 & \mbox{otherwise}
\end{array}
\right.
\end{array}
\]
 
\end{example}
\end{flex}

\begin{gram}[Conditional Probability Mass Function]
Since random variables closely correspond with events, we can
condition a random variable on another.
%
More precisely, let $X$ and $Y$ be two random variables defined on the
same probability space.
%
We define the \defn{conditional probability mass function} of $X$ with
respect to $Y$ as 
\[
\begin{array}{l}
\pmf{X \given Y}(x \given y) = \prob{X = x \given Y = y}.
\end{array}
\] 

We can rewrite this as 
\[
\begin{array}{lll}
\pmf{X \given{} Y}(x \given{} y) 
& = & \prob{X = x \given{} Y = y}
\\[2mm]
& = & \frac{\prob{X = x, Y = y}}{\prob{Y = y}}
\\[2mm]
& = & \frac{\pmf{X,Y}(x,y)}{\pmf{Y}{y}}.
\end{array}
\] 
\end{gram}

\begin{gram}[Conditional PMFs are PMFs]
Consider the function $\pmf{X \given Y}(x \given y)$ for a fixed value
of $y$.  This is a non-negative function of $x$, the event
corresponding to different values of $x$ are disjoint, and they
partition the sample space, the conditional mass functions are
normalizing
\[
\sum_{x}{\pmf{X \given Y}(x \given y)} = 1. 
\]
Conditional probability mass functions thus share the same properties
as probability mass functions.

By direct implication of its definition, we can use conditional
probability mass functions to calculate joint probability mass
functions as follows
\[
\begin{array}{lll}
\pmf{X,Y}(x,y) = \pmf{X}(x) \pmf{Y \given X}(y \given x)
\\
\pmf{X,Y}(x,y) = \pmf{Y}(y) \pmf{X \given Y}(x \given y).
\end{array}
\]
%

As we can compute total probabilities from conditional ones as we saw
earlier in this section, we can calculate marginal probability
mass functions from conditional ones:
\[
\pmf{X}(x) = \sum_{y}{\pmf{X,Y}(x,y)} = \sum_{y}{\pmf{Y}(y)\pmf{X \given Y}(x \given y)}.
\]
\end{gram}



\section{Independence}
\label{sec:probability::randvar::independence}

\begin{gram}
As with the notion of independence between events, we can also define
independence between random variables and events.
%
We say that a random variable $X$ is \defn{independent of an event}
$A$, if 
\[
\mbox{for all}~x: \prob{\event{X}{x} \cap A} = \prob{X = x} \cdot \prob{A}~.
%= \pmf{X}(x) \prob{A}.
\] 
%
When $\prob{A}$ is positive, this is equivalent to 
\[
\pmf{X \given A} (x) = \pmf{X}(x).
\]

Generalizing this to a pair of random variables, we say a random variable
$X$ is \defn{independent of a random variable} $Y$ if 
\[
\mbox{for all}~x, y: \prob{X = x, Y = y} = \prob{X = x} \cdot \prob{Y = y}
\]
%
or equivalently
\[
\mbox{for all}~x, y: \pmf{X,Y}(x,y) = \pmf{X}(x) \cdot \pmf{Y}(y).
\]

In our two dice
example, a random variable $X$ representing the value of the first
dice and a random variable $Y$ representing the value of the second
dice are independent. 
%
However $X$ is not independent of a random variable $Z$ representing
the sum of the values of the two dice.
\end{gram}


