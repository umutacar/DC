\chapter{Data Types and Operators}

\section{Part 1}

\begin{runpython}[title=Some Builtin Types]
import math
def f():
    print("This is a user-defined function")
    return 42

print("Some basic types in Python:")
print(type(2))           # int
print(type(2.2))         # float
print(type(2 < 2.2))     # bool (boolean)
print(type(type(42)))    # type

print("#####################################################")

print("And some other types we may see later in the course...")
print(type("2.2"))       # str (string or text)
print(type([1,2,3]))     # list
print(type((1,2,3)))     # tuple
print(type({1,2}))       # set
print(type({1:42}))      # dict (dictionary or map)
print(type(2+3j))        # complex  (complex number)
\end{runpython}


\begin{runpython}[title=Some Builtin Types]
import math
def f():
    print("This is a user-defined function")
    return 42

print("Some basic types in Python:")
print(type(2))           # int
print(type(2.2))         # float
print(type(2 < 2.2))     # bool (boolean)
print(type(type(42)))    # type

print("#####################################################")

print("And some other types we may see later in the course...")
print(type("2.2"))       # str (string or text)
print(type([1,2,3]))     # list
print(type((1,2,3)))     # tuple
print(type({1,2}))       # set
print(type({1:42}))      # dict (dictionary or map)
print(type(2+3j))        # complex  (complex number)
\end{runpython}

\begin{runpython}[title=Some Builtin Constants]  
print("Some builtin constants:")
print(True)
print(False)
print(None)

print("And some more constants in the math module:")
import math
print(math.pi)
print(math.e)
\end{runpython}

\begin{flex}

\begin{gram}[Some Builtin Operators]
The following table includes some built-in operators in Python.

\begin{tabular}{ll}
Category &	Operators
\\
Arithmetic &	+, -, *, /, //, **, %, - (unary), + (unary)
\\
Relational &	\textless, \textless=, \textgreater=, \textgreater, ==, !=
\\
Assignment &	+=, -=, *=, /=, //=, **=, %=, \textless\textless=, \textgreater\textgreater=
\\
Logical &	and, or, not
\end{tabular}
\end{gram}

\begin{note}
For now at least, we are not covering the bitwise operators (\textless\textless, \textgreater\textgreater, \&, |, \^, \~, \&=, |=, \^=).
\end{note}
\end{flex}

\begin{runpython}[title=Integer Division]    
print("The / operator does 'normal' float division:")
print(" 5/3  =", ( 5/3))
print()
print("The // operator does integer division:")
print(" 5//3 =", ( 5//3))
print(" 2//3 =", ( 2//3))
print("-1//3 =", (-1//3))
print("-4//3 =", (-4//3))
\end{runpython}


\begin{runpython}[title=The Modulus or Remainder Operator (\%)]
print(" 6%3 =", ( 6%3))
print(" 5%3 =", ( 5%3))
print(" 2%3 =", ( 2%3))
print(" 0%3 =", ( 0%3))
print("-4%3 =", (-4%3))
print(" 3%0 =", ( 3%0))
\end{runpython}


\section{Part 2}

\begin{runpython}[title=More of the Modulus or Remainder Operator (\%) verify that (a\%b) is equivalent to (a - (a//b)*b)]

def mod(a, b):
  return a - (a//b)*b

print(41%14, mod(41,14))
print(14%41, mod(14,41))
print(-32%9, mod(-32,9))
print(32%-9, mod(32,-9))
\end{runpython}

\begin{runpython}[title=Types Affect Semantics]
print(3 * 2)
print(3 * "abc")
print(3 + 2)
print("abc" + "def")
print(3 + "def")
\end{runpython}

\begin{runpython}[title=Operator Order (Precedence and Associativity)]
print("Precedence:")
print(2+3*4)  # prints 14, not 20
print(5+4%3)  # prints  6, not 0 (% has same precedence as *, /, and //)
print(2**3*4) # prints 32, not 4096 (** has higher precedence than *, /, //, and %)

print()

print("Associativity:")
print(5-4-3)   # prints -2, not 4 (- associates left-to-right)
print(4**3**2) # prints 262144, not 4096 (** associates right-to-left)
\end{runpython}

\begin{runpython}[title=Approximate Values of Floating-Point Numbers]
print(0.1 + 0.1 == 0.2)        # True, but...
print(0.1 + 0.1 + 0.1 == 0.3)  # False!
print(0.1 + 0.1 + 0.1)         # prints 0.30000000000000004 (uh oh)
print((0.1 + 0.1 + 0.1) - 0.3) # prints 5.55111512313e-17 (tiny, but non-zero!)
\end{runpython}

\begin{runpython}[title=Equality Testing with almostEqual]
print("The problem....")
d1 = 0.1 + 0.1 + 0.1
d2 = 0.3
print(d1 == d2)                # False (never use == with floats!)

print()
print("The solution...")
epsilon = 10**-10
print(abs(d2 - d1) < epsilon)  # True!

print()
print("Once again, using a useful helper function, almostEqual:")

def almostEqual(d1, d2):
    epsilon = 10**-10
    return (abs(d2 - d1) < epsilon)

d1 = 0.1 + 0.1 + 0.1
d2 = 0.3
print(d1 == d2)            # still False, of course
print(almostEqual(d1, d2)) # True, and now packaged in a handy reusable function!
\end{runpython}

\begin{runpython}[title=Short-Circuit Evaluation]
def yes():
    return True

def no():
    return False

def crash():
    return 1/0 # crashes!

print(no() and crash()) # Works!
print(crash() and no()) # Crashes!
print (yes() and crash()) # Never runs (due to crash), but would also crash (without short-circuiting)
\end{runpython}

\begin{runpython}[title={Once again, using the "or" operator}]
def yes():
    return True

def no():
    return False

def crash():
    return 1/0 # crashes!

print(yes() or crash()) # Works!
print(crash() or yes()) # Crashes!
print(no() or crash())  # Never runs (due to crash), but would also crash (without short-circuiting)
\end{runpython}

\begin{runpython}[title=Yet another example]
def isPositive(n):
    result = (n > 0)
    print("isPositive(",n,") =", result)
    return result

def isEven(n):
    result = (n % 2 == 0)
    print("isEven(",n,") =", result)
    return result

print("Test 1: isEven(-4) and isPositive(-4))")
print(isEven(-4) and isPositive(-4)) # Calls both functions
print("----------")
print("Test 2: isEven(-3) and isPositive(-3)")
print(isEven(-3) and isPositive(-3)) # Calls only one function!
\end{runpython}

\begin{runpython}[title=type vs isinstance]
# Both type and isinstance can be used to type-check
# In general, (isinstance(x, T)) will be more robust than (type(x) == T)

print(type("abc") == str)
print(isinstance("abc", str))

# We'll see better reasons for this when we cover OOP + inheritance later
# in the course.  For now, here is one reason:  say you wanted to check
# if a value is any kind of number (int, float, complex, etc). 
# You could do:

def isNumber(x):
    return ((type(x) == int) or
            (type(x) == float)) # are we sure this is ALL kinds of numbers?

print(isNumber(1), isNumber(1.1), isNumber(1+2j), isNumber("wow"))

# But this is cleaner, and works for all kinds of numbers, including
# complex numbers for example:

import numbers
def isNumber(x):
    return isinstance(x, numbers.Number) # works for any kind of number

print(isNumber(1), isNumber(1.1), isNumber(1+2j), isNumber("wow"))
\end{runpython}
