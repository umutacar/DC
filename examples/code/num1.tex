\begin{verbatim}
#!/usr/bin/python3
# Some calculations.  Note the lack of semicolons.  Statements end at the end
# of the line.  Also, variables need not start with a special symbol as in
# perl and some other Unix-bred languages.
fred = 18
barney = FRED = 44;                     # Case sensistive.
bill = (fred + barney * FRED - 10)
alice = 10 + bill / 100                 # Regular division does not truncate.
alice2 = 10 + bill // 100               # But the // operator provides int div.
frank = 10 + float(bill) / 100
print("fred =", fred)
print("FRED =", FRED)
print("bill =", bill)
print("alice =", alice)
print("alice2 =", alice2)
print("frank =", frank )
print()

# Each variable on the left is assigned the corresponding value on the right.
fred, alice, frank = 2*alice, fred - 1, bill + frank
print("fred =", fred)
print("alice =", alice)
print("frank =", frank )
print()

# Exchange w/o a temp.
fred, alice = alice, fred
print("fred =", fred)
print("alice =", alice)
print()

# Python allows lines to be continued by putting a backslash at the end of
# the first part.  
fred = bill + alice + frank - \
       barney
print("fred =", fred)
print()

# The compiler will also combine lines when the line break in contained
# in a grouping pair, such as parens.
joe = 3 * (fred +
        bill - alice)
print ("joe =", fred)
\end{verbatim}
