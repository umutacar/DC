\chapter{Introduction}

\paragraph{Preamble}
This chapter presents a brief overview of the methodology followed in
this book.
\paragraph{}
This is it.
\section{Overview}

\paragraph{}
Here
\subsection{}

\paragraph{Teaching note}
TODO:Updates needed to deemphasize parallelism.
% Guy: I intentionally try not to emphasize parallelism in the opening
% two paragraphs because most of the ideas transend parallel vs sequential.
% The whole next section is on parallelism.

\paragraph{Skip}
TODO:Updates needed to deemphasize parallelism.
% Guy: I intentionally try not to emphasize parallelism in the opening
% two paragraphs because most of the ideas transend parallel vs sequential.
% The whole next section is on parallelism.

\paragraph{}
This book aims to present techniques for problem solving using today's
computers, including both sequentially and in parallel.
%
For example, you might want to find the stortest path from where you
are now to the nearest caf\'e by using your computer.
%
The primary concerns will likely include correctness (the path found
indeed should end at the nearest caf\'e), 
%
efficiency (that your computer consumed a relatively small amount of
energy), and performance (the answer was computed reasonable quickly).
%

This book covers different aspects of problem solving with computers
such as 
\begin{itemize}
\item defining precisely the problem you want to solve,
\item learning the different algorithm-design techniques for solving problems
\item designing abstract data types and the data structures that
  implement them,
%
\item analyzing the cost of algorithms and data structures and
  comparing them based on their cost.
%
\end{itemize}

We are concerned both with 
%
parallel algorithms (algorithms that can perform multiple actions at
the same time)
%
and 
%
sequential algorithms (algorithms that perform a single action at a
time).
%
In our approach, however, sequential and parallel algorithms are not
particularly different.
%
%In the rest of this chapter we discuss why it is important to study
%parallelism, why it is important to separate interfaces from
%implementations, and outline some algorithm-design techniques.



\subsection{Parallelism}

\paragraph{}
The term ``parallelism'' or ``parallel computing'' refers to the
ability to run multiple computations (tasks) at the same time.
%

\paragraph{Teaching question}
Why should we care about parallelism?  
%
Do we even have parallel computers today?

\paragraph{Parallel Hardware}
Today parallelism is available in all computer systems, and at many
different scales starting with parallelism in the nano-circuits that
implement individual instructions, and working the way up to parallel
systems that occupy large data centers.  Since the early 2000s
hardware manufacturers have been placing multiple processing units,
often called ``cores'', onto a single chip.  These cores can be
general purpose processors, or more special purpose processors, such as
those found in~\defn{Graphics Processing Units} (GPUs).  Each core can
run in parallel with the others.  

At the larger scale many such parallel chips or computers can be
connected by a network and used together to solve large problems.  For
example, when you perform a simple search on the Internet, you engage
a data center with thousands of computers in some part of the world,
likely near your geographic location.
%
Many of these computers (perhaps as many as hundreds, if not
thousands) take up your query and sift through data to give you an
accurate response as quickly as possible.


%% \begin{teachask}
%% Do you know how many computers are engaged in answering a typical web
%% search? 
%% \end{teachask}
%% %

%% \begin{teachask}
%% What is the advantage of using a parallel algorithm instead of a
%% sequential one?
%% \end{teachask}


%% \begin{teachask}
%% Do you know how much energy it takes to run a computation twice as
%% fast using a sequential computer (one line of computation)? 
%% \end{teachask}

There are several reasons for why such parallel systems and thus
parallelism has become so prevalent.
%


First, parallelism is simply faster than sequential computing (where
only one computation can be run at a time).
%
For example, an Internet search is not quite effective if it cannot be
completed at ``interactive speeds'', e.g., completing under a second.
%
Similarly, a weather-forecast simulation is essentially useless if it
cannot be completed in time.

The second reason is efficiency in terms of energy usage.
%
As it turns out, performing a computation twice as fast sequentially
requires eight times as much energy.  
%
Precisely speaking, energy
consumption is a cubic function of clock frequency (speed).  With
parallelism we don't need more energy to speed up a computation. 
%
For example, to perform a computation in half the time, we need to
divide the computation into two parallel sub-computations, perform
them in parallel and combine their results.  This can require as
little as half the time as the sequential computation while consuming
the same amount of energy.
%
In reality, there are some overheads and we will need more energy, for
example, to divide the computation and combine the results.
Such overheads are usually small, e.g., constant fraction over
sequential computation, but can be larger. 

These two factors---time and energy---have become increasingly
important in the last decade, catapulting parallelism to the forefront
of computing.

\paragraph{Teaching question}
Can you think of consequences of these developments in hardware?  

Answer:
These developments in hardware make the specification, the design, and
the implementation of parallel algorithms an important topic.

\paragraph{Example}
As is historically popular in explaining algorithms, we can establish
an analogy between parallel algorithms and cooking.  As in a kitchen
with multiple cooks, in parallel algorithms you can do things in
parallel for faster turnaround time.  For example, if you want to
prepare 3 dishes with a team of cooks you can do so by asking each
cook to prepare one.
%
Doing so will often be faster that using one cook.  But there are some
overheads, for example, the work has to be divided as evenly as
possible.  Obviously, you also need more resources, e.g., each cook
might need their own kitchen utensils.

\paragraph{}
One way to quantify the advantages or parallelism is to compare
its performance to sequential computation.
%
The example below illustrates the sort of performance improvements
that can achieved today.  
%
%
These times are on a 32 core commodity
server machine.  In the table, the sequential timings use sequential
algorithms while the parallel timings use parallel algorithms.  Notice
that the~\defn{speedup} for the parallel 32 core version relative to
the sequential algorithm ranges from approximately 12 (minimum
spanning tree) to approximately 32 (sorting).

\paragraph{Example}
Sample timings (reported in seconds) for some algorithms.
  \begin{center}
  \begin{tabular}{l  c c c}
    \toprule
    \textbf{Application} & \textbf{Sequential} & \textbf{Parallel} &
    \textbf{Parallel}
\\
     & & \textbf{P = 1} & \textbf{P = 32}
\\
    \midrule
    Sort $10^7$ strings &        2.9 &  2.9 &  .095\\
    Remove duplicates for $10^7$ strings &      .66 &  1.0 & .038\\
    Minimum spanning tree for $10^7$ edges    &    1.6 & 2.5  & .14\\
    Breadth first search for $10^7$ edges  &   .82  & 1.2 &  .046\\
    \bottomrule
  \end{tabular}
  \end{center}

\paragraph{Teaching question}
But why after all, do we have to do anything differently to take
advantage of parallelism?

\paragraph{Challenges of Parallel Software}
It would be convenient to use sequential algorithms on parallel
computers, but this does not work well because parallel computing
requires a different way of organizing the computation.
%
The fundamental difference between sequential and parallel computation
is that in the latter certain computations will be performed at the
same time but this is possible only if the computations are actually~\defn{independent}, i.e., do not depend on each other.
%
Thus when designing a parallel algorithm, we have to identify the
underlying dependencies in the computation and avoid creating
unnecessary dependencies.

\paragraph{Example}
Going back to our cooking example, suppose that we want to make a
frittata in our kitchen with 4 cooks.
%
Making a frittata is not easy.
%
It involves cleaning and chopping vegetables, beating eggs,
sauteeing, as well as baking.
%
For the frittata to be good, the cooks must follow a specific receipe
and pay attention to the dependencies between various tasks.
%
For example,
%
vegetables cannot be sauteed before they are washed and chopped
%
the eggs cannot be fisked before they are broken, etc.
%

\paragraph{}
An important challenge is therefore to design algorithms that minimize
the dependencies so that more can be run in parallel.
%
This design challenge is a primary focus of this book. 

Another important challenge concerns the coding and usage of a
parallel algorithm in the real world.
%
The many forms of parallelism, ranging from small to large scale, and
from general to special purpose, have led to many different programming
languages and system for coding parallel algorithms.
%
These different programming languages and systems often target a
particular kind of hardware, and even a particular kind of problem
domain.  
%
For example, there are separate systems for coding parallel numerical
algorithms on shared memory hardware, for coding graphics algorithms
on Graphical Processing Units (GPUs), and for coding data-analytics
software on a distributed system.
%
Each such system tends to have its own programming interface, its own
cost model, and its own optimizations, making it practically
impossible to take a parallel algorithm and code it once and for all
for all possible applications.
%
As it turns out, one can easily spend weeks or even months optimizing a
parallel sorting algorithm on specific parallel hardware, such as a GPU.
%% For example, it is unlikely that unoptimized code can obtain the
%% speedups discussed in \exref{intro::example-runs}.
%% %


%% The diversity of parallel hardware and software makes it difficult to
%% learn both the high-level ideas of developing parallel algorithms and
%% the optimization techniques required to achieve efficiency on a variety
%% of different machines.
%% %
%% For example, would a particular parallel sorting algorithm be
%% theoretically efficient on a large-scale system? How about on a small
%% scale system?  What would be needed to implement an optimized parallel
%% sorting algorithm for a GPU?

Maximizing speedup by coding and optimizing an algorithm is not the
goal of this book.
%
Instead, our goal is to cover general design principles for parallel
algorithms that can be applied in essentially all parallel systems,
from the data center to the multicore chips on mobile phones.
%
We will learn to think about parallelism at a high-level, learning
general techniques for designing parallel algorithms and data
structures, and learning how to approximately analyze their costs.
%
The focus is on understanding when things can run in parallel, and
when not due to dependencies.  
%
There is much more to learn about parallelism, and we hope you
continue studying this subject.



\subsection{Work and Span}

\paragraph{}
This section describes the two measures---work and span---that we use
to analyze algorithms.  Together these measures capture both the
sequential time and the parallelism available in an algorithm.
%
We typically analyze both of these asymptotically, using
for example the big-O notation.
%


\paragraph{Work}
The~\defn{work} of an algorithm corresponds to the total number of
primitive operations performed by an algorithm.  If running on a
sequential machine, it corresponds to the sequential time.
%
On a parallel machine, however, work can be divided among multiple
processors and thus does not necessarily correspond to time.
%

The interesting question is to what extent can the work be divided and
performed in parallel.  Ideally we would like to divide the work
evenly.  If we had $W$ work and $P$ processors to work on it in
parallel, then even division would give each processor $\frac{W}{P}$
fraction of the work, and hence the total time would be $\frac{W}{P}$.
%
An algorithm that achieves such ideal division is said to
have~\defn{perfect speedup}.  Perfect speedup, however, is not always
possible.
%

\paragraph{Example}
A fully sequential algorithm, where each operation depends on prior
operations leaves no room for parallelism.
%
We can only take advantage of one processor and the time would not be
improved at all by adding more.  
%

More generally, when executing an algorithm in parallel, we cannot
break dependencies, if a task depends on another task, we have to
complete them in order.

\paragraph{Teaching question}
Can you come up with an example where perfect speedup is not possible?

\paragraph{Teaching note}
For example, when cooking a frittata, you cannot cook the egg that is
not broken, nor can we add the eggs to the pan until the vegetables
are sauteed.

\paragraph{Span}
The second measure,~\defn{span}, enables analyzing to what extent the
work of an algorithm can be divided among processors.  The~\defn{span}
of an algorithm basically corresponds to the longest sequence of
dependences in the computation.  It can be thought of as the time an
algorithm would take if we had an unlimited number of processors on an
ideal machine.

\paragraph{Example: Parallel Merge Sort}
As an example, consider the parallel $\cd{mergeSort}$ algorithm for
sorting a sequence of length $n$.  The work is the same as the
sequential time, which you might know is
\[
W(n) = O(n \lg{n}).
\] 
%
%In Chapter~\ref{ch:divide-and-conquer} 
We will see that the span for
$\cd{mergeSort}$ is
\[
S(n) = O(\lg^2{n}).
\]

Thus, when  sorting a million keys and ignoring constant factors, 
work is $10^6\lg (10^6) > 10^7$, and 
%
span is 
$\lg^2(10^6) < 500.$
%

\paragraph{Parallel Time}
Even though work and span, are abstract measures of real costs, they
can be used to predict the run-time on any number of processors.
%
Specifically, if for an algorithm the work dominates, i.e., is much
larger than, span, then we expect the algorithm to deliver good
speedups.


\paragraph{Exercise}
How would you expect the parallel mergesort algorithm, $\cd{mergeSort}$,
mentioned in the example above to perform as we increase the number of
processors dedicated to running it?

\paragraph{Solution}
Recall that the work of parallel merge sort is $O(n\lg{n})$, whereas
the span is $O(\lg^2{n})$.  
%
Since span is much smaller than the work, we would expect to get good
(close to perfect) speedups when using a small to moderate number of
processors, e.g., couple of tens or hundreds, because the work term
will dominate.
%
We would expect for example the running time to halve when we double
the number of processors.
%
We should note that in practice, speedups tend to be more conservative
due to natural overheads of parallel execution and due to other
factors such as the memory subsystem that can limit parallelism.

\paragraph{Teaching question}
How do we calculate the work and span of an algorithm?


\paragraph{Definition: Work and Span}
We calculate the work and span of algorithms in a very
simple way that just involves composing costs across subcomputations.
%
Basically we assume that sub-computations are either composed
sequentially (one must be performed after the other) or in parallel
(they can be performed at the same time).
%
We then calculate the work as the sum of the work of the
subcomputations and the span as the sum of the span of sequential
subcomputations or maximum of the work of the parallel
subcomputations.
%
More concretely, given two subcomputations with work $W_1$ and $W_2$
and span $S_1$ and $S_2$, we can calculate the work and the span of
their sequential and parallel composition as follows.
%
In calculating the overall work and span, the unit cost $1$ accounts
for the cost of (parallel or sequential) composition.


\begin{center}
\renewcommand{\arraystretch}{1.5}
\begin{tabular}{lcc}
\toprule
                          &  \bf $W$ (Work) & \bf $S$ (span)\\
\midrule
\bf Sequential composition & $1 + W_1 + W_2$ & $1 + S_1+ S_2$\\
\midrule
\bf Parallel composition   & $1 + W_1 + W_2$ & $1 + \max(S_1, S_2)$\\
\bottomrule
\end{tabular}
\end{center}

\paragraph{Note}
The intuition behind the definition of work and span is that work
simply adds, whether we perform computations sequentially or in
parallel.  The span, however, only depends on the span of the maximum
of the two parallel computations.  It might help to think of work as
the total energy consumed by a computation and span as the minimum
possible time that the computation requires.  Regardless of whether
computations are performed serially or in parallel, energy is equally
required; time, however, is determined only by the slowest
computation.

\paragraph{Example}
Suppose that we have $30$ eggs to cook using $3$ cooks.  Whether all
$3$ cooks to do the cooking or just one, the total work remains
unchanged: $30$ eggs need to be cooked.
%
Assuming that cooking an egg takes $5$ minutes, the total work
therefore is $150$ minutes.
%
The span of this job corresponds to the longest sequence of
dependences that we must follow.
%
Since we can, in principle, cook all the eggs at the same time, 
span is 5 minutes.
%

Given that we have $3$ cooks, how much time do we actually need?
%
The greedy scheduling principle tells us that we need no more that
$150/3 + 5 = 55$ minutes. That is almost a factor $3$ speedup over the
$150$ that we would need with just one cook. 
%

How do we actually realize the greedy schedule?  In this case, this is
simple, all we have to do is divide the eggs equally between our
cooks.

\paragraph{}
If algorithm $A$ has less work than algorithm $B$, but has greater
span then which algorithm is better?  In analyzing sequential
algorithms there is only one measure so it is clear when one algorithm
is asymptotically better than another, but now we have two measures.
In general the work is more important than the span.  
%
This is because the work reflects the total cost of the computation
(the processor-time product).  Therefore typically the goal is to
first reduce the work and then reduce the span by designing
asymptotically work-efficient algorithms that perform no work
than the best sequential algorithm for the same problem. 
%
However, sometimes it is worth giving up a little in work to gain a
large improvement in span.
%


\paragraph{Definition: Work Efficiency}
We say that a parallel algorithm is~\defn{(asymptotically) work
  efficient}, if the work is asymptotically the same as the time for
an optimal sequential algorithm that solves the same problem.

\paragraph{Example}
The parallel $\cd{mergeSort}$ function described in
\exref{intro::mergesort-cost} is work efficient since it does $O(n \log n)$
work, which optimal time for comparison based sorting.  In this course
we will try to develop work-efficient or close to work-efficient
algorithms.



\subsection{Specification, Problem, Implementation}

\paragraph{}
Problem solving in computer science requires reasoning precisely about
problems being studied and the properties of solutions.
%
To facilitate such reasoning,  we define problems by
specifying them and describe the desired properties of solutions at
different levels of abstraction, such as the cost and the
implementation  of the solution.

In this book, we are usually interested in two distinct classes of
problems: algorithms or algorithmic problems and data structures
problems.

\paragraph{Algorithm Specification}
We specify an algorithm by describing what is expected of the
algorithm via an~\defn{algorithm specification}.
%
For example, we can specify a sorting algorithm for sequences with
respect to a given comparison function as follows.

\paragraph{Definition: Comparison Sort}
Given a sequence $A$ of $n$ elements taken from a totally ordered set
with comparison operator $\leq$, return a comparison-sorting algorithm
sequence $B$ containing the same elements but such that $B[i] \leq
B[j]$ for $0 \leq i < j < n$.

\paragraph{Note}
The specification describes~\defn{what} the algorithm should do but it
does not describe~\defn{how} it achieves what is asked.
%
This is intentional---and is exactly the point---because there can be
many algorithms that meet a specification.
%

\paragraph{}
A crucial property of any algorithm is its resource requirements or
its~\defn{cost}.
%
For example, of the many ways algorithms for sorting a sequence, we
may prefer some over the others.  
%
We specify the cost of class of algorithms with a~\defn{cost
  specification}.  For example, the following cost specification
states that a particular class of parallel sorting algorithms performs
$O(n \log{n})$ work and $O(\log^2{n})$ span.

\paragraph{Cost specification: Comparison Sort: Efficient \& Parallel}
Assuming the comparison function $<$ does constant work, the cost for
parallel comparison sorting a sequence of length $n$ is $O(n \log n)$
work and $O(\log^2 n)$ span.

\paragraph{}
There can be many cost specifications for sorting.  For example, if we
are not interested in parallelism, we can specify $O(n \log{n})$ work
but no bounds on the span. There is another cost specification below
that requires even smaller span but allows for more work.
%
We usually care more about work and thus would prefer the first cost
specification; there might, however, be cases where the second
specification is preferable.

\paragraph{Cost specification: Comparison Sort: Inefficient but Parallel}
Assuming the comparison function $<$ does constant work, the cost for
parallel comparison sorting a sequence of length $n$ is $O(n^2)$ work
and $O(\log n)$ span.

\paragraph{Data Structure Specification}
We specify a data structure by describing what is expected of the data
structure via an~\defn{Abstract Data Type (ADT) specification}.
%
As with algorithms, we usually give cost specifications to data
structures.
%
For example, we can specify a priority queue ADT and give it a cost
specification. 
%

\paragraph{Datatype: Priority Queue}
A priority queue consists of a priority queue type and supports three
operations on values of this type.  The operation $\cd{empty}$ returns
an empty queue.  The operation $\cd{insert}$ inserts a given value with
a priority into the queue and returns the queue.  The operation
$\cd{removeMin}$ removes the value with the smallest priority from the
queue and returns it.

\paragraph{Cost specification: Priority Queue: Basic}
The work and span of a priority queue operations are as follows.
\begin{itemize}
\item $\cd{create}$: $O(1)$, $O(1)$.
\item $\cd{insert}$: $O(\log{n})$, $O(\log{n})$.
\item $\cd{removeMin}$: $O(\log{n})$, $O(\log{n})$.
\end{itemize}

\paragraph{Problem}
A~\defn{problem} requires meeting an algorithm or an ADT specification
and a corresponding cost specification.
%
Since we allow specifying algorithms and data structures, we can
distinguish between algorithms problems and data-structure problems.
%

An~\defn{algorithmic problem} or an~\defn{algorithms problem}
requires designing an algorithm that satisfies the given algorithm
specification and cost specification if any.
%

A~\defn{data-structures problem} requires meeting an ADT
specification by designing a data structure that can
support the desired operations with the required efficiency specified
by the cost specification.
%

The difference between an algorithmic problem and a data-structures
problem is that the latter involves designing a data structure and a
collection of algorithms, one for each operation, that operate on that
data structure.

\paragraph{Note}
When we consider problems, it is usually clear from the context
whether we are talking about algorithms or data structures.
%
In such cases, we use the simpler terms~\defn{specification}
and~\defn{problem} to refer to the algorithm/ADT specification and the
corresponding problem respectively.

\paragraph{Implementation}
We can solve an algorithms or a data-structures problem by presenting
an~\defn{implementation}.  
%
The term~\defn{algorithm} refers to an implementation that solves an
algorithms problem and the term~\defn{data structure} to
refer to an implementation that solves a data-structures problem.
%%

\paragraph{Note}
The distinction between problems and algorithms is common in the
literature but the distinction between abstract data types and data
structures is less so.

\paragraph{Teaching note}
Why do we think this distinction is important?

\paragraph{}
We describe an algorithm by using the pseudo-code notation based on
\pml, the language  used in this book.  For example, we can specify
the classic insertion sort algorithm as follows.
%

\[
\begin{array}{l}
\cdfun{insSort}~f~s =
\\ 
~~\cd{if}~|s| = 0~\cd{then} 
\\
~~~~\cseq{}
\\
~~\cd{else}~\cdfun{insert}~f~s[0]~(\cdfun{insSort}~f~(s[1,...,n-1]))
\end{array}
\]


In the algorithm, $f$ is the comparison function and $s$ is the input
sequence. %
The algorithm uses a function ($\cd{insert}~f~x~s$) that
takes the comparison function $f$, an element $x$, and a sequence $s$
sorted by $f$, and inserts $x$ in the appropriate place.  
%
Inserting into a sorted sequence is itself an algorithmic problem,
since we are not specifying how it is implemented, but just specifying
its functionality.  
%
We might also be given a cost specification for $\cd{insert}$, e.g., for
a sequence of length $n$ the cost of $\cd{insert}$ should be $O(n)$
work and $O(\log n)$ span.  
%
Given this cost we can determine the overall asymptotic cost of
$\cd{sort}$ using our composition rules described in the last section.
%
Since the code uses $\cd{insert}$ sequentially and since there are $n$
inserts, the algorithm $\cd{insSort}$ has $n \times O(n) =
O(n^2)$ work and $n \times O(\log n) = O(n \log n)$ span.

\paragraph{}
Similarly, we can specify a data structure by specifying the data type
used by the implementation, and the algorithms for each operation.
%
For example, we can implement a priority queue with a binary heap data
structure and describe each operation as an algorithm that operates on
this data structure.  In other words, a data structure can be viewed
as a collection of algorithms that operate on the same organization
of the data.

\paragraph{Remark: On the importance of specification}
Several reasons underline the importance of distinguishing between
specification and implementation.  

First, we want to be able to use a specification without knowing the
details of an implementation that matches that specification.
%
In many cases the specification of a problem is quite simple, but an
efficient algorithm or data structure that solves it, i.e., the
implementation, is complicated.  
%
Specifications allow us abstract from implementation details.
%

Second, we want to be able to change or improve implementations over
time.  As long as each implementation matches the same specification,
and the user relied only on the specification, then he or she can
continue using the new implementation without worrying about their
code breaking. 
%

Third, when we compare the performance of different
algorithms or data structures it is important that we are not
comparing apples with oranges.  We have to make sure the algorithms we
compare are solving the same problem, because subtle differences in
the problem specification can make a significant difference in how
efficiently that problem can be solved.
